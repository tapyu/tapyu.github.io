\documentclass{article}
\pagenumbering{gobble}

% redefine \maketitle
\makeatletter % changes the catcode of @ to 11
\def\@maketitle{%
  \newpage
  \null
  \vskip 2em%
  \begin{center}%
  \let \footnote \thanks
    {\LARGE \@title \par}%
    \vskip 1.5em%
    {\large
      \lineskip .5em%
      \begin{tabular}[t]{c}%
        \@author\\
      \end{tabular}\par}%
    \vskip 1em%
    {\large {\tt Version:}\@date}%
  \end{center}%
  \par
  \vskip 1.5em}
\makeatother % changes the catcode of @ back to 12

\usepackage{authblk}
%%%%%%%%% new commands %%%%%%%%%
\usepackage{xparse} % make command behave differently depending on the number of arguments
\usepackage{xstring} % testing a string's contents, extracting substrings, substitution of substrings

%%%%%%%%% items, tables and figs %%%%%%%%%
\makeatletter
\@ifclassloaded{beamer}{}{\usepackage{enumitem}}
\makeatother
\usepackage{xltabular}
\usepackage{float} % use the [H] option
\usepackage{tikz}
\usepackage{graphicx} % extends the basic LaTeX graphic capabilities: placement, scaling, rotation, file formats
\usepackage{standalone} % place tikz environments or other material in own source files

%%%%%%%%% references %%%%%%%%%
\makeatletter
\@ifclassloaded{beamer}{}{
	\usepackage[colorlinks=true]{hyperref} % make references blue and clickable
	\hypersetup{ % you can use allcolors=blue in hyperref options instead this setup. Color it as black to disable it
		linkcolor=blue,          % For \gls{} entries
		citecolor=blue,          % For \cite{} entries
	}
}
\makeatother
\usepackage{glossaries} % \newglossaryentry{} entries. If you want to make the entries nonclickable, load this package before hyperref
\usepackage[nonumberlist]{glossaries-extra}
%%%%%%%%% bibliography %%%%%%%%%
\usepackage[style=numeric-comp,backend=biber]{biblatex} % tells LaTeX to use the biber tool instead of the traditional bibtex tool for sorting and formatting your bibliography. Biber is a more advanced and powerful tool than bibtex and it provides more features, such as advanced sorting, Unicode support and support for various data sources.
\addbibresource{/home/tapyu/.cache/zotero/refs.bib} % add reference file, in the document, add \printbibliography to print bibliography

%%%%%%%%% math %%%%%%%%%
\usepackage{amsmath}
\usepackage{mathtools} % is an extension package to amsmath to use \DeclarePairedDelimiter
\usepackage{amsfonts}
\usepackage{amssymb}
\usepackage{esint} % for \oiint
\usepackage{newtxmath} % for Greek variants (bold, nonitalic, etc...)
\usepackage{stmaryrd} % provides new symbols, such as \llbracket \rrbracket
\usepackage{bm} % for writing tensor, e.g., $\bm{\mathcal{A}}$
\usepackage{IEEEtrantools} % provides the IEEEeqnarray environment. WARNING: this package may crash your build if the \documentclass is IEEE-like, e.g., IEEEtran. In this case, delete this line!
\usepackage{siunitx} % typesetting units and unitless numbers SI (Système International d’Unités) units (PS: use \SI{} instead \qty{} to avoid conflicts with the same command from the package physics)
% use \SI for number and unit, e.g., \SI{10}{\meter\per\second}
% use \si for unit only, e.g., \si{\meter\per\second}
% use \SIrange[options]{value1}{value2}{unit commands} for a range of values, possible options:
% 1. "range-phrase=-" ->  change the word "to" to "--"
% 2. range-units=single -> chage "1m to 10m" to "1 to 10m"
% 3. mode=text % writes the unit in full
% use the [per-mode=symbol] option in the \si or \SI command to change the writing of \per, or change the line below to set it globally
\sisetup{per-mode=symbol}
% \squared -> ^2

\usepackage{physics} % several math macros, such as
% \tr -> trace
% \rank -> rank
% \expval -> angle brackets to inner product,〈a,b〉, or ensemble average〈s(t)〉
% \abs -> |x|
% \norm -> ||x||
% \eval -> evaluated bar x|_y=a
% \order -> Big-O notation
% \Re -> real
% \Im -> imaginary
% \dd[]{} -> differential
% \dv -> derivative
% \pdv -> partial derivative

%%%%%%%%% operators %%%%%%%%%
\let\oldemptyset\emptyset % change empty set
\let\emptyset\varnothing
% circular convolution a lá Oppenheim (if possible, prefer \circledast)
\newcommand*\circconv[1]{%
	\begin{tikzpicture}
		\node[draw,circle,inner sep=1pt] {#1};
	\end{tikzpicture}}
\newcommand{\intersection}{\bigcap\limits} % intersection operator

%%%%%%%%% delimiters %%%%%%%%%
\makeatletter % changes the catcode of @ to 11
\DeclarePairedDelimiter\ceil{\lceil}{\rceil} % ⌈x⌉
\let\oldceil\ceil
\def\ceil{\@ifstar{\oldceil}{\oldceil*}} % swap the asterisk and the nonasterisk behaviors

\DeclarePairedDelimiter\floor{\lfloor}{\rfloor} % ⌊x⌋
\let\oldfloor\floor
\def\floor{\@ifstar{\oldfloor}{\oldfloor*}}
\makeatother % changes the catcode of @ back to 12

%%%%%%%%% simple functions %%%%%%%%%
\newcommand{\adj}[1]{\ensuremath{\operatorname{adj}\left(#1\right)}} % adjugate matrix
\newcommand{\cof}[1]{\ensuremath{\operatorname{cof}\left(#1\right)}} % cofactor matrix
\newcommand{\eig}[1]{\ensuremath{\operatorname{eig}\left(#1\right)}} % eigenvalues
\newcommand{\nullspace}[1]{\ensuremath{\operatorname{N}\left(#1\right)}} % nullspace or kernel of the matrix
\newcommand{\nullity}[1]{\ensuremath{\operatorname{nullity}\left(#1\right)}} % nullity=dim(N(A))
\newcommand{\spn}[1]{\ensuremath{\operatorname{span}\left\{#1\right\}}} % span of a set of vectors, \span is a reserved word
\newcommand{\range}[1]{\ensuremath{\operatorname{C}\left(#1\right)}} % range or columnspace of a matrix=span(a1,a2, ..., an), where ai is the ith column vector of the matrix A
\newcommand{\unvec}[1]{\ensuremath{\operatorname{unvec}\left(#1\right)}} % unvectorize operator
\newcommand{\diag}[1]{\ensuremath{\operatorname{diag}\left(#1\right)}} % diagonal operator
\newcommand{\dom}[1]{\ensuremath{\operatorname{dom}\left(#1\right)}} % domain of the function
\newcommand{\frob}[1]{\ensuremath{\norm{#1}_\textrm{F}}} % Frobenius norm
\renewcommand{\dim}[1]{\ensuremath{\operatorname{dim}\left(#1\right)}} % dimension of a set

%%%%%%%%% complex functions %%%%%%%%%
\NewDocumentCommand{\argmin}{ e{_} m }{% \argmin_{x ∈ A}{f(x)} or \argmin{f(x)}
	\ensuremath{%
		\IfValueTF{#1}{%
			\underset{#1}{\operatorname{arg\,min}\;#2}
		}{% else
			\operatorname{arg\,min}\;#2
		}
	}
}
\NewDocumentCommand{\argmax}{ e{_} m }{% \argmax_{x ∈ A}{f(x)} or \argmax{f(x)}
	\ensuremath{%
		\IfValueTF{#1}{%
			\underset{#1}{\operatorname{arg\,max}\;#2}
		}{% else
			\operatorname{arg\,max}\;#2
		}
	}
}
% limit in the mean (see Van Tress, chapter 6)
\NewDocumentCommand{\limean}{ E{_}{{}} m }{ % \limean_{x ∈ A}{f(x)} or \limean{f(x)}
	\ensuremath{
		\IfValueTF{#1}{%
			\underset{#1}{\operatorname{l.i.m.}\;#2}
		}{% else
			\operatorname{l.i.m.}\;#2
		}
	}
}
% \E{A}, \E_{u}{A}
% if you want only the left or right part of \E (useful for breaking line) you can use \E_{u}[left-only]{A} or \E_{u}[right-only]{A}, respectively
\NewDocumentCommand{\E}{ E{_}{{}} o m }{%
\ensuremath{%
\IfValueTF{#2}{%
\IfStrEq{#2}{left-only}{%
\operatorname{E}_{#1}\left[#3\right.%
}{% right-only
\left.#3\right]%
}%
}{%
\operatorname{E}_{#1}\left[#3\right]%
}%
}
}
\NewDocumentCommand{\cov}{O{} m}{\ensuremath{\operatorname{cov}_{#1}\left[#2\right]}} % covariance, e.g., \cov[u]{x} or \cov{x}
\let\var\undefined\NewDocumentCommand{\var}{O{} m}{\ensuremath{\operatorname{var}_{#1}\left[#2\right]}} % variance, e.g., \var[u]{x} or \var{x}
\let\vec\undefined\NewDocumentCommand{\vec}{O{} m}{\ensuremath{\operatorname{vec}_{#1}\left[#2\right]}} % vectorize operator, e.g., \vec[u]{\mathbf{A}}, [u] is optional
% normalize all trigonometry functions (new and predefined) to put a parenthesis around the input argument
\RenewDocumentCommand{\sin}{s m}{% \sin{x} -> sin(x) \sin*{x} -> sin x
	\ensuremath{
		\IfBooleanTF{#1}{\operatorname{sin} #2}{\operatorname{sin}\left(#2\right)}%
	}
}
\RenewDocumentCommand{\cos}{s m}{% \cos{x} -> cos(x) \cos*{x} -> cos x
	\ensuremath{
		\IfBooleanTF{#1}{\operatorname{cos} #2}{\operatorname{cos}\left(#2\right)}%
	}
}
\RenewDocumentCommand{\tan}{s m}{% \tan{x} -> tan(x) \tan*{x} -> tan x
	\ensuremath{
		\IfBooleanTF{#1}{\operatorname{tan} #2}{\operatorname{tan}\left(#2\right)}%
	}
}
\NewDocumentCommand{\sinc}{s m}{% \sin{x} -> sinc(x) \sinc*{x} -> sinc x
	\ensuremath{
		\IfBooleanTF{#1}{\operatorname{sinc} #2}{\operatorname{sinc}\left(#2\right)}%
	}
}
\NewDocumentCommand{\sgn}{s m}{% \sgn{x} -> sgn(x) \sng*{x} -> sgn x
	\ensuremath{
		\IfBooleanTF{#1}{\operatorname{sng} #2}{\operatorname{sgn}\left(#2\right)}%
	}
}

%%%%%%%%% minor adjusts in some mathematical symbols %%%%%%%%%
\DeclareMathAlphabet{\mathcal}{OMS}{cmsy}{m}{n} % newtxmath changes the font style of \mathcal. It prevents \mathcal from being changed
\let\oldforall\forall % put some spaces between the \forall command
\renewcommand{\forall}{\;\oldforall\;}

%%%%%%%%% comments, corrections, or revisions %%%%%%%%%
\usepackage{luacolor} % color support based on LuaTEX’s
\usepackage[soul]{lua-ul} % provide underlining, strikethrough, and highlighting using features in LuaLATEX which avoid the restrictions imposed by other methods
\newcommand{\obs}[1]{\textcolor{red}{(#1)}} % comment
\newcommand{\sizecorr}[1]{\makebox[0cm]{\phantom{$\displaystyle #1$}}} % Used to seize the height of equation
\newcommand{\ensureoperation}{\negmedspace {}} % To ensure that a new line symbol is an operation instead of a sign
\usepackage{esint}
\NewDocumentCommand{\nossekE}{O{} m}{\ensuremath{\operatorname{\bf E}_{#1}\left[#2\right]}} % statistical expectation operator, e.g., \E[u]{x} or \E{x}

\begin{document}
\title{\textbf{Notation}  \vspace{-.3cm}}
\author{Rubem Vasconcelos Pacelli\\
  {\tt rubem.engenharia@gmail.com}}
\affil{Department of Teleinformatics Engineering,\\Federal University of Ceará.\\Fortaleza, Ceará, Brazil. \vspace{-.5cm}}
\maketitle

\section{Font notation}
\begin{xltabular}{\textwidth}{XX}
  $a,b,c, \dots, A, B, C, \dots$ & Scalars \\ \hline
  $\mathbf{a}, \mathbf{b}, \mathbf{c}, \dots$ & Vectors \\ \hline
  $\mathbf{A}, \mathbf{B}, \mathbf{C}, \dots$ & Matrices \\ \hline
  $\bm{\mathcal{A}}, \bm{\mathcal{B}}, \bm{\mathcal{C}}, \dots$ & Tensors \\ \hline
  $A, B, C, \dots, \mathcal{A}, \mathcal{B}, \mathcal{C}, \dots, \mathbb{A}, \mathbb{B}, \mathbb{C}, \dots$ &  Sets\\
\end{xltabular}

\section{Signals and functions}
\subsection{Time indexing}
\begin{xltabular}{\textwidth}{p{5.7cm}X}
    \(x(t)\) & Continuous-time \(t\)\\ \hline
    \(x[n], x[k], x[m], x[i], \dots\) \(x_n, x_k, x_m, x_i, \dots\) \(x(n), x(k), x(m), x(i), \dots\) & Discrete-time \(n, k, m, i, \dots\) (parenthesis should be adopted only if there are no continuous-time signals in the context to avoid ambiguity) \\ \hline
    \(x\left[ \left( \left( n - m \right) \right)_N \right], x \left( \left( n - m \right) \right)_N\) & Circular shift in \(m\) samples within a \(N\)-samples window \cite{ingleDigitalSignalProcessing2000,oppenheimDiscreteTimeSignalProcessing2009}
\end{xltabular}
\subsection{Common functions}
\begin{xltabular}{\textwidth}{XX}
    \(\delta(t)\) & Delta function\\ \hline
    \(\delta[n], \delta_{i,j}\) & Kronecker function (\(n = i-j\))\\ \hline
    \(h(t), h[n]\) & Impulse response (continuous and discrete time)\\ \hline
    \(\tilde{x}[n], \tilde{x}(t)\) & Periodic discrete- or continuous-time signal\\ \hline
    \(\hat{x}[n], \hat{x}(t)\) & Estimate of \(x[n]\) or \(x(t)\)\\ \hline
    \(\dot{x}[m]\) & Interpolation of \(x[n]\)\\
\end{xltabular}
\subsection{Operations and symbols}
\begin{xltabular}{\textwidth}{XX}
    \(f: A \rightarrow B\)& A function \(f\) whose domain is \(A\) and codomain is \(B\)\\ \hline
    \(\mathbf{f}: A \rightarrow \mathbb{R}^n\)& A vector-valued function \(\mathbf{f}\), i.e., \(n \geq 2\)\\ \hline
    \(f^{n}, x^{n}(t), x^{n}[k]\) & \(n\)th power of the function \(f\), \(x[n]\) or \(x(t)\)\\ \hline
    \(f^{\left( n \right)},  x^{(n)}(t)\) & \(n\)th derivative of the function \(f\) or \(x(t)\)\\ \hline
    \(f', f^{\left( 1 \right)}, x'(t)\) & \(1\)th derivative of the function \(f\) or \(x(t)\)\\ \hline
    \(f'', f^{\left( 2 \right)}, x''(t)\) & \(2\)th derivative of the function \(f\) or \(x(t)\)\\ \hline
    \(\argmax[x \in \mathcal{A}]{f(x)} \) & Value of \(x\) that minimizes \(x\)\\ \hline
    \( \argmin[x \in \mathcal{A}]{f(x)} \) & Value of \(x\) that minimizes \(x\)\\ \hline
    \(f(\mathbf{x}) = \underset{\mathbf{y} \in \mathcal{A}}{\textnormal{inf }} g(\mathbf{x},\mathbf{y})\) & Infimum, i.e., \(f(\mathbf{x}) = \min{\left\{ g(\mathbf{x}, \mathbf{y}) \mid \mathbf{y} \in \mathcal{A} \wedge \left( \mathbf{x}, \mathbf{y} \right) \in \dom{g} \right\}}\), which is the greatest lower bound of this set \cite{boydConvexOptimization2004}\\ \hline
    \(f(\mathbf{x}) = \underset{\mathbf{y} \in \mathcal{A}}{\textnormal{sup }} g(\mathbf{x},\mathbf{y})\) & Supremum, i.e., \(f(\mathbf{x}) = \max{\left\{ g(\mathbf{x}, \mathbf{y}) \mid \mathbf{y} \in \mathcal{A} \wedge \left( \mathbf{x}, \mathbf{y} \right) \in \dom{g} \right\}}\), which is the least upper bound of this set \cite{boydConvexOptimization2004}\\ \hline
    \(f \circ g\) & Composition of the functions \(f\) and \(g\)\\ \hline
    \(*\) & Convolution (discrete or continuous)\\ \hline
    \(\circledast, \circconv{N}\) & Circular convolution \cite{oppenheimDiscreteTimeSignalProcessing2009,dinizDigitalSignalProcessing2010}\\
\end{xltabular}
\subsection{Transformations}
\begin{xltabular}{\textwidth}{XX}
    \(W_N\) & Twiddle factor, \(e^{-j\frac{2\pi}{N}}\) \cite{ingleDigitalSignalProcessing2000}\\ \hline
    \(\mathcal{F}\left\{ \cdot \right\}\) & Fourier transform\\ \hline
    \(\mathcal{L}\left\{ \cdot \right\}\) & Laplace transform\\ \hline
    \(\mathcal{Z}\left\{ \cdot \right\}\) & \(z\)-transform\\ \hline
    \(\hat{x}(t), \hat{x}[n]\) & Hilbert transform of \(x(t)\) or \(x[n]\)\\ \hline
    \(X(s)\) & Laplace transform of \(x(t)\)\\ \hline
    \(X(f)\) & Fourier transform (FT) (in linear frequency, \(\unit{\Hz}\)) of \(x(t)\)\\ \hline
    \(X(j\omega)\) & Fourier transform (FT) (in angular frequency, \(\unit{\radian\per\sec}\)) of \(x(t)\)\\ \hline
    \(X(e^{j\omega})\) & Discrete-time Fourier transform (DTFT) of \(x[n]\)\\ \hline
    \(X[k], X(k), X_k\) & Discrete Fourier transform (DFT) or fast Fourier transform (FFT) of \(x[n]\), or even the Fourier series (FS) of the periodic signal \(x(t)\)\\ \hline
    \(\tilde{X}[k], \tilde{X}(k), \tilde{X}_k\) & Discrete Fourier series (DFS) of \(\tilde{x}[n]\)\\ \hline
    \(X(z)\) & \(z\)-transform of \(x[n]\)\\
\end{xltabular}

\section{Probability, statistics, and stochastic processes}
\subsection{Operators and symbols}
\begin{xltabular}{\textwidth}{XX}
    \(\E{\cdot}, \nossekE{\cdot}, E\left[ \cdot \right], \mathbb{E}\left[ \cdot \right]\) & Statistical expectation operator \cite{nossekAdaptiveArraySignal2015,dinizAdaptiveFilteringAlgorithms2002}\\ \hline
    \(\E[u]{\cdot}, \nossekE[u]{\cdot},  E_u\left[ \cdot \right], \mathbb{E}_u\left[ \cdot \right]\) & Statistical expectation operator with respect to \(u\)\\ \hline
    \(\expval{\cdot}\) & Ensamble average\\ \hline
    \(\var{\cdot}, \textnormal{VAR}[\cdot]\) & Variance operator \cite{haykinAdaptiveFilterTheory2002,leon-garciaProbabilityStatisticsRandom2007,proakisDigitalCommunications2007,bishopPatternRecognitionMachine2006}\\ \hline
    \(\var[u]{\cdot} \left[ \cdot \right], \textnormal{VAR}_u[\cdot]\) & Variance operator with respect to \(u\)\\ \hline
    \(\cov{\cdot}, \textnormal{COV}[\cdot]\) & Covariance operator \cite{bishopPatternRecognitionMachine2006}\\ \hline
    \(\cov[u]{\cdot}, \textnormal{COV}_u[\cdot]\) & Covariance operator with respect to \(u\)\\ \hline
    \(\mu_x\) & Mean of the random variable \(x\) \\ \hline
    \(\boldsymbol{\muup}_\mathbf{x}, \mathbf{m}_\mathbf{x}\) & Mean vector of the random variable \(\mathbf{x}\) \cite{brownIntroductionRandomSignals1997} \\ \hline
    \(\mu_n\) & \(n\)th-order moment of a random variable \\ \hline
    \(\sigma_x^2, \kappa_2\) & Variance of the random variable \(x\)\\ \hline
    \(\mathcal{K}_x, \mu_4\) & Kurtosis (4th-order moment) of the random variable \(x\)\\ \hline
    \(\kappa_n\) & \(n\)th-order cumulant of a random variable \\ \hline
    \(\rho_{x,y}\) & Pearson correlation coefficient between \(x\) and \(y\)\\ \hline
    \(a\sim P\) & Random variable \(a\) with distribution \(P\) \\ \hline
    \(\mathcal{R}\) & Rayleigh's quotient
\end{xltabular}
\subsection{Stochastic processes}
\begin{xltabular}{\textwidth}{XX}
    \(r_x(\tau), R_x(\tau)\) & Autocorrelation function of the signal \(x(t)\) or \(x[n]\) \cite{nossekAdaptiveArraySignal2015}\\ \hline
    \(S_x(f), S_x(j\omega)\) & Power spectral density (PSD) of \(x(t)\) in linear (\(f\)) or angular (\(\omega\)) frequency\\ \hline
    \(S_{x,y}(f), S_{x,y}(j\omega)\) & Cross PSD of \(x(t)\) and \(y(t)\) in linear or angular (\(\omega\)) frequency\\ \hline
    \(\mathbf{R}_\mathbf{x}\) & (Auto)correlation matrix of \(\mathbf{x}(n)\) \\ \hline
    \(r_{x,d}(\tau), R_{x,d}(\tau)\) & Cross-correlation between \(x[n]\) and \(d[n]\) or \(x(t)\) and \(d(t)\) \cite{nossekAdaptiveArraySignal2015}\\ \hline
    \(\mathbf{R}_\mathbf{xy}\) & Cross-correlation matrix of \(\mathbf{x}(n)\) and \(\mathbf{y}(n)\)\\ \hline
    \(\mathbf{p}_{\mathbf{x}d}\)& Cross-correlation vector between \(\mathbf{x}(n)\) and \(d(n)\) \cite{dinizAdaptiveFiltering1997} \\ \hline
    \(c_x(\tau), C_x(\tau)\) & Autocovariance function of the signal \(x(t)\) or \(x[n]\) \cite{nossekAdaptiveArraySignal2015}\\ \hline
    \(\mathbf{C}_\mathbf{x}, \mathbf{K}_\mathbf{x}, \boldsymbol{\Sigmaup}_\mathbf{x}, \textnormal{cov}\left[ \mathbf{x} \right]\) & (Auto)covariance matrix of \(\mathbf{x}\) \cite{vantreesOptimumArrayProcessing2002,proakisDigitalCommunications2007,leon-garciaProbabilityStatisticsRandom2007,haykinAdaptiveFilterTheory2002} \\ \hline
    \(c_{xy}(\tau), C_{xy}(\tau)\) & Cross-covariance function of the signal \(x(t)\) or \(x[n]\) \cite{nossekAdaptiveArraySignal2015} \\ \hline
    \(\mathbf{C}_{\mathbf{xy}}, \mathbf{K}_{\mathbf{xy}}, \boldsymbol{\Sigmaup}_{\mathbf{xy}}\) & Cross-covariance matrix of \(\mathbf{x}\) and \(\mathbf{y}\)
\end{xltabular}

\subsection{Functions}
\begin{xltabular}{\textwidth}{XX}
    \(Q(\cdot)\) & \(Q\)-function, i.e., \(P\left[ \mathcal{N}(0,1) > x \right] \) \cite{proakisDigitalCommunications2007} \\ \hline
    \(\textnormal{erf}(\cdot)\) & Error function \cite{proakisDigitalCommunications2007}\\ \hline
    \(\textnormal{erfc}(\cdot)\) & Complementary error function i.e., \(\textnormal{erfc}(x) = 2Q(\sqrt{2}x) - \textnormal{erf}(x)\) \cite{proakisDigitalCommunications2007}\\ \hline
    \(P[A]\) & Probability of the event or set \(A\) \cite{leon-garciaProbabilityStatisticsRandom2007}\\ \hline
    \(p(\cdot), f(\cdot)\) & Probability density function (PDF) or probability mass function (PMF) \cite{leon-garciaProbabilityStatisticsRandom2007}\\ \hline
    \(p(x\mid A)\) & Conditional PDF or PMF \cite{leon-garciaProbabilityStatisticsRandom2007}\\ \hline
    \(F(\cdot)\) & Cumulative distribution function (CDF)\\ \hline
    \(\Phi_x(\omega), M_x(j\omega), E\left[ e^{j \omega x} \right]\) & First characteristic function (CF) of \(x\) \cite{proakisDigitalCommunications2007,theodoridisMachineLearningBayesian2020a} \\ \hline
    \(M_x(t), \Phi_x(-j t), E\left[ e^{t x} \right]\) & Moment-generating function (MGF) of \(x\) \cite{proakisDigitalCommunications2007,theodoridisMachineLearningBayesian2020a} \\ \hline
    \(\Psi_x(\omega), \ln \Phi_x(\omega), \ln E\left[ e^{j \omega x}\right]\) & Second characteristic function \\ \hline
    \(K_x(t), \ln E\left[ e^{t x} \right], \ln M_x(t)\) & Cumulant-generating function (CGF) of \(x\) \cite{haykinAdaptiveFilterTheory2002} \\
\end{xltabular}

\subsection{Distributions}
\begin{xltabular}{\textwidth}{XX}
    \(\mathcal{N}(\mu, \sigma^2)\) & Gaussian distribution of a random variable with mean \(\mu\) and variance \(\sigma^{2}\). The same notation can be used to denote a real-valued white Gaussian process with mean equal to \(\mu\) and power spectral density equal to \(N_0/2\), e.g., \(s(t) \sim \mathcal{N}(\mu, N_0/2)\)\\ \hline
    \(\mathcal{CN}(\mu, \sigma^2)\) & Complex Gaussian distribution of a random variable with mean \(\mu\) and variance \(\sigma^{2}\). The same notation can be used to denote a complex-valued white Gaussian process with mean equal to \(\mu\) and power spectral density equal to \(N_0\), e.g., \(s(t) \sim \mathcal{CN}(\mu, N_0)\)\\ \hline
    \(\mathcal{N}(\boldsymbol{\muup}, \boldsymbol{\Sigmaup})\) & Gaussian distribution of a vector random variable with mean \(\boldsymbol{\muup}\) and covariance matrix \(\boldsymbol{\Sigmaup}\)\\ \hline
    \(\mathcal{CN}(\boldsymbol{\muup}, \boldsymbol{\Sigmaup})\) & Complex Gaussian distribution of a vector random variable with mean \(\boldsymbol{\muup}\) and covariance matrix \(\boldsymbol{\Sigmaup}\)\\ \hline
    \(\mathcal{U}(a,b)\) & Uniform distribution from \(a\) to \(b\)\\ \hline
    \(\chi^2 (n), \chi^2_n\) & Chi-square distribution with \(n\) degree of freedom (assuming that the Gaussians are \(\mathcal{N}(0,1)\))\\ \hline
    \(\textnormal{Exp}(\lambda)\) & Exponential distribution with rate parameter \(\lambda\)\\ \hline
    \(\Gamma(\alpha, \beta)\) & Gamma distribution with shape parameter \(\alpha\) and rate parameter \(\beta\)\\ \hline
    \(\Gamma(\alpha, \theta)\) & Gamma distribution with shape parameter \(\alpha\) and scale parameter \(\theta\ = 1/\beta\)\\ \hline
    \(\textnormal{Nakagami}(m, \Omega)\) & Nakagami-m distribution with shape parameter \(m\) and spread parameter \(\Omega\) \\ \hline
    \(\textnormal{Rayleigh}(\sigma)\) & Rayleigh distribution with scale parameter \(\sigma\)\\ \hline
    \(\textnormal{Rayleigh}(\Omega)\) & Rayleigh distribution with the second moment \(\Omega = E\left[ x^2 \right] = 2\sigma^2\)\\ \hline
    \(\textnormal{Rice}(s, \sigma)\) & Rice distribution with noncentrality parameter (specular component) \(s\) and \(\sigma\)\\ \hline
    \(\textnormal{Rice}(A, K)\) & Rice distribution with Rice factor \(K=s^2/2\sigma^2\) and scale parameter \(A = s^2 + 2\sigma^2\)
\end{xltabular}

\section{Statistical signal processing}
\begin{xltabular}{\textwidth}{XX}
    \(\boldsymbol{\nabla}f, \mathbf{g}\) & Gradient descent vector \\ \hline
    \(\boldsymbol{\nabla}_{x}f, \mathbf{g}_{x}\) & Gradient descent vector with respect \(x\)\\ \hline
    \(\mathbf{g}\) (or \(\hat{\mathbf{g}}\) if the gradient vector is \(\mathbf{g}\)) & Stochastic gradient descent (SGD) \\ \hline
    \(J(\cdot), \mathcal{E}(\cdot)\) & Cost-function or objective function\\ \hline
    \(\Lambda(\cdot)\) & Likelihood function\\ \hline
    \(\Lambda_l(\cdot)\) & Log-likelihood function\\ \hline
    \(\hat{x}(t)\) or \(\hat{x}[n]\) & Estimate of \(x(t)\) or \(x[n]\)\\ \hline
    \(\hat{\boldsymbol{\muup}}_x, \hat{\mathbf{m}}_x\) & Sample mean of \(x[n]\) or \(x(t)\) \\ \hline
    \(\hat{\boldsymbol{\muup}}_\mathbf{x}, \hat{\mathbf{m}}_\mathbf{x}\) & Sample mean vector of \(\mathbf{x}[n]\) or \(\mathbf{x}(t)\)\\ \hline
    \(\hat{r}_x(\tau), \hat{R}_x(\tau)\) & Estimated autocorrelation function of the signal \(x(t)\) or \(x[n]\)\\ \hline
    \(\hat{S}_x(f), \hat{S}_x(j\omega)\) & Estimated power spectral density (PSD) of \(x(t)\) in linear (\(f\)) or angular (\(\omega\)) frequency\\ \hline
    \(\hat{\mathbf{R}}_\mathbf{x}\) & Sample (auto)correlation matrix \\ \hline
    \(\hat{r}_{x,d}(\tau), \hat{R}_{x,d}(\tau)\) & Estimated cross-correlation between \(x[n]\) and \(d[n]\) or \(x(t)\) and \(d(t)\)\\ \hline
    \(\hat{S}_{x,y}(f), \hat{S}_{x,y}(j\omega)\) & Estimated cross PSD of \(x(t)\) and \(y(t)\) in linear or angular (\(\omega\)) frequency\\ \hline
    \(\hat{\mathbf{R}}_\mathbf{xy}\) & Sample cross-correlation matrix of \(\mathbf{R}_\mathbf{xy}\) \\ \hline
    \(\hat{\rho}_{x,y}\) & Estimated Pearson correlation coefficient between \(x\) and \(y\)\\ \hline
    \(\hat{c}_x(\tau), \hat{C}_x(\tau)\) & Estimated autocovariance function of the signal \(x(t)\) or \(x[n]\)\\ \hline
    \(\hat{\mathbf{C}}_\mathbf{x}, \hat{\mathbf{K}}_\mathbf{x}, \hat{\boldsymbol{\Sigmaup}}_\mathbf{x}\) & Sample (auto)covariance matrix \\ \hline
    \(\hat{c}_{xy}(\tau), \hat{C}_{xy}(\tau)\) & Estimated cross-covariance function of the signal \(x(t)\) or \(x[n]\)\\ \hline
    \(\hat{\mathbf{C}}_{\mathbf{xy}}, \hat{\mathbf{K}}_{\mathbf{xy}}, \hat{\boldsymbol{\Sigmaup}}_{\mathbf{xy}}\) & Sample cross-covariance matrix \\ \hline
    \(\mathbf{w}, \boldsymbol{\thetaup}\) & Parameters, coefficients, or weights vector \\ \hline
    \(\mathbf{w}_o, \mathbf{w}^{\star}, \boldsymbol{\thetaup}_o, \boldsymbol{\thetaup}^{\star}\) & Optimum value of the parameters, coefficients, or weights vector \\ \hline
    \(\mathbf{W}\) & Matrix of the weights \\ \hline
    \(\mathbf{J}\) & Jacobian matrix\\ \hline
    \(\mathbf{H}\) & Hessian matrix \\ \hline
    \(\hat{\mathbf{H}}\) & Estimate of the Hessian matrix
\end{xltabular}

\section{Linear Algebra}
\subsection{Common matrices and vectors}
\begin{xltabular}{\textwidth}{XX}
    \(\mathbf{W}, \mathbf{D}\) & Diagonal matrix \\ \hline
    \(\mathbf{P}\) & Projection matrix; Permutation matrix \\ \hline
    \(\mathbf{J}\) & Jordan matrix \\ \hline
    \(\mathbf{L}\) & Lower matrix\\ \hline
    \(\mathbf{U}\) & Upper matrix\\ \hline
    \(\mathbf{C}\) & Cofactor matrix\\ \hline
    \(\mathbf{C}_\mathbf{A}, \cof{\mathbf{A}}\) & Cofactor matrix of \(\mathbf{A}\)\\ \hline
    \(\mathbf{S}\) & Symmetric matrix\\ \hline
    \(\mathbf{Q}\) & Orthogonal matrix\\ \hline
    \(\mathbf{I}_N\) & \(N\times N\)-dimensional identity matrix\\ \hline
    \(\mathbf{0}_{M\times N}\) & \(M\times N\)-dimensional null matrix\\ \hline
    \(\mathbf{0}_{N}\) & \(N\)-dimensional null vector\\ \hline
    \(\mathbf{1}_{M\times N}\) & \(M\times N\)-dimensional ones matrix\\ \hline
    \(\mathbf{1}_{N}\) & \(N\)-dimensional ones vector\\ \hline
    \(\mathbf{0}\) & Null matrix, vector, or tensor (dimensionality understood by context)\\ \hline
    \(\mathbf{1}\) & Ones matrix, vector, or tensor (dimensionality understood by context)\\
\end{xltabular}

\subsection{Indexing}
\begin{xltabular}[l]{\linewidth}{XX}
    \(x_{i_1,i_2, \dots, i_N}, \left[ \bm{\mathcal{X}} \right]_{i_1,i_2, \dots, i_N}\) & Element in the position \((i_1,i_2, \dots, i_N)\) of the tensor \(\bm{\mathcal{X}}\)\\ \hline
    \(\bm{\mathcal{X}}^{(n)}\) & \(n\)th tensor of a nontemporal sequence\\ \hline
    \(\mathbf{x}_{n}, \mathbf{x}_{:n}\) & \(n\)th column of the matrix \(X\)\\ \hline
    \(\mathbf{x}_{n:}\) & \(n\)th row of the matrix \(X\)\\ \hline
    \(\mathbf{x}_{i_1,\dots,i_{n-1}, :, i_{n+1},\dots, i_N}\) & Mode-\(n\) fiber of the tensor \(\bm{\mathcal{X}}\)\\ \hline
    \(\mathbf{x}_{:,i_2,i_3}\) & Column fiber (mode-\(1\) fiber) of the thrid-order tensor \(\bm{\mathcal{X}}\)\\ \hline
    \(\mathbf{x}_{i_1,:,i_3}\) & Row fiber (mode-\(2\) fiber) of the thrid-order tensor \(\bm{\mathcal{X}}\)\\ \hline
    \(\mathbf{x}_{i_1,i_2,:}\) & Tube fiber (mode-\(3\) fiber) of the thrid-order tensor \(\bm{\mathcal{X}}\)\\ \hline
    \(\mathbf{X}_{i_1,:,:}\) & Horizontal slice of the thrid-order tensor \(\bm{\mathcal{X}}\)\\ \hline
    \(\mathbf{X}_{:,i_2,:}\) & Lateral slices slice of the thrid-order tensor \(\bm{\mathcal{X}}\)\\ \hline
    \(\mathbf{X}_{i_3}, \mathbf{X}_{:,:,i_3}\) & Frontal slices slice of the thrid-order tensor \(\bm{\mathcal{X}}\)
\end{xltabular}

\subsection{General operations}
\begin{xltabular}{\textwidth}{XX}
    \(\expval{\mathbf{a}, \mathbf{b}}, \mathbf{a}^\top\mathbf{b}, \mathbf{a}\cdot\mathbf{b}\) & Inner or dot product\\ \hline
    \(\mathbf{a}\circ\mathbf{b}, \mathbf{a}\mathbf{b}^\top\) & Outer product\\ \hline
    \(\otimes\) & Kronecker product\\ \hline
    \(\odot\) & Hadamard (or Schur) (elementwise) product\\ \hline
    \(\cdot^{\odot n}\) & \(n\)th-order Hadamard power\\ \hline
    \(\cdot^{\odot \frac{1}{n}}\) & \(n\)th-order Hadamard root\\ \hline
    \(\oslash\) & Hadamard (or Schur) (elementwise) division\\ \hline
    \(\diamond\) & Khatri-Rao product\\ \hline
    \(\otimes\) & Kronecker Product\\ \hline
    \(\times_n\) & \(n\)-mode product\\
\end{xltabular}

\subsection{Operations with matrices and tensors}
\begin{xltabular}{\textwidth}{XX}
    \(\mathbf{A}^{-1}\) & Inverse matrix\\ \hline
    \(\mathbf{A}^+, \mathbf{A}^{\dagger}\) & Moore-Penrose left pseudoinverse\\ \hline
    \(\mathbf{A}^\top\) & Transpose\\ \hline
    \(\mathbf{A}^{-\top}\) & Transpose of the inverse, i.e., \(\left( \mathbf{A}^{-1} \right)^{\top} = \left( \mathbf{A}^{\top} \right)^{-1}\) \cite{petersenMatrixCookbook2008,golubMatrixComputations2013}\\ \hline
    \(\mathbf{A}^*\) & Complex conjugate\\ \hline
    \(\mathbf{A}^\mathsf{H}\) & Hermitian\\ \hline
    \(\frob{\mathbf{A}}\)& Frobenius norm \\ \hline
    \(\norm{\mathbf{A}}\) & Matrix norm\\ \hline
    \(\abs{\mathbf{A}}, \textnormal{det}\left( \mathbf{A} \right)\) & Determinant\\ \hline
    \(\diag{\mathbf{A}}\) & The elements in the diagonal of \(\mathbf{A}\) \\ \hline
    \(\vec[]{\mathbf{A}}\) &  Vectorization: stacks the columns of the matrix \(\mathbf{A}\) into a long column vector\\ \hline
    \(\vec[d]{\mathbf{A}}\) &  Extracts the diagonal elements of a square matrix and returns them
    in a column vector\\ \hline
    \(\vec[l]{\mathbf{A}}\) & Extracts the elements strictly below the main diagonal of a square matrix in a column-wise manner and returns them into a column vector\\ \hline
    \(\vec[u]{\mathbf{A}}\) & Extracts the elements strictly above the main diagonal of a square matrix in a column-wise manner and returns them into a column vector\\ \hline
    \(\vec[b]{\mathbf{A}}\) & Block vectorization operator: stacks square block matrices of the input into a long block column matrix\\ \hline
    \(\unvec{\mathbf{A}}\)& Reshapes a column vector into a matrix\\ \hline
    \(\tr{\mathbf{A}}\)& trace\\ \hline
    \(\mathbf{X}_{(n)}\) & \(n\)-mode matricization of the tensor \(\bm{\mathcal{X}}\)\\
\end{xltabular}
\subsection{Operations with vectors}
\begin{xltabular}{\textwidth}{XX}
    \(\norm{\mathbf{a}}\) & \(l_1\) norm, 1-norm, or Manhatan norm\\ \hline
    \(\norm{\mathbf{a}}, \norm{\mathbf{a}}_2\) & \(l_2\) norm, 2-norm, or Euclidean norm\\ \hline
    \(\norm{\mathbf{a}}_p\) & \(l_p\) norm, \(p\)-norm, or Minkowski norm\\ \hline
    \(\norm{\mathbf{a}}_\infty\) & \(l_\infty\) norm, \(\infty\)-norm, or Chebyshev norm\\ \hline
    \(\diag{\mathbf{a}}\) & Diagonalization: a square, diagonal matrix with entries given by the vector \(\mathbf{a}\)\\
\end{xltabular}

\subsection{Decompositions}
\begin{xltabular}{\textwidth}{XX}
    \(\boldsymbol{\Lambdaup}\) & Eigenvalue matrix \cite{strangIntroductionLinearAlgebra1993}\\ \hline
    \(\mathbf{Q}\) & Eigenvectors matrix; Orthogonal matrix of the QR decomposition\cite{strangIntroductionLinearAlgebra1993}\\ \hline
    \(\mathbf{R}\) & Upper triangular matrix of the QR decomposition\cite{strangIntroductionLinearAlgebra1993}\\ \hline
    \(\mathbf{U}\) & Left singular vectors\cite{strangIntroductionLinearAlgebra1993}\\ \hline
    \(\mathbf{U}_r\) & Left singular nondegenerated vectors\\ \hline
    \(\boldsymbol{\Sigmaup}\) & Singular value matrix\\ \hline
    \(\boldsymbol{\Sigmaup}_r\) & Singular value matrix with nonzero singular values in the main diagonal\\ \hline
    \(\boldsymbol{\Sigmaup}^{+}\) & Singular value matrix of the pseudoinverse \cite{strangIntroductionLinearAlgebra1993}\\ \hline
    \(\boldsymbol{\Sigmaup}^{+}_r\) & Singular value matrix of the pseudoinverse with nonzero singular values in the main diagonal\\ \hline
    \(\mathbf{V}\) & Right singular vectors \cite{strangIntroductionLinearAlgebra1993}\\ \hline
    \(\mathbf{V}_r\) & Right singular nondegenerated vectors\\ \hline
    \(\eig{\mathbf{A}}\) & Set of the eigenvalues of \(\mathbf{A}\) \cite{chellappaSignalProcessingTheory2014,leon-garciaProbabilityStatisticsRandom2007,petersenMatrixCookbook2008}\\ \hline
    \(\llbracket \mathbf{A}, \mathbf{B}, \mathbf{C}, \dots \rrbracket\) & CANDECOMP/PARAFAC (CP) decomposition of the tensor \(\bm{\mathcal{X}}\) from the outer product of column vectors of \(\mathbf{A}\), \(\mathbf{B}\), \(\mathbf{C}, \dots\)\\ \hline
    \(\llbracket \boldsymbol{\lambdaup}; \mathbf{A}, \mathbf{B}, \mathbf{C}, \dots \rrbracket\) & Normalized CANDECOMP/PARAFAC (CP) decomposition of the tensor \(\bm{\mathcal{X}}\) from the outer product of column vectors of \(\mathbf{A}\), \(\mathbf{B}\), \(\mathbf{C}, \dots\)\\
\end{xltabular}
\subsection{Spaces}
\begin{xltabular}{\textwidth}{XX}
    \(\spn{\mathbf{a}_1, \mathbf{a}_2, \dots, \mathbf{a}_n}\) & Vector space spanned by the argument vectors \cite{golubMatrixComputations2013}\\ \hline
    \(\range{\mathbf{A}}\), \(\mathrm{columnspace}(\mathbf{A})\), \(\mathrm{range}(\mathbf{A})\), \(\spn{\mathbf{A}}\), \(\mathrm{image}(\mathbf{A})\) & Columnspace, range or image, i.e., the space \(\spn{\mathbf{a}_1,\mathbf{a}_2, \dots, \mathbf{a}_n}\), where \(\mathbf{a}_i\) is the ith column vector of the matrix \(\mathbf{A}\) \cite{strangIntroductionLinearAlgebra1993, nossekAdaptiveArraySignal2015}\\ \hline
    \(\range{\mathbf{A}^\mathsf{H}}\) & Row space (also called left columnspace) \cite{strangIntroductionLinearAlgebra1993, nossekAdaptiveArraySignal2015}\\ \hline
    \(\nullspace{\mathbf{A}}, \mathrm{nullspace}(\mathbf{A}), \mathrm{null}(\mathbf{A}), \mathrm{kernel}(\mathbf{A})\) & Nullspace (or kernel space) \cite{strangIntroductionLinearAlgebra1993, nossekAdaptiveArraySignal2015,theodoridisMachineLearningBayesian2020}\\ \hline
    \(\nullspace{\mathbf{A^\mathsf{H}}}\) & Left nullspace\\ \hline
    \(\rank{\mathbf{A}}\) & Rank, that is, \(\dim{\spn{\mathbf{A}}} = \dim{\range{\mathbf{A}}}\) \cite{nossekAdaptiveArraySignal2015} \\ \hline
    \(\nullity{\mathbf{A}}\) & Nullity of \(\mathbf{A}\), i.e., \(\dim{\nullspace{\mathbf{A}}}\)\\ \hline
    \(\mathbf{a} \perp \mathbf{b}\) & \(\mathbf{a}\) is orthogonal to \(\mathbf{b}\)\\ \hline
    \(\mathbf{a} \not\perp \mathbf{b}\) & \(\mathbf{a}\) is not orthogonal to \(\mathbf{b}\)\\
\end{xltabular}
\subsection{Inequalities}
\begin{xltabular}{\textwidth}{XX}
    \(\bm{\mathcal{X}} \leq 0\) & Nonnegative tensor\\ \hline
    \(\mathbf{a} \preceq_K \mathbf{b}\) & Generalized inequality meaning that \(\mathbf{b}-\mathbf{a}\) belongs to the conic subset \(K\) in the space \(\mathbb{R}^{n}\)\cite{boydConvexOptimization2004}\\ \hline
    \(\mathbf{a} \prec_K \mathbf{b}\) & Strict generalized inequality meaning that \(\mathbf{b}-\mathbf{a}\) belongs to the interior of the conic subset \(K\) in the space \(\mathbb{R}^{n}\)\cite{boydConvexOptimization2004}\\ \hline
    \(\mathbf{a} \preceq \mathbf{b}\) & Generalized inequality meaning that \(\mathbf{b}-\mathbf{a}\) belongs to the nonnegative orthant conic subset, \(\mathbb{R}_{+}^{n}\), in the space \(\mathbb{R}^{n}\).\cite{boydConvexOptimization2004}\\ \hline
    \(\mathbf{a} \prec \mathbf{b}\) & Strict generalized inequality meaning that \(\mathbf{b}-\mathbf{a}\) belongs to the positive orthant conic subset, \(\mathbb{R}_{++}^{n}\), in the space \(\mathbb{R}^{n}\)\cite{boydConvexOptimization2004}\\ \hline
    \(\mathbf{A} \preceq_K \mathbf{B}\) & Generalized inequality meaning that \(\mathbf{B}-\mathbf{A}\) belongs to the conic subset \(K\) in the space \(\mathbb{S}^{n}\)\cite{boydConvexOptimization2004}\\ \hline
    \(\mathbf{A} \prec_K \mathbf{B}\) & Strict generalized inequality meaning that \(\mathbf{B}-\mathbf{A}\) belongs to the interior of the conic subset \(K\) in the space \(\mathbb{S}^{n}\)\cite{boydConvexOptimization2004}\\ \hline
    \(\mathbf{A} \preceq \mathbf{B}\) & Generalized inequality meaning that \(\mathbf{B}-\mathbf{A}\) belongs to the positive semidefinite conic subset, \(\mathbb{S}_{+}^{n}\), in the space \(\mathbb{S}^{n}\)\cite{boydConvexOptimization2004}\\ \hline
    \(\mathbf{A} \prec \mathbf{B}\) & Strict generalized inequality meaning that \(\mathbf{B}-\mathbf{A}\) belongs to the positive orthant conic subset, \(\mathbb{S}_{++}^{n}\), in the space \(\mathbb{S}^{n}\)\cite{boydConvexOptimization2004}
\end{xltabular}

\section{Communication systems}
\begin{xltabular}{\textwidth}{XX}
    \(B\) & One-sided bandwidth of the transmitted signal, in Hz\\ \hline
    \(W\) & One-sided bandwidth of the transmitted signal, in rad/s\\ \hline
    \(x_i\) & Real or in-phase part of \(x\)\\ \hline
    \(x_q\) & Imaginary or quadrature part of \(x\)\\ \hline
    \(f_c, f_{RF}\) & Carrier frequency (in Hertz)\\ \hline
    \(f_L\) & Carrier frequency in L-band (in Hertz)\\ \hline
    \(f_{IF}\) & Intermediate frequency (in Hertz)\\ \hline
    \(f_{s}\) & Sampling frequency or sampling rate (in Hertz)\\ \hline
    \(T_{s}\) & Sampling time interval/duration/period\\ \hline
    \(R\) & Bit rate\\ \hline
    \(T\) & Bit interval/duration/period\\ \hline
    \(T_c\) & Chip interval/duration/period\\ \hline
    \(T_{sy}, T_{sym}\) & Symbol/signaling\cite{proakisDigitalCommunications2007} interval/duration/period\\ \hline
    \(s_{RF}\) & Transmitted signal in RF\\ \hline
    \(s_{FI}\) & Transmitted signal in FI\\ \hline
    \(s, s_l\) & Lowpass (or baseband) equivalent signal or envelope complex of transmitted signal\\ \hline
    \(r_{RF}\) & Received signal in RF\\ \hline
    \(r_{FI}\) & Received signal in FI\\ \hline
    \(r, r_l\) & Lowpass (or baseband) equivalent signal or envelope complex of received signal\\ \hline
    \(\phi\) & Signal phase\\ \hline
    \(\phi_0\) & Initial phase\\ \hline
    \(\eta_{RF}, w_{RF}\) & Noise in RF\\ \hline
    \(\eta_{FI}, w_{FI}\) & Noise in FI\\ \hline
    \(\eta, w\) & Noise in baseband\\ \hline
    \(\tau\) & Timing delay \\ \hline
    \(\Delta\tau\) & Timing error (delay - estimated) \\ \hline
    \(\varphi\) & Phase offset \\ \hline
    \(\Delta\varphi\) & Phase error (offset - estimated) \\ \hline
    \(f_d\) & Linear Doppler frequency\\ \hline
    \(\Delta f_d\) & Frequency error (Doppler frequency - estimated)\\ \hline
    \(\nu\) & Angular Doppler frequency\\ \hline
    \(\Delta \nu\) & Frequency error (Doppler frequency - estimated)\\ \hline
    \(\gamma, A\) & Transmitted signal amplitude\\ \hline
    \(\gamma_0, A_0\) & Combined effect of the path loss and antenna gain
\end{xltabular}

\section{Discrete mathematics}
\subsection{Set theory}
\begin{xltabular}{\textwidth}{XX}
    \(A + B\) & Set addition (Minkowski sum), i.e., \(\left\{ \mathbf{v} \in \mathbb{R}^{n} \mid \mathbf{v} = \mathbf{x}+\mathbf{y}, \forall \; \mathbf{x} \in \mathcal{X} \wedge \mathbf{y} \in \mathcal{Y} \right\}\) \cite{kouvaritakisModelPredictiveControl2016}\\ \hline
    \(A - B\) &  Minkowski difference, i.e., \(\left\{ \mathbf{v} \in \mathbb{R}^{n} \mid \mathbf{v} = \mathbf{x}-\mathbf{y}, \forall \; \mathbf{x} \in \mathcal{X} \wedge \mathbf{y} \in \mathcal{Y} \right\}\)\\ \hline
    \(A \ominus B\) & Pontryagin difference, i.e., \(\left\{ \mathbf{v} \in \mathbb{R}^{n} \mid \mathbf{v} + \mathbf{y} \in \mathcal{X} , \forall \; \mathbf{y} \in \mathcal{Y} \right\}\) \cite{kouvaritakisModelPredictiveControl2016}\\ \hline
    \(A \setminus B, A-B\) & Set difference or set subtraction, i.e., \(A \setminus B = \left\{ x \vert x \in A \wedge x \not\in B \right\}\) the set containing the elements of \(A\) that are not in \(B\) \cite{rosenDiscreteMathematicsIts2011}\\ \hline
    \(A \cup B\) & Set of union\\ \hline
    \(A \cap B\) & Set of intersection\\ \hline
    \(A \times B\) & Cartesian product\\ \hline
    \(A^n\) & \(\underbrace{A \times A \times \dots \times A}_{n \text{ times}}\)\\ \hline
    \(A^{\perp}\) & Orthogonal complement of \(A\), e.g., \(\nullspace{\mathbf{A}} = \range{\mathbf{A}^{\top}}^{\perp}\) \cite{boydConvexOptimization2004}\\ \hline
    \(A \oplus B\) & Direct sum, i.e., each \(\mathbf{v} \in \left\{ \sum \mathbf{a}_i \mid \mathbf{a}_i \in S_i, i=1,\dots,k \right\}\) has a unique representation of \(\sum \mathbf{a}_i\) with \(\mathbf{a}_i \in S_i\). That is, they expand to a space. Note that \(\left\{ S_i \right\}\) might not be orthogonal each other \cite{golubMatrixComputations2013}\\ \hline
    \(A \overset{\perp}{\oplus} B\) & Direct sum of two space that are orthogonal and span a \(n\)-dimensional space, e.g., \(\range{\mathbf{A}^{\top}} \overset{\perp}{\oplus} \range{\mathbf{A}^{\top}}^{\perp} = \mathbb{R}^{n}\) (this decomposition of \(\mathbb{R}^{n}\) is called the orthogonal decomposition induced by \(\mathbf{A}\)) \cite{boydConvexOptimization2004}\\ \hline
    \(\bar{A}, A^{c}\) & Complement set (given $U$)\\ \hline
    \(\#A, \abs{A}\) & Cardinality\\ \hline
    \(a \in A\)& \(a\) is element of \(A\) \\ \hline
    \(a \notin A\)& \(a\) is not element of \(A\) \\ \hline
    \(\left\{ 1,2, \dots, n \right\}\) & Discrete set containing the integer elements \(1,2, \dots, n\)\\ \hline
    \(U\) & Universe\\ \hline
    \(2^A\) & Power set of \(A\)\\ \hline
    \(\mathbb{R}\) & Set of real numbers\\ \hline
    \(\mathbb{C}\)& Set of complex numbers\\ \hline
    \(\mathbb{Z}\) & Set of integer number\\ \hline
    \(\mathbb{B} = \left\{ 0, 1 \right\}\) & Boolean set\\ \hline % Circuit Complexity and Neural Networks - Ian Parberry; Further Improvements in the Boolean Domain
    \(\emptyset\) & Empty set\\ \hline
    \(\mathbb{N}\) & Set of natural numbers\\ \hline
    \(\mathbb{K} \in \left\{ \mathbb{R}, \mathbb{C} \right\}\) & Real or complex space (field)\\ \hline
    \(\mathbb{K}^{I_1\times I_2 \times \dots \times I_N}\) & \(I_1\times I_2 \times \dots \times I_N\)-dimensional real (or complex) space\\ \hline
    \(\mathbb{K}_{+}\) & Nonnegative real (or complex) space \cite{boydConvexOptimization2004}\\ \hline
    \(\mathbb{K}_{++}\) & Positive real (or complex) space, i.e., \(\mathbb{K}_{++} = \mathbb{K}_{+}\setminus\left\{ \mathbf{0} \right\}\) \cite{boydConvexOptimization2004}\\ \hline
    \(\mathbb{S}^{n}, \mathcal{S}^{n}\) & Conic set of the symmetric matrices in \(\mathbb{R}^{n\times n}\) \cite{boydConvexOptimization2004}\\ \hline
    \(\mathbb{S}_{+}^{n}, \mathcal{S}_{+}^{n}\) & Conic set of the symmetric positive semidefinite matrices in \(\mathbb{R}^{n\times n}\) \cite{boydConvexOptimization2004}\\ \hline
    \(\mathbb{S}_{++}^{n}, \mathcal{S}_{++}^{n}\) & Conic set of the symmetric positive definite matrices in \(\mathbb{R}^{n\times n}\), i.e., \(\mathbb{S}_{++}^{n} = \mathbb{S}_{+}^{n}\setminus \left\{ \mathbf{0} \right\}\) \cite{boydConvexOptimization2004}\\ \hline
    \(\mathbb{H}^{n}\) & Set of all hermitian matrices in \(\mathbb{C}^{n\times n}\)\\ \hline
    \([a, b]\) & Closed interval of a real set from \(a\) to \(b\)\\ \hline
    \((a, b)\) & Opened interval of a real set from \(a\) to \(b\)\\ \hline
    \([a, b), (a, b]\) & Half-opened intervals of a real set from \(a\) to \(b\)\\
\end{xltabular}

\subsection{Quantifiers, inferences}
\begin{xltabular}{\textwidth}{XX}
    \(\forall\) & For all (universal quantifier) \cite{grahamConcreteMathematicsFoundation1989}\\ \hline
    \(\exists\) & There exists (existential quantifier) \cite{grahamConcreteMathematicsFoundation1989}\\ \hline
    \(\nexists\) & There does not exist \cite{grahamConcreteMathematicsFoundation1989}\\ \hline
    \(\exists!\) & There exist an unique \cite{grahamConcreteMathematicsFoundation1989}\\ \hline
    \(\in\) & Belongs to \cite{grahamConcreteMathematicsFoundation1989}\\ \hline
    \(\not\in\) & Does not belong to \cite{grahamConcreteMathematicsFoundation1989}\\ \hline
    \(\because\) & Because \cite{grahamConcreteMathematicsFoundation1989}\\ \hline
    \(\mid, :\) & Such that, sometimes that parantheses is used \cite{grahamConcreteMathematicsFoundation1989}\\ \hline
    \(, , \left( \cdot \right)\) & Used to separate the quantifier with restricted domain from the its scope, e.g., \(\forall \; x < 0 \left( x^{2} > 0 \right)\) or \(\forall \; x < 0, x^{2} > 0\) \cite{grahamConcreteMathematicsFoundation1989}\\ \hline
    \(\therefore\) & Therefore \cite{grahamConcreteMathematicsFoundation1989}\\
\end{xltabular}

\subsection{Propositional Logic}
\begin{xltabular}{\textwidth}{XX}
    \(\lnot a\) & Logical negation of \(a\) \cite{rosenDiscreteMathematicsIts2011}\\ \hline
    \(a \wedge b\) & Conjunction (logical AND) operator between \(a\) and \(b\)\cite{rosenDiscreteMathematicsIts2011}\\ \hline
    \(a \vee b\) & Disjunction (logical OR) operator between \(a\) and \(b\)\cite{rosenDiscreteMathematicsIts2011}\\ \hline
    \(a \oplus b\) & Exclusive OR (logical XOR) operator between \(a\) and \(b\)\cite{rosenDiscreteMathematicsIts2011}\\ \hline
    \(a \rightarrow b\) & Implication (or conditional) statement\cite{rosenDiscreteMathematicsIts2011}\\ \hline
    \(a \leftrightarrow b\) & Bi-implication (or biconditional) statement, i.e., \(\left( a \rightarrow b \right) \wedge (b \rightarrow a )\) \cite{rosenDiscreteMathematicsIts2011}\\ \hline
    \(a \equiv b, a \iff b, a \Leftrightarrow b \) & Logical equivalence, i.e., \(a \leftrightarrow b\) is a tautology\cite{rosenDiscreteMathematicsIts2011}\\
\end{xltabular}

\subsection{Operations}
\begin{xltabular}{\textwidth}{XX}
    \(\abs{a}\) & Absolute value of \(a\)\\ \hline
    \(\log\) & Base-10 logarithm or decimal logarithm\\ \hline
    \(\ln\) & Natual logarithm\\ \hline
    \(\textnormal{Re}\left\{ x \right\}\) & Real part of \(x\)\\ \hline
    \(\textnormal{Im}\left\{ x \right\}\) & Imaginary part of \(x\)\\ \hline
    \(\angle\cdot\) & Phase (complex argument)\\ \hline
    \(x\;\mathrm{mod}\;y\) & Remainder, i.e., \(x-y\floor{x/y}\), for \(y \neq 0\)\\ \hline
    \(x\;\mathrm{div}\;y\) & Quotient \cite{rosenDiscreteMathematicsIts2011}\\ \hline
    \(x \equiv y\;(\mathrm{mod}\;m)\) & Congruent, i.e.,  \(m \backslash (x-y)\) \cite{rosenDiscreteMathematicsIts2011}\\ \hline
    \(\mathrm{frac}\left(x\right)\) & Fractional part, i.e., \(x\;\mathrm{mod}\;1\) \cite{grahamConcreteMathematicsFoundation1989} \\ \hline
    \(a \backslash b\), \(a \mid b\) & \(b\) is a positive integer multiple of \(a\), i.e., \( \exists\; n \in \mathbb{Z}_{++} \mid b = n a \) \cite{grahamConcreteMathematicsFoundation1989,rosenDiscreteMathematicsIts2011} \\ \hline
    \(a \centernot\backslash b\), \(a \centernot\mid b\) & \(b\) is not a positive integer multiple of \(a\), i.e., \( \nexists\; n \in \mathbb{Z}_{++} \mid b = n a \) \cite{grahamConcreteMathematicsFoundation1989,rosenDiscreteMathematicsIts2011} \\ \hline
    \(\ceil{\cdot}\) & Ceiling operation \cite{grahamConcreteMathematicsFoundation1989}\\ \hline
    \(\floor{\cdot}\) & Floor operation \cite{grahamConcreteMathematicsFoundation1989}
\end{xltabular}

\section{Physics}
\begin{xltabular}{\textwidth}{XX}
    \(\mathbf{\Phi}\) & Electric flux (scalar) (in \(\si{\volt\meter}\))\\ \hline
    \(\mathbf{D}\) & Electric flux density, electric displacement, or electric induction vector (in \(\si{\coulomb\per\square\meter}\))\\ \hline
    \(\mathbf{J}\) & Electric current density vector (in \(\si{\ampere\per\square\meter}\))\\ \hline
    \(\mathbf{H}\) & Magnetic field vector (in \(\si{\ampere\per\meter}\))\\ \hline
    \(\mathbf{B}\) & Magnetic flux density vector (in \(\si{\weber\per\meter\squared} = \si{\tesla}\))\\ \hline
    \(q\) & Electric charge strength/magnitude (in \(\si{\coulomb}\)) \\ \hline
    \(\rho\) & Electric charge density (for volumes) (in \(\si{\coulomb\per\meter^3}\)) \\ \hline
    \(\rho_s\) & Electric charge density (for surface) (in \(\si{\coulomb\per\meter^2}\)) \\ \hline
    \(\rho_l\) & Electric charge density (for volumes) (in \(\si{\coulomb\per\meter}\)) \\ \hline
    \(\mathbf{f}\) & Electrostatic force (Coulomb force), (in \(\si{\kilo\gram\meter\per\second\squared}\)) \\ \hline
    \(\varepsilon\) & Electric permittivity(in \(\si{\farad\per\meter}\)) \cite{ramoFieldsWavesCommunication1994} \\ \hline
    \(\varepsilon_r\) & Relative electric permittivity or dielectric constant (in \(\si{\farad\per\meter}\)) \cite{ramoFieldsWavesCommunication1994} \\ \hline
    \(\varepsilon_0\) & Electric permittivity in vacuum, \(\SI{8.854e-12}{\farad\per\meter}\) \cite{ramoFieldsWavesCommunication1994} \\ \hline
    \(\mathbf{E}\) & Electric field vector (in \(\si{\volt\per\meter}\))\\ \hline
    \(\mu\) & Magnetic permeability \\ \hline
    \(\mu_0\) & Magnetic permeability in vacuum \\
\end{xltabular}

\section{Calculus}
\begin{xltabular}{\textwidth}{XX}
    \(\nabla\) & Vector differential operator (Nabla symbol), i.e., \(\nabla f\) is the gradient of the scalar-valued function \(f\), i.e., \(f: \mathbb{R}^n \rightarrow \mathbb{R}\) \\ \hline
    \(t, (u,v)\) & Parametric variables commonly used, \(t\) for one variable, \((u,v)\) for two variables\cite{stewartCalculus2011}\\ \hline
    \(\mathbf{r}(t)\) & Vector position \((x(t), y(t), z(t))\) parametrized by \(t\)\cite{stewartCalculus2011}\\ \hline
    \(\mathbf{r}'(t)\) & First derivative of \(\mathbf{r}(t)\), i.e., the tangent vector of the curve \((x(t), y(t), z(t))\) \cite{stewartCalculus2011}\\ \hline
    \(\mathbf{T}(t), \mathbf{u}(t)\) & Tangent unit vector of \(\mathbf{r}(t)\), i.e., \newline  \(\mathbf{u}(t) = \mathbf{r}'(t)/\abs{\mathbf{r}'(t)}\)\cite{stewartCalculus2011,kreyszigAdvancedEngineeringMathematics2008}\\ \hline
    \(\mathbf{n}(t), \left( \frac{y'(t)}{\abs{\mathbf{r}'(t)}}, -\frac{x'(t)}{\abs{\mathbf{r}'(t)}} \right)\) & Normal vector of \(\mathbf{r}(t)\), i.e., \newline \(\mathbf{n}(t)\perp \mathbf{T}(t) \)\cite{stewartCalculus2011}\\ \hline
    \(C\)& Contour that traveled by \(\mathbf{r}(t)\), for \(a \leq t \leq b\) \cite{stewartCalculus2011}\\ \hline
    \(L, L(C)\) & Total length of the contour \(C\) (which can be defined the vector \(\mathbf{r}\), parametrized by \(t\)), i.e., \(L_C = \int_a^b \abs{\mathbf{r}'(t)} \dd{t}\)\cite{stewartCalculus2011}\\ \hline
    \(s(t)\) & Length of the arc, which can be defined by the vector \(\mathbf{r}\) and \(t\), that is, \(s(t) = \int_a^t \abs{\mathbf{r}'(u)} \dd{u}\) (\(s(b) = L\))\cite{stewartCalculus2011}\\ \hline
    \(\dd{s}\) & Differential operator of the length of the contour \(C\), i.e., \(\dd{s} = \abs{\mathbf{r}'(t)} \dd{t}\)\\ \hline
    \(\int_C f(\mathbf{r}) \dd{s}, \int_a^b f(\mathbf{r}(t)) \abs{\mathbf{r}'(t)} \dd{t}\) & Line integral of the function \(f: \mathbb{R}^{n} \rightarrow \mathbb{R}\) along the contour \(C\) \cite{apostolCalculus2ndEdn1967,stewartCalculus2011} \\ \hline
    \(\int_C \mathbf{F}\cdot\dd{\mathbf{r}}, \int_a^b \mathbf{F}(\mathbf{r}(t)) \cdot \mathbf{r}'(t) \dd{t}, \int_C \mathbf{F}\cdot\mathbf{T} \dd{s}\) & Line integral of vector field \(\mathbf{F}\) along the contour \(C\)  \cite{apostolCalculus2ndEdn1967,stewartCalculus2011} \\ \hline
    \(\int_\mathbf{a}^\mathbf{b} \mathbf{F}, \int_\mathbf{a}^\mathbf{b} \mathbf{F}\cdot\dd{\mathbf{r}}\) & Alternative notation to the line integral, where the parametric variable \(t\) goes from \(a\) to \(b\), making \(r\) goes from \(\mathbf{r}(a) = \mathbf{a}\) to \(\mathbf{r}(b) = \mathbf{b}\) \cite{apostolCalculus2ndEdn1967} \\ \hline
    \(\oint_C, \varointctrclockwise_C\) & Closed line integral along the contour \(C\) \\ \hline
    \(\mathbf{r}(u,v)\) & Vector position \((x(u,v), y(u,v), z(u,v))\) parametrized by \((u,v)\)\\ \hline
    \(\mathbf{r}_u\) & \((\partial x/ \partial u, \partial y/ \partial u, \partial z/ \partial u)\)\\ \hline
    \(\mathbf{r}_v\) & \((\partial x/ \partial v, \partial y/ \partial v, \partial z/ \partial v)\)\\ \hline
    \(\dd{A}\) & Differential operator of a 2D area (denoted by \(D\) or \(R\)) in the \(\mathbb{R}^2\) domain. This differential operator can be solved in different ways (rectangular, polar, cylindric, etc) \cite{stewartCalculus2011} \\ \hline
    \(D, R\) & Integration domain in which \(\dd{A}\) is integrated, i.e., \(\iint_D f \dd{A}\) \cite{stewartCalculus2011} \\ \hline
    \(S\)& Smooth surface \(S\), i.e., a 2D area in a 3D space (\(\mathbb{R}^3\) domain) \\ \hline
    \(\dd{S}, \abs{\mathbf{r}_u\times\mathbf{r}_v} \dd{A} \)& Differential operator of a 2D area in a 3D domain (an surface). Note that \(\dd{S} = \abs{\mathbf{r}_u\times\mathbf{r}_v} \dd{A}\) should be accompanied with the change of the integration interval(from \(S\) to \(D\)) \\ \hline
    \(A(S), \iint_S \dd{S}, \iint_D \abs{\mathbf{r}_u\times\mathbf{r}_v} \dd{A}\) & Area of the surface \(S\) parametrized by \((u,v)\), in which \(\dd{A}\) is the area defined in the \(D\) domain (which is form by the \(u\)-by-\(v\) graph)\\ \hline
    \(\dd{V}\) & Differential operator of a shape volume (denoted by \(E\)) in \(\mathbb{R}^3\) domain, i.e., \(\iiint_E \dd{V} = V\) \\ \hline
    \(E\) & Integration domain in which \(\dd{V}\) is integrated, i.e., \(\iiint_E f \dd{V}\) \cite{stewartCalculus2011} \\ \hline
    \(V, \iint_D f \dd{A}, \iiint_E f \dd{V}\) & Volume of the function \(f\) over the regions \(D\) (in the case of double integrais) or \(E\) (in the case of triple integrais) \\ \hline
    \(\iint_S f \dd{S}, \iint_D f \abs{\mathbf{r}_u\times\mathbf{r}_v} \dd{A}\)& Surface integral over \(S\) \\ \hline
    \(\mathbf{n}(u,v), \frac{\mathbf{r}_u(u,v)\times\mathbf{r}_v(u,v)}{\abs{\mathbf{r}_u(u,v)\times\mathbf{r}_v(u,v)}}\) & Normal vector of of the smooth surface \(S\) \\ \hline
    \(\iint_S \mathbf{F}\cdot \mathbf{n} \dd{S}, \iint_S \mathbf{F} \cdot \dd{\mathbf{S}}, \newline \iint_D \mathbf{F} \cdot (\mathbf{r}_u\times\mathbf{r}_v) \dd{A}\) & Flux integral of vector field \(\mathbf{F}\) through the smooth surface \(S\) (\(\mathbf{n} \dd{S} \triangleq \dd{\mathbf{S}}\)) \\ \hline
    \(\nabla \times \mathbf{F} , \textnormal{curl } \mathbf{F}\) & Curl (rotacional) of the vector field \(\mathbf{F}\)\\ \hline
    \(\nabla \cdot \mathbf{F} , \textnormal{div } \mathbf{F}\) & Divercence of the vector field \(\mathbf{F}\)\\ \hline
    \(\nabla^2 f, \nabla \cdot (\nabla f), \Delta f, \newline \partial^2f/\partial x^2 + \partial^2f/\partial y^2 + \partial^2f/\partial z^2\) & Scalar Laplacian operator (performed on a scalar-valued function \(f: \mathbb{R}^{n} \rightarrow \mathbb{R}\))\\ \hline
    \(\nabla^2 \mathbf{F}, \nabla \times \nabla \times \mathbf{F} - \nabla(\nabla \cdot \mathbf{F}), \Delta \mathbf{F}, \newline (\partial^2\mathbf{F}/\partial x^2 , \partial^2\mathbf{F}/\partial y^2 , \partial^2\mathbf{F}/\partial z^2)\) & Vector Laplacian operator (performed on a vector field, i.e., a vector-valued function, \(\mathbf{F}: \mathbb{R}^{n} \rightarrow \mathbb{R}^{n}\)). \(\nabla^2\) denotes the scalar (vector) Laplacian if the function is scalar-valued (vector-valued)\\
\end{xltabular}

\section{Generic mathematical symbols}
\begin{xltabular}{\textwidth}{XX}
    \(\blacksquare\) & Q.E.D. \\ \hline
    \(\triangleq\) & Equal by definition\\ \hline
    \(:=, \leftarrow\) & Assignment \cite{rosenDiscreteMathematicsIts2011}\\ \hline
    \(\neq\) & Not equal\\ \hline
    \(\infty\) & Infinity\\ \hline
    \(j\) & \(\sqrt{-1}\)\\
\end{xltabular}

\section{Generic mathematical functions}
\begin{xltabular}{\textwidth}{XX}
    \(\mathcal{O}(\cdot), O(\cdot)\) & Big-O notation\\ \hline
    \(\Gamma(\cdot)\) & Gamma function\\ \hline
    \(Q(\cdot)\) & Quantization function
\end{xltabular}

\section{Abbreviations}
\begin{xltabular}{\textwidth}{XX}
    wrt. & With respect to\\ \hline
    st. & Subject to\\ \hline
    iff. & If and only if\\ \hline
    EVD & Eigenvalue decomposition, or eigendecomposition \cite{nossekAdaptiveArraySignal2015}\\ \hline
    SVD & Singular value decomposition\\ \hline
    CP & CANDECOMP/PARAFAC\\
\end{xltabular}

\printbibliography

\end{document}
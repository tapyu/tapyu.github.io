\documentclass{article}
\pagenumbering{gobble}

% redefine \maketitle
\makeatletter % changes the catcode of @ to 11
\def\@maketitle{%
  \newpage
  \null
  \vskip 2em%
  \begin{center}%
  \let \footnote \thanks
    {\LARGE \@title \par}%
    \vskip 1.5em%
    {\large
      \lineskip .5em%
      \begin{tabular}[t]{c}%
        \@author\\
      \end{tabular}\par}%
    \vskip 1em%
    {\large {\tt Version:}\@date}%
  \end{center}%
  \par
  \vskip 1.5em}
\makeatother % changes the catcode of @ back to 12

\usepackage{authblk}
%%%%%%%%% new commands %%%%%%%%%
\usepackage{xparse} % make command behave differently depending on the number of arguments
\usepackage{xstring} % testing a string's contents, extracting substrings, substitution of substrings

%%%%%%%%% items, tables and figs %%%%%%%%%
\makeatletter
\@ifclassloaded{beamer}{}{\usepackage{enumitem}}
\makeatother
\usepackage{xltabular}
\usepackage{float} % use the [H] option
\usepackage{tikz}
\usepackage{graphicx} % extends the basic LaTeX graphic capabilities: placement, scaling, rotation, file formats
\usepackage{standalone} % place tikz environments or other material in own source files

%%%%%%%%% references %%%%%%%%%
\makeatletter
\@ifclassloaded{beamer}{}{
	\usepackage[colorlinks=true]{hyperref} % make references blue and clickable
	\hypersetup{ % you can use allcolors=blue in hyperref options instead this setup. Color it as black to disable it
		linkcolor=blue,          % For \gls{} entries
		citecolor=blue,          % For \cite{} entries
	}
}
\makeatother
\usepackage{glossaries} % \newglossaryentry{} entries. If you want to make the entries nonclickable, load this package before hyperref
\usepackage[nonumberlist]{glossaries-extra}
%%%%%%%%% bibliography %%%%%%%%%
\usepackage[style=numeric-comp,backend=biber]{biblatex} % tells LaTeX to use the biber tool instead of the traditional bibtex tool for sorting and formatting your bibliography. Biber is a more advanced and powerful tool than bibtex and it provides more features, such as advanced sorting, Unicode support and support for various data sources.
\addbibresource{/home/tapyu/.cache/zotero/refs.bib} % add reference file, in the document, add \printbibliography to print bibliography

%%%%%%%%% math %%%%%%%%%
\usepackage{amsmath}
\usepackage{mathtools} % is an extension package to amsmath to use \DeclarePairedDelimiter
\usepackage{amsfonts}
\usepackage{amssymb}
\usepackage{esint} % for \oiint
\usepackage{newtxmath} % for Greek variants (bold, nonitalic, etc...)
\usepackage{stmaryrd} % provides new symbols, such as \llbracket \rrbracket
\usepackage{bm} % for writing tensor, e.g., $\bm{\mathcal{A}}$
\usepackage{IEEEtrantools} % provides the IEEEeqnarray environment. WARNING: this package may crash your build if the \documentclass is IEEE-like, e.g., IEEEtran. In this case, delete this line!
\usepackage{siunitx} % typesetting units and unitless numbers SI (Système International d’Unités) units (PS: use \SI{} instead \qty{} to avoid conflicts with the same command from the package physics)
% use \SI for number and unit, e.g., \SI{10}{\meter\per\second}
% use \si for unit only, e.g., \si{\meter\per\second}
% use \SIrange[options]{value1}{value2}{unit commands} for a range of values, possible options:
% 1. "range-phrase=-" ->  change the word "to" to "--"
% 2. range-units=single -> chage "1m to 10m" to "1 to 10m"
% 3. mode=text % writes the unit in full
% use the [per-mode=symbol] option in the \si or \SI command to change the writing of \per, or change the line below to set it globally
\sisetup{per-mode=symbol}
% \squared -> ^2

\usepackage{physics} % several math macros, such as
% \tr -> trace
% \rank -> rank
% \expval -> angle brackets to inner product,〈a,b〉, or ensemble average〈s(t)〉
% \abs -> |x|
% \norm -> ||x||
% \eval -> evaluated bar x|_y=a
% \order -> Big-O notation
% \Re -> real
% \Im -> imaginary
% \dd[]{} -> differential
% \dv -> derivative
% \pdv -> partial derivative

%%%%%%%%% operators %%%%%%%%%
\let\oldemptyset\emptyset % change empty set
\let\emptyset\varnothing
% circular convolution a lá Oppenheim (if possible, prefer \circledast)
\newcommand*\circconv[1]{%
	\begin{tikzpicture}
		\node[draw,circle,inner sep=1pt] {#1};
	\end{tikzpicture}}
\newcommand{\intersection}{\bigcap\limits} % intersection operator

%%%%%%%%% delimiters %%%%%%%%%
\makeatletter % changes the catcode of @ to 11
\DeclarePairedDelimiter\ceil{\lceil}{\rceil} % ⌈x⌉
\let\oldceil\ceil
\def\ceil{\@ifstar{\oldceil}{\oldceil*}} % swap the asterisk and the nonasterisk behaviors

\DeclarePairedDelimiter\floor{\lfloor}{\rfloor} % ⌊x⌋
\let\oldfloor\floor
\def\floor{\@ifstar{\oldfloor}{\oldfloor*}}
\makeatother % changes the catcode of @ back to 12

%%%%%%%%% simple functions %%%%%%%%%
\newcommand{\adj}[1]{\ensuremath{\operatorname{adj}\left(#1\right)}} % adjugate matrix
\newcommand{\cof}[1]{\ensuremath{\operatorname{cof}\left(#1\right)}} % cofactor matrix
\newcommand{\eig}[1]{\ensuremath{\operatorname{eig}\left(#1\right)}} % eigenvalues
\newcommand{\nullspace}[1]{\ensuremath{\operatorname{N}\left(#1\right)}} % nullspace or kernel of the matrix
\newcommand{\nullity}[1]{\ensuremath{\operatorname{nullity}\left(#1\right)}} % nullity=dim(N(A))
\newcommand{\spn}[1]{\ensuremath{\operatorname{span}\left\{#1\right\}}} % span of a set of vectors, \span is a reserved word
\newcommand{\range}[1]{\ensuremath{\operatorname{C}\left(#1\right)}} % range or columnspace of a matrix=span(a1,a2, ..., an), where ai is the ith column vector of the matrix A
\newcommand{\unvec}[1]{\ensuremath{\operatorname{unvec}\left(#1\right)}} % unvectorize operator
\newcommand{\diag}[1]{\ensuremath{\operatorname{diag}\left(#1\right)}} % diagonal operator
\newcommand{\dom}[1]{\ensuremath{\operatorname{dom}\left(#1\right)}} % domain of the function
\newcommand{\frob}[1]{\ensuremath{\norm{#1}_\textrm{F}}} % Frobenius norm
\renewcommand{\dim}[1]{\ensuremath{\operatorname{dim}\left(#1\right)}} % dimension of a set

%%%%%%%%% complex functions %%%%%%%%%
\NewDocumentCommand{\argmin}{ e{_} m }{% \argmin_{x ∈ A}{f(x)} or \argmin{f(x)}
	\ensuremath{%
		\IfValueTF{#1}{%
			\underset{#1}{\operatorname{arg\,min}\;#2}
		}{% else
			\operatorname{arg\,min}\;#2
		}
	}
}
\NewDocumentCommand{\argmax}{ e{_} m }{% \argmax_{x ∈ A}{f(x)} or \argmax{f(x)}
	\ensuremath{%
		\IfValueTF{#1}{%
			\underset{#1}{\operatorname{arg\,max}\;#2}
		}{% else
			\operatorname{arg\,max}\;#2
		}
	}
}
% limit in the mean (see Van Tress, chapter 6)
\NewDocumentCommand{\limean}{ E{_}{{}} m }{ % \limean_{x ∈ A}{f(x)} or \limean{f(x)}
	\ensuremath{
		\IfValueTF{#1}{%
			\underset{#1}{\operatorname{l.i.m.}\;#2}
		}{% else
			\operatorname{l.i.m.}\;#2
		}
	}
}
% \E{A}, \E_{u}{A}
% if you want only the left or right part of \E (useful for breaking line) you can use \E_{u}[left-only]{A} or \E_{u}[right-only]{A}, respectively
\NewDocumentCommand{\E}{ E{_}{{}} o m }{%
\ensuremath{%
\IfValueTF{#2}{%
\IfStrEq{#2}{left-only}{%
\operatorname{E}_{#1}\left[#3\right.%
}{% right-only
\left.#3\right]%
}%
}{%
\operatorname{E}_{#1}\left[#3\right]%
}%
}
}
\NewDocumentCommand{\cov}{O{} m}{\ensuremath{\operatorname{cov}_{#1}\left[#2\right]}} % covariance, e.g., \cov[u]{x} or \cov{x}
\let\var\undefined\NewDocumentCommand{\var}{O{} m}{\ensuremath{\operatorname{var}_{#1}\left[#2\right]}} % variance, e.g., \var[u]{x} or \var{x}
\let\vec\undefined\NewDocumentCommand{\vec}{O{} m}{\ensuremath{\operatorname{vec}_{#1}\left[#2\right]}} % vectorize operator, e.g., \vec[u]{\mathbf{A}}, [u] is optional
% normalize all trigonometry functions (new and predefined) to put a parenthesis around the input argument
\RenewDocumentCommand{\sin}{s m}{% \sin{x} -> sin(x) \sin*{x} -> sin x
	\ensuremath{
		\IfBooleanTF{#1}{\operatorname{sin} #2}{\operatorname{sin}\left(#2\right)}%
	}
}
\RenewDocumentCommand{\cos}{s m}{% \cos{x} -> cos(x) \cos*{x} -> cos x
	\ensuremath{
		\IfBooleanTF{#1}{\operatorname{cos} #2}{\operatorname{cos}\left(#2\right)}%
	}
}
\RenewDocumentCommand{\tan}{s m}{% \tan{x} -> tan(x) \tan*{x} -> tan x
	\ensuremath{
		\IfBooleanTF{#1}{\operatorname{tan} #2}{\operatorname{tan}\left(#2\right)}%
	}
}
\NewDocumentCommand{\sinc}{s m}{% \sin{x} -> sinc(x) \sinc*{x} -> sinc x
	\ensuremath{
		\IfBooleanTF{#1}{\operatorname{sinc} #2}{\operatorname{sinc}\left(#2\right)}%
	}
}
\NewDocumentCommand{\sgn}{s m}{% \sgn{x} -> sgn(x) \sng*{x} -> sgn x
	\ensuremath{
		\IfBooleanTF{#1}{\operatorname{sng} #2}{\operatorname{sgn}\left(#2\right)}%
	}
}

%%%%%%%%% minor adjusts in some mathematical symbols %%%%%%%%%
\DeclareMathAlphabet{\mathcal}{OMS}{cmsy}{m}{n} % newtxmath changes the font style of \mathcal. It prevents \mathcal from being changed
\let\oldforall\forall % put some spaces between the \forall command
\renewcommand{\forall}{\;\oldforall\;}

%%%%%%%%% comments, corrections, or revisions %%%%%%%%%
\usepackage{luacolor} % color support based on LuaTEX’s
\usepackage[soul]{lua-ul} % provide underlining, strikethrough, and highlighting using features in LuaLATEX which avoid the restrictions imposed by other methods
\newcommand{\obs}[1]{\textcolor{red}{(#1)}} % comment
\newcommand{\sizecorr}[1]{\makebox[0cm]{\phantom{$\displaystyle #1$}}} % Used to seize the height of equation
\newcommand{\ensureoperation}{\negmedspace {}} % To ensure that a new line symbol is an operation instead of a sign
\usepackage{esint}
\usepackage{bm}
\NewDocumentCommand{\nossekE}{O{} m}{\ensuremath{\operatorname{\bf E}_{#1}\left[#2\right]}} % statistical expectation operator, e.g., \E[u]{x} or \E{x}
\addbibresource{local_ref.bib}

\begin{document}
\title{\textbf{Notation}  \vspace{-.3cm}}
\author{Rubem Vasconcelos Pacelli\\
{\tt rubem.engenharia@gmail.com}}
\affil{Department of Teleinformatics Engineering,\\Federal University of Ceará.\\Fortaleza, Ceará, Brazil. \vspace{-.5cm}}
\maketitle

\tableofcontents
\newpage

\section{Font notation}
\begin{xltabular}{\textwidth}{XX}
	$a,b,c, \dots, A, B, C, \dots$                                                                            & Scalars  \\ \hline
	$\mathbf{a}, \mathbf{b}, \mathbf{c}, \dots$                                                               & Vectors  \\ \hline
	$\mathbf{A}, \mathbf{B}, \mathbf{C}, \dots$                                                               & Matrices \\ \hline
	$\bm{\mathcal{A}}, \bm{\mathcal{B}}, \bm{\mathcal{C}}, \dots$                                             & Tensors  \\ \hline
	$A, B, C, \dots, \mathcal{A}, \mathcal{B}, \mathcal{C}, \dots, \mathbb{A}, \mathbb{B}, \mathbb{C}, \dots$ & Sets     \\
\end{xltabular}

\section{Signals and functions}
\subsection{Time indexing}
\begin{xltabular}{\textwidth}{p{5.7cm}X}
	\(x(t)\)                                                                                                                                                                                      & Continuous-time \(t\)                                                                                                                              \\ \hline
	\(x[n], x[k], x[m], x[i], \dots\) \(x_n, x_k, x_m, x_i, \dots\) \(x(n), x(k), x(m), x(i), \dots\)                                                                                             & Discrete-time \(n, k, m, i, \dots\) (parenthesis should be adopted only if there are no continuous-time signals in the context to avoid ambiguity) \\ \hline
	\(x\left[ \left( \left( n - m \right) \right)_N \right]\)\cite{oppenheimDiscreteTimeSignalProcessing2009}, \(x \left( \left( n - m \right) \right)_N\)\cite{ingleDigitalSignalProcessing2000} & Circular shift in \(m\) samples within a \(N\)-samples window
\end{xltabular}
\subsection{Common signals}
\begin{xltabular}{\textwidth}{XX}
	\(\delta(t)\)                  & Delta function                                  \\ \hline
	\(\delta[n], \delta_{i,j}\)    & Kronecker function (\(n = i-j\))                \\ \hline
	\(h(t), h[n]\)                 & Impulse response (continuous and discrete time) \\ \hline
	\(\tilde{x}[n], \tilde{x}(t)\) & Periodic discrete- or continuous-time signal    \\ \hline
	\(\hat{x}[n], \hat{x}(t)\)     & Estimate of \(x[n]\) or \(x(t)\)                \\ \hline
	\(\dot{x}[m]\)                 & Interpolation of \(x[n]\)                       \\
\end{xltabular}
\subsection{Common functions}
\begin{xltabular}{\textwidth}{XX}
	\(\mathcal{O}(\cdot), O(\cdot)\) & Big-O notation                                                  \\ \hline
	\(\Gamma(\cdot)\)                & Gamma function                                                  \\ \hline
	\(Q(\cdot)\)                     & Quantization function                                           \\ \hline
	\(\textnormal{sgn}(\cdot)\)      & Signum function                                                 \\ \hline
	\(\textnormal{tanh}(\cdot)\)      & Hyperbolic tangent function                                    \\ \hline
	\(I_\alpha(\cdot)\)              & Modified Bessel function of the first kind and order \(\alpha\) \\ \hline
	\(\left( \begin{array}{cc}
			         n \\
			         k
		         \end{array} \right)\)       & Binomial coefficient
\end{xltabular}
\subsection{Operations and symbols}
\begin{xltabular}{\textwidth}{XX}
	\(f: A \rightarrow B\)                                                                                                     & A function \(f\) whose domain is \(A\) and codomain is \(B\)                                                                                                                                                                                                  \\ \hline
	\(\mathbf{f}: A \rightarrow \mathbb{R}^n\)                                                                                 & A vector-valued function \(\mathbf{f}\), i.e., \(n \geq 2\)                                                                                                                                                                                                   \\ \hline
	\(f^{n}, x^{n}(t), x^{n}[k]\)                                                                                              & \(n\)th power of the function \(f\), \(x[n]\) or \(x(t)\)                                                                                                                                                                                                     \\ \hline
	\(f^{\left( n \right)},  x^{(n)}(t)\)                                                                                      & \(n\)th derivative of the function \(f\) or \(x(t)\)                                                                                                                                                                                                          \\ \hline
	\(f', f^{\left( 1 \right)}, x'(t)\)                                                                                        & \(1\)th derivative of the function \(f\) or \(x(t)\)                                                                                                                                                                                                          \\ \hline
	\(f'', f^{\left( 2 \right)}, x''(t)\)                                                                                      & \(2\)th derivative of the function \(f\) or \(x(t)\)                                                                                                                                                                                                          \\ \hline
	\(\argmax[x \in \mathcal{A}]{f(x)} \)                                                                                      & Value of \(x\) that minimizes \(x\)                                                                                                                                                                                                                           \\ \hline
	\( \argmin[x \in \mathcal{A}]{f(x)} \)                                                                                     & Value of \(x\) that minimizes \(x\)                                                                                                                                                                                                                           \\ \hline
	\(f(\mathbf{x}) = \underset{\mathbf{y} \in \mathcal{A}}{\textnormal{inf }} g(\mathbf{x},\mathbf{y})\)                      & Infimum, i.e., \(f(\mathbf{x}) = \min{\left\{ g(\mathbf{x}, \mathbf{y}) \mid \mathbf{y} \in \mathcal{A} \wedge \left( \mathbf{x}, \mathbf{y} \right) \in \dom{g} \right\}}\), which is the greatest lower bound of this set \cite{boydConvexOptimization2004} \\ \hline
	\(f(\mathbf{x}) = \underset{\mathbf{y} \in \mathcal{A}}{\textnormal{sup }} g(\mathbf{x},\mathbf{y})\)                      & Supremum, i.e., \(f(\mathbf{x}) = \max{\left\{ g(\mathbf{x}, \mathbf{y}) \mid \mathbf{y} \in \mathcal{A} \wedge \left( \mathbf{x}, \mathbf{y} \right) \in \dom{g} \right\}}\), which is the least upper bound of this set \cite{boydConvexOptimization2004}   \\ \hline
	\(f \circ g\)                                                                                                              & Composition of the functions \(f\) and \(g\)                                                                                                                                                                                                                  \\ \hline
	\(*\)                                                                                                                      & Convolution (discrete or continuous)                                                                                                                                                                                                                          \\ \hline
	\(\circledast\) \cite{dinizDigitalSignalProcessing2010}, \(\circconv{N}\) \cite{oppenheimDiscreteTimeSignalProcessing2009} & Circular convolution                                                                                                                                                                                                                                          \\
\end{xltabular}
\subsection{Digital signal processing}
\begin{xltabular}{\textwidth}{XX}
	\(W_N\)                                     & Twiddle factor, \(e^{-j\frac{2\pi}{N}}\) \cite{ingleDigitalSignalProcessing2000}            \\ \hline
	\(N\)                                     & Number of samples in the DFT/FFT                                                              \\ \hline
	\(\Omega\) \cite{ingleDigitalSignalProcessing2000}                                      & Continuous angular frequency (in \(\si{\radian\per\second}\))                                                               \\ \hline
	\(\omega\)                                     & Discrete angular frequency. As \(\omega\) is also used to denote continuous angular frequency outside the DSP context, it is always convenient to state that it denotes the discrete frequency when it does                                                                \\ \hline
	\(f_c\)                                     & Continuous linear frequency (in \(\si{\hertz}\))                                                               \\ \hline
	\(f\)                                     & Discrete linear frequency. As \(f\) is also used to denote continuous linear frequency outside the DSP context, it is always convenient to state that it denotes the discrete frequency when it does                                                                \\ \hline
    \(\mathcal{R}_N [n]\) & Rectangular window used to cut off the discrete sequences \cite{ingleDigitalSignalProcessing2000} \\ \hline
    \(T\)\cite{oppenheimDiscreteTimeSignalProcessing2009}, \(T_s\) & Sampling period \\ \hline
    \(f_s\) & Sampling frequency (in \(\si{\hertz}\)), i.e., \(1/T\) \\ \hline
    \(\Omega_s\) & Sampling frequency (in \(\si{\radian\per\second}\)), i.e., \(2\pi f_s\) \\ \hline
    \(\Omega_N\) \cite{oppenheimDiscreteTimeSignalProcessing2009}, \(B\) & One-sided effective bandwidth of the continuous-time signal spectrum \\ \hline
    \(\omega_s\) & Stop frequency \cite{ingleDigitalSignalProcessing2000} \\ \hline
    \(\omega_p\) & Pass frequency \cite{ingleDigitalSignalProcessing2000} \\ \hline
    \(\Delta \omega\) & \(\omega_s - \omega_p\) \cite{ingleDigitalSignalProcessing2000} \\ \hline
    \(\omega_c\) & Cutoff frequency \cite{ingleDigitalSignalProcessing2000} \\ \hline
    \(s(t)\) & Impulse train \\ \hline
    \(\textrm{gdr}\left[ H (e^{j\omega}) \right]\) \cite{oppenheimDiscreteTimeSignalProcessing2009} & Group delay of \(H (e^{j\omega})\) \\ \hline
    \(\angle H (e^{j\omega})\) \cite{oppenheimDiscreteTimeSignalProcessing2009} & Phase response of \(H (e^{j\omega})\) \\ \hline
    \(\abs{H (e^{j\omega})}\) \cite{oppenheimDiscreteTimeSignalProcessing2009} & Magnitude (or gain) of \(H (e^{j\omega})\) \\ \hline
    \(x_c(t)\) \cite{oppenheimDiscreteTimeSignalProcessing2009}, \(x(t)\) & Continuous-time signal \\ \hline
    \(x_s(t)\) & Sampled version of \(x(t)\), i.e., \(x(t)s(t)\) \\ \hline
    \(x_r(t)\) & Reconstruction of \(x(t)\) from interpolation \\ \hline
    \(\tilde{x}[n]\) & Periodic extension of the the aperiodic signal \(x[n]\) \\ \hline
\end{xltabular}
\subsection{Transformations}
\begin{xltabular}{\textwidth}{XX}
	\(\mathcal{F}\left\{ \cdot \right\}\)       & Fourier transform (FT)                                                                                                                             \\ \hline
	\(\mathrm{DTFT}\left\{ \cdot \right\}\), \(\mathrm{DFS}\left\{ \cdot \right\}\), \(\mathrm{FFT}\left\{ \cdot \right\}\)       & Discrete-time Fourier Transform (DTFT), Discrete Fourier Transform (DFT), Discrete Fourier Series (DFS), respectively \\ \hline
	\(\mathcal{L}\left\{ \cdot \right\}\)       & Laplace transform                                                                                                                             \\ \hline
	\(\mathcal{Z}\left\{ \cdot \right\}\)       & \(z\)-transform                                                                                                                               \\ \hline
	\(\hat{x}(t), \hat{x}[n]\)                  & Hilbert transform of \(x(t)\) or \(x[n]\)                                                                                                     \\ \hline
	\(X(s)\)                                    & Laplace transform of \(x(t)\)                                                                                                                 \\ \hline
	\(X(f)\)                                    & Fourier transform (FT) (in linear frequency, \(\unit{\Hz}\)) of \(x(t)\)                                                                      \\ \hline
	\(X(j\omega)\)                              & Fourier transform (FT) (in angular frequency, \(\unit{\radian\per\sec}\)) of \(x(t)\)                                                         \\ \hline
	\(X(e^{j\omega})\)                          & Discrete-time Fourier transform (DTFT) of \(x[n]\)                                                                                            \\ \hline
	\(X[k], X(k), X_k\)                         & Discrete Fourier transform (DFT) or fast Fourier transform (FFT) of \(x[n]\), or even the Fourier series (FS) of the periodic signal \(x(t)\) \\ \hline
	\(\tilde{X}[k], \tilde{X}(k), \tilde{X}_k\) & Discrete Fourier series (DFS) of \(\tilde{x}[n]\)                                                                                             \\ \hline
	\(X(z)\)                                    & \(z\)-transform of \(x[n]\)                                                                                                                   \\
\end{xltabular}

\section{Probability, statistics, and stochastic processes}
\subsection{Operators and symbols}
\begin{xltabular}{\textwidth}{XX}
	\(\E{\cdot}\), \(\nossekE{\cdot}\) \cite{nossekAdaptiveArraySignal2015}, \(E\left[ \cdot \right], \mathbb{E}\left[ \cdot \right]\)           & Statistical expectation operator \cite{dinizAdaptiveFilteringAlgorithms2002}                                                                                            \\ \hline
	\(\E[u]{\cdot}\), \(\nossekE[u]{\cdot}\) \cite{nossekAdaptiveArraySignal2015}, \(E_u\left[ \cdot \right], \mathbb{E}_u\left[ \cdot \right]\) & Statistical expectation operator with respect to \(u\)                                                                                                                  \\ \hline
	\(\expval{\cdot}\)                                                                                                                           & Ensemble average                                                                                                                                                        \\ \hline
	\(\var{\cdot}, \textnormal{VAR}[\cdot]\)                                                                                                     & Variance operator \cite{haykinAdaptiveFilterTheory2002,leon-garciaProbabilityStatisticsRandom2007,proakisDigitalCommunications2007,bishopPatternRecognitionMachine2006} \\ \hline
	\(\var[u]{\cdot} \left[ \cdot \right], \textnormal{VAR}_u[\cdot]\)                                                                           & Variance operator with respect to \(u\)                                                                                                                                 \\ \hline
	\(\cov{\cdot}, \textnormal{COV}[\cdot]\)                                                                                                     & Covariance operator \cite{bishopPatternRecognitionMachine2006}                                                                                                          \\ \hline
	\(\cov[u]{\cdot}, \textnormal{COV}_u[\cdot]\)                                                                                                & Covariance operator with respect to \(u\)                                                                                                                               \\ \hline
	\(\mu_x\)                                                                                                                                    & Mean of the random variable \(x\)                                                                                                                                       \\ \hline
	\(\boldsymbol{\muup}_\mathbf{x}, \mathbf{m}_\mathbf{x}\)                                                                                     & Mean vector of the random variable \(\mathbf{x}\) \cite{brownIntroductionRandomSignals1997}                                                                             \\ \hline
	\(\mu_n\)                                                                                                                                    & \(n\)th-order moment of a random variable                                                                                                                               \\ \hline
	\(\sigma_x^2, \kappa_2\)                                                                                                                     & Variance of the random variable \(x\)                                                                                                                                   \\ \hline
	\(\mathcal{K}_x, \mu_4\)                                                                                                                     & Kurtosis (4th-order moment) of the random variable \(x\)                                                                                                                \\ \hline
	\(\kappa_n\)                                                                                                                                 & \(n\)th-order cumulant of a random variable                                                                                                                             \\ \hline
	\(\rho_{x,y}\)                                                                                                                               & Pearson correlation coefficient between \(x\) and \(y\)                                                                                                                 \\ \hline
	\(a\sim P\)                                                                                                                                  & Random variable \(a\) with distribution \(P\)                                                                                                                           \\ \hline
	\(\mathcal{R}\)                                                                                                                              & Rayleigh's quotient
\end{xltabular}
\subsection{Stochastic processes}
\begin{xltabular}{\textwidth}{XX}
	\(r_x(\tau), R_x(\tau)\)                                                                                                      & Autocorrelation function of the signal \(x(t)\) or \(x[n]\) \cite{nossekAdaptiveArraySignal2015}                                                                                                                                   \\ \hline
	\(S_x(f), S_x(j\omega)\)                                                                                                      & Power spectral density (PSD) of \(x(t)\) in linear (\(f\)) or angular (\(\omega\)) frequency                                                                                                                                       \\ \hline
	\(S_{x,y}(f), S_{x,y}(j\omega)\)                                                                                              & Cross PSD of \(x(t)\) and \(y(t)\) in linear or angular (\(\omega\)) frequency                                                                                                                                                     \\ \hline
	\(\mathbf{R}_\mathbf{x}\)                                                                                                     & (Auto)correlation matrix of \(\mathbf{x}(n)\)                                                                                                                                                                                      \\ \hline
	\(r_{x,d}(\tau), R_{x,d}(\tau)\)                                                                                              & Cross-correlation between \(x[n]\) and \(d[n]\) or \(x(t)\) and \(d(t)\) \cite{nossekAdaptiveArraySignal2015}                                                                                                                      \\ \hline
	\(\mathbf{R}_\mathbf{xy}\)                                                                                                    & Cross-correlation matrix of \(\mathbf{x}(n)\) and \(\mathbf{y}(n)\)                                                                                                                                                                \\ \hline
	\(\mathbf{p}_{\mathbf{x}d}\)                                                                                                  & Cross-correlation vector between \(\mathbf{x}(n)\) and \(d(n)\) \cite{dinizAdaptiveFiltering1997}                                                                                                                                  \\ \hline
	\(c_x(\tau), C_x(\tau)\)                                                                                                      & Autocovariance function of the signal \(x(t)\) or \(x[n]\) \cite{nossekAdaptiveArraySignal2015}                                                                                                                                    \\ \hline
	\(\mathbf{C}_\mathbf{x}, \mathbf{K}_\mathbf{x}, \boldsymbol{\Sigmaup}_\mathbf{x}, \textnormal{cov}\left[ \mathbf{x} \right]\) & (Auto)covariance matrix of \(\mathbf{x}\) \cite{vantreesOptimumArrayProcessing2002,proakisDigitalCommunications2007,leon-garciaProbabilityStatisticsRandom2007,haykinAdaptiveFilterTheory2002,bishopPatternRecognitionMachine2006} \\ \hline
	\(c_{xy}(\tau), C_{xy}(\tau)\)                                                                                                & Cross-covariance function of the signal \(x(t)\) or \(x[n]\) \cite{nossekAdaptiveArraySignal2015}                                                                                                                                  \\ \hline
	\(\mathbf{C}_{\mathbf{xy}}, \mathbf{K}_{\mathbf{xy}}, \boldsymbol{\Sigmaup}_{\mathbf{xy}}\)                                   & Cross-covariance matrix of \(\mathbf{x}\) and \(\mathbf{y}\)
\end{xltabular}

\subsection{Functions}
\begin{xltabular}{\textwidth}{XX}
	\(Q(\cdot)\)                                                              & \(Q\)-function, i.e., \(P\left[ \mathcal{N}(0,1) > x \right] \) \cite{proakisDigitalCommunications2007}                                   \\ \hline
	\(\textnormal{erf}(\cdot)\)                                               & Error function \cite{proakisDigitalCommunications2007}                                                                                    \\ \hline
	\(\textnormal{erfc}(\cdot)\)                                              & Complementary error function i.e., \(\textnormal{erfc}(x) = 2Q(\sqrt{2}x) - \textnormal{erf}(x)\) \cite{proakisDigitalCommunications2007} \\ \hline
	\(P[A]\)                                                                  & Probability of the event or set \(A\) \cite{leon-garciaProbabilityStatisticsRandom2007}                                                   \\ \hline
	\(p(\cdot), f(\cdot)\)                                                    & Probability density function (PDF) or probability mass function (PMF) \cite{leon-garciaProbabilityStatisticsRandom2007}                   \\ \hline
	\(p(x\mid A)\)                                                            & Conditional PDF or PMF \cite{leon-garciaProbabilityStatisticsRandom2007}                                                                  \\ \hline
	\(F(\cdot)\)                                                              & Cumulative distribution function (CDF)                                                                                                    \\ \hline
	\(\Phi_x(\omega), M_x(j\omega), E\left[ e^{j \omega x} \right]\)          & First characteristic function (CF) of \(x\) \cite{proakisDigitalCommunications2007,theodoridisMachineLearningBayesian2020}                \\ \hline
	\(M_x(t), \Phi_x(-j t), E\left[ e^{t x} \right]\)                         & Moment-generating function (MGF) of \(x\) \cite{proakisDigitalCommunications2007,theodoridisMachineLearningBayesian2020}                  \\ \hline
	\(\Psi_x(\omega), \ln \Phi_x(\omega), \ln E\left[ e^{j \omega x}\right]\) & Second characteristic function                                                                                                            \\ \hline
	\(K_x(t), \ln E\left[ e^{t x} \right], \ln M_x(t)\)                       & Cumulant-generating function (CGF) of \(x\) \cite{haykinAdaptiveFilterTheory2002}                                                         \\
\end{xltabular}

\subsection{Distributions}
\begin{xltabular}{\textwidth}{XX}
	\(\mathcal{N}(\mu, \sigma^2)\)                              & Gaussian distribution of a random variable with mean \(\mu\) and variance \(\sigma^{2}\)                                                   \\ \hline
	\(\mathcal{CN}(\mu, \sigma^2)\)                             & Complex Gaussian distribution of a random variable with mean \(\mu\) and variance \(\sigma^{2}\)                                           \\ \hline
	\(\mathcal{N}(\boldsymbol{\muup}, \boldsymbol{\Sigmaup})\)  & Gaussian distribution of a vector random variable with mean \(\boldsymbol{\muup}\) and covariance matrix \(\boldsymbol{\Sigmaup}\)         \\ \hline
	\(\mathcal{CN}(\boldsymbol{\muup}, \boldsymbol{\Sigmaup})\) & Complex Gaussian distribution of a vector random variable with mean \(\boldsymbol{\muup}\) and covariance matrix \(\boldsymbol{\Sigmaup}\) \\ \hline
	\(\mathcal{U}(a,b)\)                                        & Uniform distribution from \(a\) to \(b\)                                                                                                   \\ \hline
	\(\chi^2 (n), \chi^2_n\)                                    & Chi-square distribution with \(n\) degree of freedom (assuming that the Gaussians are \(\mathcal{N}(0,1)\))                                \\ \hline
	\(\textnormal{Exp}(\lambda)\)                               & Exponential distribution with rate parameter \(\lambda\)                                                                                   \\ \hline
	\(\Gamma(\alpha, \beta)\)                                   & Gamma distribution with shape parameter \(\alpha\) and rate parameter \(\beta\)                                                            \\ \hline
	\(\Gamma(\alpha, \theta)\)                                  & Gamma distribution with shape parameter \(\alpha\) and scale parameter \(\theta\ = 1/\beta\)                                               \\ \hline
	\(\textnormal{Nakagami}(m, \Omega)\)                        & Nakagami-m distribution with shape parameter or fading figure \(m\) and spread, scale, or shape parameter \(\Omega\)                       \\ \hline
	\(\textnormal{Rayleigh}(\sigma)\)                           & Rayleigh distribution with scale parameter \(\sigma\)                                                                                      \\ \hline
	\(\textnormal{Rayleigh}(\Omega)\)                           & Rayleigh distribution with the second moment \(\Omega = E\left[ x^2 \right] = 2\sigma^2\)                                                  \\ \hline
	\(\textnormal{Rice}(s, \sigma)\)                            & Rice distribution with noncentrality parameter \(s\) and \(\sigma\). \(s^2\) represent the specular component power                        \\ \hline
	\(\textnormal{Rice}(A, K)\)                                 & Rice distribution with Rice factor \(K=s^2/2\sigma^2\) and scale parameter \(A = s^2 + 2\sigma^2\)
\end{xltabular}

\section{Machine learning, optimization theory, and \newline statistical signal processing}
\subsection{Matrix Calculus}
\begin{xltabular}{\textwidth}{XX}
	\(\mathbf{g}, \nabla f, \frac{\partial f}{\partial \mathbf{w}}\) & Gradient descent vector, ``used'' in the steepest (or gradient) descent method        \\ \hline
	\(\mathbf{g}_{\mathbf{x}}, \nabla_{\mathbf{w}}f, \frac{\partial f}{\partial \mathbf{w}}\)                         & Gradient descent vector with respect \(\mathbf{w}\) \cite{bishopPatternRecognitionMachine2006} \\ \hline
	\(\mathbf{J}, \frac{\partial \mathbf{y}^{\top}}{\partial \mathbf{x}}\)                                                       & Jacobian matrix.                                                                       \\ \hline
	\(\mathbf{H}\), \(\frac{\partial^2 f}{\partial \mathbf{w}^2}\), \(\nabla^2 f\) \cite{haykinNeuralNetworksLearning2009}               & Hessian matrix. The notation \(\nabla^2\) is sometimes used in Matrix Calculus to denote the second-order vector derivative. However, it must be avoided since, in Vector Calculus, \(\nabla^2\) also denotes the Laplacian operator which in turn may be scalar or vector Laplacian operator depending on whether \(f\) is scalar- or vector-valued, respectively. Some discussion about can be found in \cite{4693212, 1353761, 4560326} \\ \hline
\end{xltabular}
\subsection{Estimated terms}
\begin{xltabular}{\textwidth}{XX}
	\(\mathbf{g}\) (or \(\hat{\mathbf{g}}\) if the gradient vector is \(\mathbf{g}\))                             & Stochastic gradient descent (SGD), i.e., instantaneous approximation of gradient descent vector            \\ \hline
	\(\hat{x}(t)\) or \(\hat{x}[n]\)                                                                              & Estimate of \(x(t)\) or \(x[n]\)                                                                           \\ \hline
	\(\hat{\boldsymbol{\muup}}_x, \hat{\mathbf{m}}_x\)                                                            & Sample mean of \(x[n]\) or \(x(t)\)                                                                        \\ \hline
	\(\hat{\boldsymbol{\muup}}_\mathbf{x}, \hat{\mathbf{m}}_\mathbf{x}\)                                          & Sample mean vector of \(\mathbf{x}[n]\) or \(\mathbf{x}(t)\)                                               \\ \hline
	\(\hat{r}_x(\tau), \hat{R}_x(\tau)\)                                                                          & Estimated autocorrelation function of the signal \(x(t)\) or \(x[n]\) \cite{nossekAdaptiveArraySignal2015} \\ \hline
	\(\hat{S}_x(f), \hat{S}_x(j\omega)\)                                                                          & Estimated power spectral density (PSD) of \(x(t)\) in linear (\(f\)) or angular (\(\omega\)) frequency     \\ \hline
	\(\hat{\mathbf{R}}_\mathbf{x}\)                                                                               & Sample (auto)correlation matrix                                                                            \\ \hline
	\(\hat{r}_{x,d}(\tau), \hat{R}_{x,d}(\tau)\)                                                                  & Estimated cross-correlation between \(x[n]\) and \(d[n]\) or \(x(t)\) and \(d(t)\)                         \\ \hline
	\(\hat{S}_{x,y}(f), \hat{S}_{x,y}(j\omega)\)                                                                  & Estimated cross PSD of \(x(t)\) and \(y(t)\) in linear or angular (\(\omega\)) frequency                   \\ \hline
	\(\hat{\mathbf{R}}_\mathbf{xy}\)                                                                              & Sample cross-correlation matrix of \(\mathbf{R}_\mathbf{xy}\)                                              \\ \hline
	\(\hat{c}_x(\tau), \hat{C}_x(\tau)\)                                                                          & Estimated autocovariance function of the signal \(x(t)\) or \(x[n]\)                                       \\ \hline
	\(\hat{\mathbf{C}}_\mathbf{x}, \hat{\mathbf{K}}_\mathbf{x}, \hat{\boldsymbol{\Sigmaup}}_\mathbf{x}\)          & Sample (auto)covariance matrix                                                                             \\ \hline
	\(\hat{c}_{xy}(\tau), \hat{C}_{xy}(\tau)\)                                                                    & Estimated cross-covariance function of the signal \(x(t)\) or \(x[n]\)                                     \\ \hline
	\(\hat{\mathbf{C}}_{\mathbf{xy}}, \hat{\mathbf{K}}_{\mathbf{xy}}, \hat{\boldsymbol{\Sigmaup}}_{\mathbf{xy}}\) & Sample cross-covariance matrix                                                                             \\ \hline
	\(\hat{\mathbf{H}}\)                                                                                          & Estimate of the Hessian matrix
\end{xltabular}

\subsection{Signals, (hyper)parameters, system performance, and criteria}
\begin{xltabular}{\textwidth}{XX}
	\(N\)                                                                                                                 & Number of instances (or samples), i.e., \(n \in \left\{ 1, 2, \dots, N \right\}\)                                                                                                                                            \\ \hline
	\(N_{\textnormal{trn}}\)                                                                                              & Number of instances in the training set, i.e., \(n \in \left\{ 1, 2, \dots, N_\textnormal{trn} \right\}\)     \\ \hline
	\(N_{\textnormal{tst}}\)                                                                                              & Number of instances in the test set, i.e., \(n \in \left\{ 1, 2, \dots, N_\textnormal{tst} \right\}\)     \\ \hline
	\(N_{\textnormal{val}}\)                                                                                              & Number of instances in the validation set, i.e., \(n \in \left\{ 1, 2, \dots, N_\textnormal{val} \right\}\)     \\ \hline
	\(N_e\)                                                                                                               & Number of epochs                                                                                                                                                                                                             \\ \hline
	\(N_a\)                                                                                                               & Number os attributes                                                                                                                                                                                                         \\ \hline
	\(K\) \cite{bishopPatternRecognitionMachine2006}                                                                                                                 & Number of classes (which is the number of outputs in multiclass problems). Use \(k\) to iterate over it                                                                                                                                                                                                            \\ \hline
    \(L\) & Number of layers. Use \(l\) to iterate over it \\ \hline
    \(m_l\) \cite{bishopPatternRecognitionMachine2006}, \(M_l\), \(J\) \cite{bishopPatternRecognitionMachine2006} & Number of neurons at the \(l\)th layer. You might prefer \(J\) in the case of the single-layer perceptron (use \(j\) to iterate over it). If you want to iterate through it, a sensible variation of Haykin notation is \(M_l\), where \(m_l\) can be used as an iterator. \(m_0\) is the length of the input vector without the bias. \\ \hline
	\(\mathbf{x}(n), \mathbf{x}_n\)                                                                                       & Input signal (in \(\mathbb{R}^{N_a + 1}\))  \\ \hline
	\(x_0(n)\)                                                                                                      & Dummy input of the bais, which is usually \(\pm 1\). \(+1\) is prefered \cite{bishopPatternRecognitionMachine2006,haykinNeuralNetworksLearning2009}.                                                                                                                                                                                    \\ \hline
    \(\varphi(\cdot)\)\cite{haykinNeuralNetworksLearning2009}, \(h(\cdot)\)\cite{bishopPatternRecognitionMachine2006}                                                                                                    & Activation function                                                                                                                                                                                                          \\ \hline
    \(\varphi'(v_{m_l}^{(l)}(n))\)\cite{haykinNeuralNetworksLearning2009}, \(\frac{\partial y_{m_l}^{(l)}(n)}{\partial v_{m_l}^{(l)}(n)}\) \cite{haykinNeuralNetworksLearning2009}                                                                                                    & Partial derivative of the activation function with respect to \(v_{m_l}^{(l)}(n)\) (\(m_l\) neuron at \(l\)th layer)                                                                                                                                                                                                          \\ \hline
	\(y_{m_l}^{(l)}(n), \varphi \left( v_{m_l}^{(l)}(n) \right)\)                                                                                    & Output signal of the \(m_l\)th neuron at the \(l\)th layer                                                                                                                                                          \\ \hline
	\(\mathbf{y}^{(l)}(n)\)                                                                                       & Output signal of the \(l\)th layer                                                                                                                                                          \\ \hline
	\(\mathbf{y}(n)\), \(\mathbf{y}^{(L)}(n)\)                                                                                       & Output of the neural network                                                                                                                                                  \\ \hline
	\(\mathbf{d}(n), \mathbf{d}_n\)                                                                                       & Desired label (in case of supervised learning). For multiclass classification, one-hot encoding is usually used. For binary (scalar) classification, however antipodal encoding, i.e., \(\left\{ -1, 1 \right\}\) is more recommended \cite{haykinNeuralNetworksLearning2009}. \\ \hline
	\(e_{m_l}(n)\)                                                                                                         & Error signal of the neuron \(m_l\) at the \(l\)th layer                                                                                                                                                                                                                 \\ \hline
	\(\mathbf{e}(n)\), \(\mathbf{d}(n) - \mathbf{y}(n)\)                                                                                                         & Error signal                                                                                                                                                                                                                 \\ \hline
	\(\mathbf{w}_{m_l}^{(l)}(n), \boldsymbol{\thetaup}_{m_l}^{(l)}(n)\)
    \(\begin{bmatrix}
        w_{m_l,0}^{(l)}(n) & w_{m_l,1}^{(l)}(n) & \dots & w_{m_l,m_{l-1}}^{(l)}(n)
    \end{bmatrix}\)                                    & Parameters, coefficients, or weights vector in the \(l\)th layer. In the case of Single Layer Perceptrons or adaptive filters, the superscript is omitted                                                                                                                                                       \\ \hline
	\(w_{m_l, 0}^{(l)}(n), b_{m_l}^{(l)}(n)\)                                                                                                      & Bias (the first term of the weight vector) of the \(l\)th layer                                                                                                                                                                                    \\ \hline
	\(\mathbf{W}(n)\), \(\begin{bmatrix}
        \mathbf{w}(1) & \mathbf{w}(2) & \cdots & \mathbf{w}(N)
    \end{bmatrix}^\top\)                                                                                                        & Matrix of the weights                                                                                                                                                                                                        \\ \hline
    \(\tilde{\mathbf{W}}(n)\)                                                                                                        & Matrix of the weights, but without the bias                                                                                                                                                                                                        \\ \hline
	\(v_{m_l}^{(l)}(n)\), \(\mathbf{w}_{m_l}^{(l)\top}(n) \mathbf{y}_{m_{l-1}}^{(l-1)}(n)\)                                                                        & Induced local field or activation potential. At the first layer \(\mathbf{y}_{m_{0}}^{(0)}(n) = \mathbf{x}(n)\) \cite{bishopPatternRecognitionMachine2006}                                                                                                                    \\ \hline
    \(\mathbf{v}^{(l)}(n), \mathbf{W}^{(l)}(n) \mathbf{y}_{m_{l-1}}^{(l-1)}(n)\) & Vector of the local fields at the \(l\)th layer \\ \hline
	\(\mathbf{w}^{\star}, \mathbf{w}_o, \boldsymbol{\thetaup}^{\star}, \boldsymbol{\thetaup}_o\)                          & Optimum value of the parameters, coefficients, or weights vector (\(\mathbf{w}^\ast\) is also used \cite{bishopPatternRecognitionMachine2006} but it is not recommended as it may be confused with the conjugation operator) \\ \hline
    \(\delta_{m_l}^{(l)}(n)\), \(\frac{\partial\mathscr{E}(n)}{\partial v_{m_l}^{(l)} (n)}\) &  Local gradient of the \(m_l\)th neuron of the \(l\)th layer. \\ \hline
    \(\boldsymbol{\deltaup}^{(l)}(n)\) & Vector of the local gradients of all neurons at the \(l\)th layer \\ \hline
	\(\mathbf{X}, \begin{bmatrix}
		              \mathbf{x}(1) & \mathbf{x}(2) & \cdots & \mathbf{x}(N)
	              \end{bmatrix}\)                                                               & Data matrix                                                                                                                                                                                                                                            \\ \hline
	\(\eta(n)\)                                                                                                           & Learning rate hyperparameter \cite{bishopPatternRecognitionMachine2006}                                                                                                                                                      \\ \hline
	\(\mathscr{R}\)                                                                                                       & Bayes risk or average risk \cite{bishopPatternRecognitionMachine2006}                                                                                                                                                        \\ \hline
	\(c_{ij}, C_{ij}\)                                                                                                    & Misclassification cost in deciding in favor of class \(\mathscr{C}_i\) (represented in the subspace \(\mathscr{H}_i\)) when the \(\mathscr{C}_j\) is the true class (used in Bayes classifiers/detectors) \cite{bishopPatternRecognitionMachine2006,CharlesPES}    \\ \hline
	\(\mathscr{C}_k\)                                                                                                     & \(k\)th class \cite{bishopPatternRecognitionMachine2006}                                                                                                                                                                     \\ \hline
	\(\mathscr{T}\)                                                                                                       & Training set, i.e., the set \(\left\{ \mathbf{x}(n), d(n) \right\}\) that is used in the training phase \cite{bishopPatternRecognitionMachine2006}                                                                           \\ \hline
	\(\mathscr{H}_k\)                                                                                                     & Subspace of the training vector belonging to the class \(\mathscr{C}_k\)                                                                                                                                                     \\ \hline
	\(\mathscr{H}\)                                                                                                       & Complete space of the input vector, i.e., \(\mathscr{H}_1 \cup \mathscr{H}_2 \cup \cdots \mathscr{H}_K \)                                                                                                                    \\ \hline
    \(\mathscr{X}\) \cite{haykinNeuralNetworksLearning2009} & Set of all vectors in the training, batch, validation, or test dataset that was misclassified \\ \hline
	\(\mathscr{E}(\mathbf{w}), \mathscr{E}(\mathbf{w}(n)), \mathscr{E}(n)\)                                               & Cost function or objective function (the way it is written depends on the purpose of the text)                                                                                                                               \\ \hline
	\(J(\mathbf{w}), J(\mathbf{w}(n)), J(n)\)                                                                             & Alternative to the cost function                                                                                                                                                                                             \\ \hline
	\(\Delta\mathscr{E}(\mathbf{w}(n)), \Delta\mathscr{E}(n), \mathscr{E}(\mathbf{w}(n+1)) - \mathscr{E}(\mathbf{w}(n))\) & Cost function or objective function (the way it is written depends on the purpose of the text)                                                                                                                               \\ \hline
	\(\mathscr{E}_{\textnormal{av}}(\cdot)\)                                                                              & Error energy averaged over the training sample or the empirical risk  \cite{bishopPatternRecognitionMachine2006}                                                                                                             \\ \hline
	\(\Lambda(\cdot)\)                                                                                                    & Likelihood function                                                                                                                                                                                                          \\ \hline
	\(\Lambda_l(\cdot)\)                                                                                                  & Log-likelihood function                                                                                                                                                                                                      \\ \hline
	\(\hat{\rho}_{x,y}\)                                                                                                  & Estimated Pearson correlation coefficient between \(x\) and \(y\)                                                                                                                                                            \\ \hline
	\(\rho\)                                                                                                              & Distance of the margin of separation between two classes (Support Vector Machine, SVM)                                                                                                                                       \\ \hline
	\(g(\cdot)\)                                                                                                          & Discriminant function, i.e., \(g(\mathbf{w}^{\star}) = 0\)
\end{xltabular}

\section{Linear Algebra}
\subsection{Common matrices and vectors}
\begin{xltabular}{\textwidth}{XX}
	\(\mathbf{W}, \mathbf{D}\)                  & Diagonal matrix                                                       \\ \hline
	\(\mathbf{P}\)                              & Projection matrix; Permutation matrix                                 \\ \hline
	\(\mathbf{J}\)                              & Jordan matrix                                                         \\ \hline
	\(\mathbf{L}\)                              & Lower matrix                                                          \\ \hline
	\(\mathbf{U}\)                              & Upper matrix                                                          \\ \hline
	\(\mathbf{C}\)                              & Cofactor matrix                                                       \\ \hline
	\(\mathbf{C}_\mathbf{A}, \cof{\mathbf{A}}\) & Cofactor matrix of \(\mathbf{A}\)                                     \\ \hline
	\(\mathbf{S}\)                              & Symmetric matrix                                                      \\ \hline
	\(\mathbf{Q}\)                              & Orthogonal matrix                                                     \\ \hline
	\(\mathbf{I}_N\)                            & \(N\times N\)-dimensional identity matrix                             \\ \hline
	\(\mathbf{0}_{M\times N}\)                  & \(M\times N\)-dimensional null matrix                                 \\ \hline
	\(\mathbf{0}_{N}\)                          & \(N\)-dimensional null vector                                         \\ \hline
	\(\mathbf{1}_{M\times N}\)                  & \(M\times N\)-dimensional ones matrix                                 \\ \hline
	\(\mathbf{1}_{N}\)                          & \(N\)-dimensional ones vector                                         \\ \hline
	\(\mathbf{0}\)                              & Null matrix, vector, or tensor (dimensionality understood by context) \\ \hline
	\(\mathbf{1}\)                              & Ones matrix, vector, or tensor (dimensionality understood by context) \\
\end{xltabular}

\subsection{Indexing}
\begin{xltabular}[l]{\linewidth}{XX}
	\(x_{i_1,i_2, \dots, i_N}, \left[ \bm{\mathcal{X}} \right]_{i_1,i_2, \dots, i_N}\) & Element in the position \((i_1,i_2, \dots, i_N)\) of the tensor \(\bm{\mathcal{X}}\) \\ \hline
	\(\bm{\mathcal{X}}^{(n)}\)                                                         & \(n\)th tensor of a nontemporal sequence                                             \\ \hline
	\(\mathbf{x}_{n}, \mathbf{x}_{:n}\)                                                & \(n\)th column of the matrix \(X\)                                                   \\ \hline
	\(\mathbf{x}_{n:}\)                                                                & \(n\)th row of the matrix \(X\)                                                      \\ \hline
	\(\mathbf{x}_{i_1,\dots,i_{n-1}, :, i_{n+1},\dots, i_N}\)                          & Mode-\(n\) fiber of the tensor \(\bm{\mathcal{X}}\)                                  \\ \hline
	\(\mathbf{x}_{:,i_2,i_3}\)                                                         & Column fiber (mode-\(1\) fiber) of the thrid-order tensor \(\bm{\mathcal{X}}\)       \\ \hline
	\(\mathbf{x}_{i_1,:,i_3}\)                                                         & Row fiber (mode-\(2\) fiber) of the thrid-order tensor \(\bm{\mathcal{X}}\)          \\ \hline
	\(\mathbf{x}_{i_1,i_2,:}\)                                                         & Tube fiber (mode-\(3\) fiber) of the thrid-order tensor \(\bm{\mathcal{X}}\)         \\ \hline
	\(\mathbf{X}_{i_1,:,:}\)                                                           & Horizontal slice of the thrid-order tensor \(\bm{\mathcal{X}}\)                      \\ \hline
	\(\mathbf{X}_{:,i_2,:}\)                                                           & Lateral slices slice of the thrid-order tensor \(\bm{\mathcal{X}}\)                  \\ \hline
	\(\mathbf{X}_{i_3}, \mathbf{X}_{:,:,i_3}\)                                         & Frontal slices slice of the thrid-order tensor \(\bm{\mathcal{X}}\)
\end{xltabular}

\subsection{General operations}
\begin{xltabular}{\textwidth}{XX}
	\(\expval{\mathbf{a}, \mathbf{b}}, \mathbf{a}^\top\mathbf{b}, \mathbf{a}\cdot\mathbf{b}\) & Inner or dot product                       \\ \hline
	\(\mathbf{a}\circ\mathbf{b}, \mathbf{a}\mathbf{b}^\top\)                                  & Outer product                              \\ \hline
	\(\otimes\)                                                                               & Kronecker product                          \\ \hline
	\(\odot\)                                                                                 & Hadamard (or Schur) (elementwise) product  \\ \hline
	\(\cdot^{\odot n}\)                                                                       & \(n\)th-order Hadamard power               \\ \hline
	\(\cdot^{\odot \frac{1}{n}}\)                                                             & \(n\)th-order Hadamard root                \\ \hline
	\(\oslash\)                                                                               & Hadamard (or Schur) (elementwise) division \\ \hline
	\(\diamond\)                                                                              & Khatri-Rao product                         \\ \hline
	\(\otimes\)                                                                               & Kronecker Product                          \\ \hline
	\(\times_n\)                                                                              & \(n\)-mode product                         \\
\end{xltabular}

\subsection{Operations with matrices and tensors}
\begin{xltabular}{\textwidth}{XX}
	\(\mathbf{A}^{-1}\)                                                                                  & Inverse matrix                                                                                                                                                                  \\ \hline
	\(\mathbf{A}^+, \mathbf{A}^{\dagger}\)                                                               & Moore-Penrose left pseudoinverse                                                                                                                                                \\ \hline
	\(\mathbf{A}^\top, \mathbf{A}^T, \mathbf{A}^t, \mathbf{A}^{'}\) \cite{searleMatrixAlgebraUseful2017} & Transpose                                                                                                                                                                       \\ \hline
	\(\mathbf{A}^{-\top}\)                                                                               & Transpose of the inverse, i.e., \(\left( \mathbf{A}^{-1} \right)^{\top} = \left( \mathbf{A}^{\top} \right)^{-1}\) \cite{petersenMatrixCookbook2008,golubMatrixComputations2013} \\ \hline
	\(\mathbf{A}^*\)                                                                                     & Complex conjugate                                                                                                                                                               \\ \hline
	\(\mathbf{A}^\mathsf{H}\)                                                                            & Hermitian                                                                                                                                                                       \\ \hline
	\(\frob{\mathbf{A}}\)                                                                                & Frobenius norm                                                                                                                                                                  \\ \hline
	\(\norm{\mathbf{A}}\)                                                                                & Matrix norm                                                                                                                                                                     \\ \hline
	\(\abs{\mathbf{A}}, \textnormal{det}\left( \mathbf{A} \right)\)                                      & Determinant                                                                                                                                                                     \\ \hline
	\(\diag{\mathbf{A}}\)                                                                                & The elements in the diagonal of \(\mathbf{A}\)                                                                                                                                  \\ \hline
	\(\vec[]{\mathbf{A}}\)                                                                               & Vectorization: stacks the columns of the matrix \(\mathbf{A}\) into a long column vector                                                                                        \\ \hline
	\(\vec[d]{\mathbf{A}}\)                                                                              & Extracts the diagonal elements of a square matrix and returns them
	in a column vector                                                                                                                                                                                                                                                                     \\ \hline
	\(\vec[l]{\mathbf{A}}\)                                                                              & Extracts the elements strictly below the main diagonal of a square matrix in a column-wise manner and returns them into a column vector                                         \\ \hline
	\(\vec[u]{\mathbf{A}}\)                                                                              & Extracts the elements strictly above the main diagonal of a square matrix in a column-wise manner and returns them into a column vector                                         \\ \hline
	\(\vec[b]{\mathbf{A}}\)                                                                              & Block vectorization operator: stacks square block matrices of the input into a long block column matrix                                                                         \\ \hline
	\(\unvec{\mathbf{A}}\)                                                                               & Reshapes a column vector into a matrix                                                                                                                                          \\ \hline
	\(\tr{\mathbf{A}}\)                                                                                  & trace                                                                                                                                                                           \\ \hline
	\(\mathbf{X}_{(n)}\)                                                                                 & \(n\)-mode matricization of the tensor \(\bm{\mathcal{X}}\)                                                                                                                     \\
\end{xltabular}
\subsection{Operations with vectors}
\begin{xltabular}{\textwidth}{XX}
	\(\norm{\mathbf{a}}\)                      & \(l_1\) norm, 1-norm, or Manhattan norm                                                    \\ \hline
	\(\norm{\mathbf{a}}, \norm{\mathbf{a}}_2\) & \(l_2\) norm, 2-norm, or Euclidean norm                                                    \\ \hline
	\(\norm{\mathbf{a}}_p\)                    & \(l_p\) norm, \(p\)-norm, or Minkowski norm                                                \\ \hline
	\(\norm{\mathbf{a}}_\infty\)               & \(l_\infty\) norm, \(\infty\)-norm, or Chebyshev norm                                      \\ \hline
	\(\diag{\mathbf{a}}\)                      & Diagonalization: a square, diagonal matrix with entries given by the vector \(\mathbf{a}\) \\
\end{xltabular}

\subsection{Decompositions}
\begin{xltabular}{\textwidth}{XX}
	\(\boldsymbol{\Lambdaup}\)                                                                  & Eigenvalue matrix \cite{strangIntroductionLinearAlgebra1993}                                                                                                                         \\ \hline
	\(\mathbf{Q}\)                                                                              & Eigenvectors matrix; Orthogonal matrix of the QR decomposition\cite{strangIntroductionLinearAlgebra1993}                                                                             \\ \hline
	\(\mathbf{R}\)                                                                              & Upper triangular matrix of the QR decomposition\cite{strangIntroductionLinearAlgebra1993}                                                                                            \\ \hline
	\(\mathbf{U}\)                                                                              & Left singular vectors\cite{strangIntroductionLinearAlgebra1993}                                                                                                                      \\ \hline
	\(\mathbf{U}_r\)                                                                            & Left singular nondegenerated vectors                                                                                                                                                 \\ \hline
	\(\boldsymbol{\Sigmaup}\)                                                                   & Singular value matrix                                                                                                                                                                \\ \hline
	\(\boldsymbol{\Sigmaup}_r\)                                                                 & Singular value matrix with nonzero singular values in the main diagonal                                                                                                              \\ \hline
	\(\boldsymbol{\Sigmaup}^{+}\)                                                               & Singular value matrix of the pseudoinverse \cite{strangIntroductionLinearAlgebra1993}                                                                                                \\ \hline
	\(\boldsymbol{\Sigmaup}^{+}_r\)                                                             & Singular value matrix of the pseudoinverse with nonzero singular values in the main diagonal                                                                                         \\ \hline
	\(\mathbf{V}\)                                                                              & Right singular vectors \cite{strangIntroductionLinearAlgebra1993}                                                                                                                    \\ \hline
	\(\mathbf{V}_r\)                                                                            & Right singular nondegenerated vectors                                                                                                                                                \\ \hline
	\(\eig{\mathbf{A}}\)                                                                        & Set of the eigenvalues of \(\mathbf{A}\) \cite{chellappaSignalProcessingTheory2014,leon-garciaProbabilityStatisticsRandom2007,petersenMatrixCookbook2008}                            \\ \hline
	\(\llbracket \mathbf{A}, \mathbf{B}, \mathbf{C}, \dots \rrbracket\)                         & CANDECOMP/PARAFAC (CP) decomposition of the tensor \(\bm{\mathcal{X}}\) from the outer product of column vectors of \(\mathbf{A}\), \(\mathbf{B}\), \(\mathbf{C}, \dots\)            \\ \hline
	\(\llbracket \boldsymbol{\lambdaup}; \mathbf{A}, \mathbf{B}, \mathbf{C}, \dots \rrbracket\) & Normalized CANDECOMP/PARAFAC (CP) decomposition of the tensor \(\bm{\mathcal{X}}\) from the outer product of column vectors of \(\mathbf{A}\), \(\mathbf{B}\), \(\mathbf{C}, \dots\) \\
\end{xltabular}
\subsection{Spaces and sets}
\subsubsection{Common spaces and sets}
\begin{xltabular}{\textwidth}{XX}
	\(\mathbb{R}\)                                             & Set of real numbers                                                                                                                                \\ \hline
	\([a, b]\)                                                 & Closed interval of a real set from \(a\) to \(b\)                                                                                                  \\ \hline
	\((a, b)\)                                                 & Opened interval of a real set from \(a\) to \(b\)                                                                                                  \\ \hline
	\([a, b), (a, b]\)                                         & Half-opened intervals of a real set from \(a\) to \(b\)                                                                                            \\ \hline
	\(\mathbb{C}\)                                             & Set of complex numbers                                                                                                                             \\ \hline
	\(\mathbb{Z}\)                                             & Set of integer number                                                                                                                              \\ \hline
	\(\left\{ 1,2, \dots, n \right\}\)                         & Discrete set containing the integer elements \(1,2, \dots, n\)                                                                                     \\ \hline
	\(\mathbb{B} = \left\{ 0, 1 \right\}\)                     & Boolean set                                                                                                                                        \\ \hline % Circuit Complexity and Neural Networks - Ian Parberry; Further Improvements in the Boolean Domain
	\(\emptyset\)                                              & Empty set                                                                                                                                          \\ \hline
	\(\mathbb{N}\)                                             & Set of natural numbers                                                                                                                             \\ \hline
	\(\mathbb{K} \in \left\{ \mathbb{R}, \mathbb{C} \right\}\) & Real or complex space (field)                                                                                                                      \\ \hline
	\(\mathbb{K}^{I_1\times I_2 \times \dots \times I_N}\)     & \(I_1\times I_2 \times \dots \times I_N\)-dimensional real (or complex) space                                                                      \\ \hline
	\(\mathbb{K}_{+}\)                                         & Nonnegative real (or complex) space \cite{boydConvexOptimization2004}                                                                              \\ \hline
	\(\mathbb{K}_{++}\)                                        & Positive real (or complex) space, i.e., \(\mathbb{K}_{++} = \mathbb{K}_{+}\setminus\left\{ \mathbf{0} \right\}\) \cite{boydConvexOptimization2004} \\ \hline
	\(U\)                                                      & Universe                                                                                                                                           \\ \hline
	\(2^A\)                                                    & Power set of \(A\)                                                                                                                                 \\ \hline
\end{xltabular}

\subsubsection{Convex sets (or spaces)}
\begin{xltabular}{\textwidth}{XX}
	\(\mathbb{S}^{n}\) \cite{dattorroConvexOptimizationEuclidean2010}, \(\mathcal{S}^{n}\) \cite{boydConvexOptimization2004}           & Conic set of the symmetric matrices in \(\mathbb{R}^{n\times n}\)                         \\ \hline
	\(\mathbb{S}_{+}^{n}\), \(\mathcal{S}_{+}^{n}\)   & Conic set of the symmetric positive semidefinite matrices in \(\mathbb{R}^{n\times n}\) \cite{boydConvexOptimization2004}                                                                                      \\ \hline
	\(\mathbb{S}_{++}^{n}, \mathcal{S}_{++}^{n}\) & Conic set of the symmetric positive definite matrices in \(\mathbb{R}^{n\times n}\), i.e., \(\mathbb{S}_{++}^{n} = \mathbb{S}_{+}^{n}\setminus \left\{ \mathbf{0} \right\}\) \cite{boydConvexOptimization2004} \\ \hline
	\(\mathbb{H}^{n}\)                            & Set of all hermitian matrices in \(\mathbb{C}^{n\times n}\)                                                                                                                                                    \\ \hline
	\(\textnormal{conv } C\)                      & Convex hull                                                                                                                                                                                                    \\ \hline
	\(\textnormal{aff } C\)                       & Affune hull                                                                                                                                                                                                    \\ \hline
	\(\mathcal{R}\)                               & Ray                                                                                                                                                                                                            \\ \hline
	\(\mathcal{H}\)                               & Hyperplane                                                                                                                                                                                                     \\ \hline
	\(\mathcal{H}_{+}, \mathcal{H}_{-}\)          & Positive/negative halfspace                                                                                                                                                                                    \\ \hline
	\(B(\mathbf{x}_c, r)\)                        & Euclidean ball with radium \(r\) and centered at \(\mathbf{x}_c\)                                                                                                                                              \\ \hline
	\(\mathcal{E}\)                               & Ellipsoid                                                                                                                                                                                                      \\ \hline
	\(C\)                                         & Norm cone                                                                                                                                                                                                      \\ \hline
	\(K\)                                         & Proper cone                                                                                                                                                                                                    \\ \hline
	\(K^*\)                                       & Dual cone                                                                                                                                                                                                      \\ \hline
	\(\mathcal{P}\)                               & Polyhedra                                                                                                                                                                                                      \\ \hline
	\(S\)                                         & Simplex                                                                                                                                                                                                        \\ \hline
	\(C_\alpha\)                                  & \(\alpha\)-sublevel set                                                                                                                                                                                        \\ \hline
	\(\textnormal{epi } f\)                       & Epigraph of the function \(f\)                                                                                                                                                                                 \\ \hline
	\(\textnormal{hypo } f\)                      & Hypograph of the function \(f\)
\end{xltabular}

\subsubsection{Spaces from matrices or vectors}
\begin{xltabular}{\textwidth}{XX}
	\(\spn{\mathbf{a}_1, \mathbf{a}_2, \dots, \mathbf{a}_n}\)                                                                                          & Vector space spanned by the argument vectors \cite{golubMatrixComputations2013}                                                                                                                                                                               \\ \hline
	\(\range{\mathbf{A}}\), \(\mathrm{columnspace}(\mathbf{A})\), \(\mathrm{range}(\mathbf{A})\), \(\spn{\mathbf{A}}\), \(\mathrm{image}(\mathbf{A})\) & Columnspace, range or image, i.e., the space \(\spn{\mathbf{a}_1,\mathbf{a}_2, \dots, \mathbf{a}_n}\), where \(\mathbf{a}_i\) is the ith column vector of the matrix \(\mathbf{A}\) \cite{strangIntroductionLinearAlgebra1993, nossekAdaptiveArraySignal2015} \\ \hline
	\(\range{\mathbf{A}^\mathsf{H}}\)                                                                                                                  & Row space (also called left columnspace) \cite{strangIntroductionLinearAlgebra1993, nossekAdaptiveArraySignal2015}                                                                                                                                            \\ \hline
	\(\nullspace{\mathbf{A}}, \mathrm{nullspace}(\mathbf{A}), \mathrm{null}(\mathbf{A}), \mathrm{kernel}(\mathbf{A})\)                                 & Nullspace (or kernel space) \cite{strangIntroductionLinearAlgebra1993, nossekAdaptiveArraySignal2015,theodoridisMachineLearningBayesian2020}                                                                                                                  \\ \hline
	\(\nullspace{\mathbf{A^\mathsf{H}}}\)                                                                                                              & Left nullspace                                                                                                                                                                                                                                                \\ \hline
	\(\rank{\mathbf{A}}\)                                                                                                                              & Rank, that is, \(\dim{\spn{\mathbf{A}}} = \dim{\range{\mathbf{A}}}\) \cite{nossekAdaptiveArraySignal2015}                                                                                                                                                     \\ \hline
	\(\nullity{\mathbf{A}}\)                                                                                                                           & Nullity of \(\mathbf{A}\), i.e., \(\dim{\nullspace{\mathbf{A}}}\)                                                                                                                                                                                             \\ \hline
\end{xltabular}

\subsection{Set operations}
\begin{xltabular}{\textwidth}{XX}
	\(A + B\)                           & Set addition (Minkowski sum), i.e., \(\left\{ \mathbf{v} \in \mathbb{R}^{n} \mid \mathbf{v} = \mathbf{x}+\mathbf{y}, \forall \; \mathbf{x} \in \mathcal{X} \wedge \mathbf{y} \in \mathcal{Y} \right\}\) \cite{kouvaritakisModelPredictiveControl2016}                                                                                             \\ \hline
	\(A - B\)                           & Minkowski difference, i.e., \(\left\{ \mathbf{v} \in \mathbb{R}^{n} \mid \mathbf{v} = \mathbf{x}-\mathbf{y}, \forall \; \mathbf{x} \in \mathcal{X} \wedge \mathbf{y} \in \mathcal{Y} \right\}\)                                                                                                                                                   \\ \hline
	\(A \ominus B\)                     & Pontryagin difference, i.e., \(\left\{ \mathbf{v} \in \mathbb{R}^{n} \mid \mathbf{v} + \mathbf{y} \in \mathcal{X} , \forall \; \mathbf{y} \in \mathcal{Y} \right\}\) \cite{kouvaritakisModelPredictiveControl2016}                                                                                                                                \\ \hline
	\(A \setminus B, A-B\)              & Set difference or set subtraction, i.e., \(A \setminus B = \left\{ x \vert x \in A \wedge x \not\in B \right\}\) the set containing the elements of \(A\) that are not in \(B\) \cite{rosenDiscreteMathematicsIts2011}                                                                                                                            \\ \hline
	\(A \cup B\)                        & Set of union                                                                                                                                                                                                                                                                                                                                      \\ \hline
	\(A \cap B\)                        & Set of intersection                                                                                                                                                                                                                                                                                                                               \\ \hline
	\(A \times B\)                      & Cartesian product                                                                                                                                                                                                                                                                                                                                 \\ \hline
	\(A^n\)                             & \(\underbrace{A \times A \times \dots \times A}_{n \text{ times}}\)                                                                                                                                                                                                                                                                               \\ \hline
	\(A^{\perp}\)                       & Orthogonal complement of \(A\), e.g., \(\nullspace{\mathbf{A}} = \range{\mathbf{A}^{\top}}^{\perp}\) \cite{boydConvexOptimization2004}                                                                                                                                                                                                            \\ \hline
	\(\mathbf{a} \perp \mathbf{b}\)     & \(\mathbf{a}\) is orthogonal to \(\mathbf{b}\)                                                                                                                                                                                                                                                                                                    \\ \hline
	\(\mathbf{a} \not\perp \mathbf{b}\) & \(\mathbf{a}\) is not orthogonal to \(\mathbf{b}\)                                                                                                                                                                                                                                                                                                \\ \hline
	\(A \oplus B\)                      & Direct sum, i.e., each \(\mathbf{v} \in \left\{ \sum \mathbf{a}_i \mid \mathbf{a}_i \in S_i, i=1,\dots,k \right\}\) has a unique representation of \(\sum \mathbf{a}_i\) with \(\mathbf{a}_i \in S_i\). That is, they expand to a space. Note that \(\left\{ S_i \right\}\) might not be orthogonal each other \cite{golubMatrixComputations2013} \\ \hline
	\(A \overset{\perp}{\oplus} B\)     & Direct sum of two spaces that are orthogonal and span a \(n\)-dimensional space, e.g., \(\range{\mathbf{A}^{\top}} \overset{\perp}{\oplus} \range{\mathbf{A}^{\top}}^{\perp} = \mathbb{R}^{n}\) (this decomposition of \(\mathbb{R}^{n}\) is called the orthogonal decomposition induced by \(\mathbf{A}\)) \cite{boydConvexOptimization2004}     \\ \hline
	\(\bar{A}, A^{c}\)                  & Complement set (given $U$)                                                                                                                                                                                                                                                                                                                        \\ \hline
	\(\#A, \abs{A}\)                    & Cardinality of \(A\)                                                                                                                                                                                                                                                                                                                                       \\ \hline
	\(a \in A\)                         & \(a\) is element of \(A\)                                                                                                                                                                                                                                                                                                                         \\ \hline
	\(a \notin A\)                      & \(a\) is not element of \(A\)                                                                                                                                                                                                                                                                                                                     \\ \hline
\end{xltabular}

\subsection{Inequalities}
\begin{xltabular}{\textwidth}{XX}
	\(\bm{\mathcal{X}} \leq 0\)         & Nonnegative tensor                                                                                                                                                                                           \\ \hline
	\(\mathbf{a} \preceq_K \mathbf{b}\) & Generalized inequality meaning that \(\mathbf{b}-\mathbf{a}\) belongs to the conic subset \(K\) in the space \(\mathbb{R}^{n}\)\cite{boydConvexOptimization2004}                                             \\ \hline
	\(\mathbf{a} \prec_K \mathbf{b}\)   & Strict generalized inequality meaning that \(\mathbf{b}-\mathbf{a}\) belongs to the interior of the conic subset \(K\) in the space \(\mathbb{R}^{n}\)\cite{boydConvexOptimization2004}                      \\ \hline
	\(\mathbf{a} \preceq \mathbf{b}\)   & Generalized inequality meaning that \(\mathbf{b}-\mathbf{a}\) belongs to the nonnegative orthant conic subset, \(\mathbb{R}_{+}^{n}\), in the space \(\mathbb{R}^{n}\).\cite{boydConvexOptimization2004}     \\ \hline
	\(\mathbf{a} \prec \mathbf{b}\)     & Strict generalized inequality meaning that \(\mathbf{b}-\mathbf{a}\) belongs to the positive orthant conic subset, \(\mathbb{R}_{++}^{n}\), in the space \(\mathbb{R}^{n}\)\cite{boydConvexOptimization2004} \\ \hline
	\(\mathbf{A} \preceq_K \mathbf{B}\) & Generalized inequality meaning that \(\mathbf{B}-\mathbf{A}\) belongs to the conic subset \(K\) in the space \(\mathbb{S}^{n}\)\cite{boydConvexOptimization2004}                                             \\ \hline
	\(\mathbf{A} \prec_K \mathbf{B}\)   & Strict generalized inequality meaning that \(\mathbf{B}-\mathbf{A}\) belongs to the interior of the conic subset \(K\) in the space \(\mathbb{S}^{n}\)\cite{boydConvexOptimization2004}                      \\ \hline
	\(\mathbf{A} \preceq \mathbf{B}\)   & Generalized inequality meaning that \(\mathbf{B}-\mathbf{A}\) belongs to the positive semidefinite conic subset, \(\mathbb{S}_{+}^{n}\), in the space \(\mathbb{S}^{n}\)\cite{boydConvexOptimization2004}    \\ \hline
	\(\mathbf{A} \prec \mathbf{B}\)     & Strict generalized inequality meaning that \(\mathbf{B}-\mathbf{A}\) belongs to the positive orthant conic subset, \(\mathbb{S}_{++}^{n}\), in the space \(\mathbb{S}^{n}\)\cite{boydConvexOptimization2004}
\end{xltabular}

\section{Communication systems}
\subsection{Common symbols}
\begin{xltabular}{\textwidth}{XX}
	\(B\)                 & One-sided bandwidth of the transmitted signal, in Hz                              \\ \hline
	\(W\)                 & One-sided bandwidth of the transmitted signal, in rad/s                           \\ \hline
	\(x_i\)               & Real or in-phase part of \(x\)                                                    \\ \hline
	\(x_q\)               & Imaginary or quadrature part of \(x\)                                             \\ \hline
	\(f_c, f_{RF}\)       & Carrier frequency (in Hertz)                                                      \\ \hline
	\(f_L\)               & Carrier frequency in L-band (in Hertz)                                            \\ \hline
	\(f_{IF}\)            & Intermediate frequency (in Hertz)                                                 \\ \hline
	\(f_{s}\)             & Sampling frequency or sampling rate (in Hertz)                                    \\ \hline
	\(T_{s}\)             & Sampling time interval/duration/period                                            \\ \hline
	\(R\)                 & Bit rate                                                                          \\ \hline
	\(T\)                 & Bit interval/duration/period                                                      \\ \hline
	\(T_c\)               & Chip interval/duration/period                                                     \\ \hline
	\(T_{sy}, T_{sym}\)   & Symbol/signaling\cite{proakisDigitalCommunications2007} interval/duration/period  \\ \hline
	\(s_{RF}\)            & Transmitted signal in RF                                                          \\ \hline
	\(s_{FI}\)            & Transmitted signal in FI                                                          \\ \hline
	\(s, s_l\)            & Lowpass (or baseband) equivalent signal or envelope complex of transmitted signal \\ \hline
	\(r_{RF}\)            & Received signal in RF                                                             \\ \hline
	\(r_{FI}\)            & Received signal in FI                                                             \\ \hline
	\(r, r_l\)            & Lowpass (or baseband) equivalent signal or envelope complex of received signal    \\ \hline
	\(\phi\)              & Signal phase                                                                      \\ \hline
	\(\phi_0\)            & Initial phase                                                                     \\ \hline
	\(\eta_{RF}, w_{RF}\) & Noise in RF                                                                       \\ \hline
	\(\eta_{FI}, w_{FI}\) & Noise in FI                                                                       \\ \hline
	\(\eta, w\)           & Noise in baseband                                                                 \\ \hline
	\(\tau\)              & Timing delay                                                                      \\ \hline
	\(\Delta\tau\)        & Timing error (delay - estimated)                                                  \\ \hline
	\(\varphi\)           & Phase offset                                                                      \\ \hline
	\(\Delta\varphi\)     & Phase error (offset - estimated)                                                  \\ \hline
	\(f_d\)               & Linear Doppler frequency                                                          \\ \hline
	\(\Delta f_d\)        & Frequency error (Doppler frequency - estimated)                                   \\ \hline
	\(\nu\)               & Angular Doppler frequency                                                         \\ \hline
	\(\Delta \nu\)        & Frequency error (Doppler frequency - estimated)                                   \\ \hline
	\(\gamma, A\)         & Transmitted signal amplitude                                                      \\ \hline
	\(\gamma_0, A_0\)     & Combined effect of the path loss and antenna gain
\end{xltabular}
\subsection{Fading multipath channels}
\begin{xltabular}{\textwidth}{XX}
	\(t \overset{\mathcal{F}}{\leftrightarrow} \lambda\)  \cite{proakisDigitalCommunications2007}                                                                                                   & Support temporal of the signal. \(\lambda\) is obtained after taking the Fourier transform on \(t\).                                                                \\ \hline
	\(\tau \overset{\mathcal{F}}{\leftrightarrow} f\)  \cite{proakisDigitalCommunications2007}                                                                                                      & Second support temporal of the signal (\(c(t)\) varies with with the input at the time \(\tau\)). \(f\) is obtained after taking the Fourier transform on \(\tau\). \\ \hline
	\(c(t, \tau)\) \cite{proakisDigitalCommunications2007}                                                                                                                                          & Complex envelope of the channel response at the time \(t\) due to an impulse applied at the \(t - \tau\)                                                            \\ \hline
	\(C(f,t)\) \cite{proakisDigitalCommunications2007}                                                                                                                                              & Transfer function of \(c(t, \tau)\) in \(\tau\)                                                                                                                     \\ \hline
	\(\alpha(t, \tau)\) \cite{proakisDigitalCommunications2007}                                                                                                                                     & Attenuation of \(c(t, \tau)\), i.e., \(c(t, \tau) = \alpha(t, \tau) e^{e\pi f_c \tau}\)                                                                             \\ \hline
	\(R_c(\tau_1, \tau_2, \Delta t)\) \cite{proakisDigitalCommunications2007}                                                                                                                       & Autocorrelation function of \(c(t, \tau)\), i.e., \(R_c(\tau_1, \tau_2, \Delta t) = \E{c^*(t, \tau_1), c^*(t + \Delta t, \tau_2)}\)                                 \\ \hline
	\(R_c(\tau, \Delta t)\) \cite{proakisDigitalCommunications2007}                                                                                                                                 & Autocorrelation function of \(c(t, \tau)\) assuming uncorrelated scattering                                                                                         \\ \hline
	\(R_c(\tau), \eval{R_c(\tau, \Delta t)}_{\Delta t = 0}\) \cite{proakisDigitalCommunications2007}                                                                                                & Multipath intensity profile or delay power spectrum                                                                                                                 \\ \hline
	\(R_C(\Delta f, \Delta t), R_C(f_1, f_2; \Delta t)\), \(\E{C(f_1,t), C(f_2, t + \Delta t)}\), \(\mathcal{F}_\tau \left\{ R_c(\tau, \Delta t) \right\}\) \cite{goldsmithWirelessCommunications2005} & Spaced-frequency, spaced-time correlation function (\(\Delta f = f_2 - f_1\))                                                                                       \\ \hline
	\(R_C(\Delta f)\), \(\eval{R_C(\Delta f, \Delta t)}_{\Delta t = 0}\) \cite{proakisDigitalCommunications2007}, \(\mathcal{F}\left\{ R_c(\tau) \right\}\) \cite{goldsmithWirelessCommunications2005}                                             & Spaced-frequency correlation function                                                                                                                               \\ \hline
	\((\Delta f)_c\)                                                                                                                                        & Coherence bandwidth of \(c(t)\), that is, the frequency interval in which \(R_C(\Delta f)\) is nonzero \cite{proakisDigitalCommunications2007}                                                              \\ \hline
	\(T_m\)                                                                                                                                                 & Multipath spread of the channel, that is, the time interval in which \(R_c(\tau)\) is nonzero (\(T_m \approx 1/(\Delta f)_c \)) \cite{proakisDigitalCommunications2007}                                 \\ \hline
	\(R_C(\Delta t), \eval{R_C(\Delta f, \Delta t)}_{\Delta f = 0}\)                                                                                        & Spaced-time correlation function \cite{proakisDigitalCommunications2007}                                                                                                                                    \\ \hline
	\(S_C(\lambda)\) \cite{proakisDigitalCommunications2007}, \(\mathcal{F}\left\{ R_C (\Delta t) \right\}\) \cite{goldsmithWirelessCommunications2005}                                                                                            & Doppler power spectrum                                                                                                                                              \\ \hline
	\((\Delta t)_c\)                                                                                                                                        & Coherence time of \(c(t)\), that is, the time interval in which \(R_C(\Delta t)\) is nonzero \cite{proakisDigitalCommunications2007}                                                                        \\ \hline
	\(B_m\)                                                                                                                                                 & Multipath spread of the channel, that is, the frequency interval in which \(S_c(\lambda)\) is nonzero (\(B_d \approx 1/(\Delta t)_c \)) \cite{proakisDigitalCommunications2007}                             \\ \hline
	\(S_C(\tau, \lambda)\) \cite{proakisDigitalCommunications2007}, \(\mathcal{F}_{\Delta f, \Delta t}\left\{ R_C (\Delta f, \Delta t) \right\}\) \cite{goldsmithWirelessCommunications2005}                                                       & Scattering function                                                                                                                                                 \\ \hline
\end{xltabular}

\section{Discrete mathematics}
\subsection{Quantifiers, inferences}
\begin{xltabular}{\textwidth}{XX}
	\(\forall\)                  & For all (universal quantifier) \cite{grahamConcreteMathematicsFoundation1989}                                                                                                                                \\ \hline
	\(\exists\)                  & There exists (existential quantifier) \cite{grahamConcreteMathematicsFoundation1989}                                                                                                                         \\ \hline
	\(\nexists\)                 & There does not exist \cite{grahamConcreteMathematicsFoundation1989}                                                                                                                                          \\ \hline
	\(\exists!\)                 & There exists an unique \cite{grahamConcreteMathematicsFoundation1989}                                                                                                                                         \\ \hline
	\(\exists_{n}\)              & There exists exactly \(n\) \cite{rosenDiscreteMathematicsIts2011}                                                                                                                                         \\ \hline
	\(\in\)                      & Belongs to \cite{grahamConcreteMathematicsFoundation1989}                                                                                                                                                    \\ \hline
	\(\not\in\)                  & Does not belong to \cite{grahamConcreteMathematicsFoundation1989}                                                                                                                                            \\ \hline
	\(\because\)                 & Because \cite{grahamConcreteMathematicsFoundation1989}                                                                                                                                                       \\ \hline
	\(\mid, :\)                  & Such that, sometimes that parentheses is used \cite{grahamConcreteMathematicsFoundation1989}                                                                                                                 \\ \hline
	\(, , \left( \cdot \right)\) & Used to separate the quantifier with restricted domain from its scope, e.g., \(\forall \; x < 0 \left( x^{2} > 0 \right)\) or \(\forall \; x < 0, x^{2} > 0\) \cite{grahamConcreteMathematicsFoundation1989} \\ \hline
	\(\therefore\)               & Therefore \cite{grahamConcreteMathematicsFoundation1989}                                                                                                                                                     \\
\end{xltabular}

\subsection{Propositional Logic}
\begin{xltabular}{\textwidth}{XX}
	\(\lnot a\)                                    & Logical negation of \(a\) \cite{rosenDiscreteMathematicsIts2011}                                                                                       \\ \hline
	\(a \wedge b\)                                 & Conjunction (logical AND) operator between \(a\) and \(b\)\cite{rosenDiscreteMathematicsIts2011}                                                       \\ \hline
	\(a \vee b\)                                   & Disjunction (logical OR) operator between \(a\) and \(b\)\cite{rosenDiscreteMathematicsIts2011}                                                        \\ \hline
	\(a \oplus b\)                                 & Exclusive OR (logical XOR) operator between \(a\) and \(b\)\cite{rosenDiscreteMathematicsIts2011}                                                      \\ \hline
	\(a \rightarrow b\)                            & Implication (or conditional) statement\cite{rosenDiscreteMathematicsIts2011}                                                                           \\ \hline
	\(a \leftrightarrow b\)                        & Bi-implication (or biconditional) statement, i.e., \(\left( a \rightarrow b \right) \wedge (b \rightarrow a )\) \cite{rosenDiscreteMathematicsIts2011} \\ \hline
	\(a \equiv b, a \iff b, a \Leftrightarrow b \) & Logical equivalence, i.e., \(a \leftrightarrow b\) is a tautology\cite{rosenDiscreteMathematicsIts2011}                                                \\
\end{xltabular}

\subsection{Operations}
\begin{xltabular}{\textwidth}{XX}
	\(\abs{a}\)                                          & Absolute value of \(a\)                                                                                                                                                                     \\ \hline
	\(\log\)                                             & Base-10 logarithm or decimal logarithm                                                                                                                                                      \\ \hline
	\(\ln\)                                              & Natual logarithm                                                                                                                                                                            \\ \hline
	\(\textnormal{Re}\left\{ x \right\}\)                & Real part of \(x\)                                                                                                                                                                          \\ \hline
	\(\textnormal{Im}\left\{ x \right\}\)                & Imaginary part of \(x\)                                                                                                                                                                     \\ \hline
	\(\angle\cdot\)                                      & Phase (complex argument)                                                                                                                                                                    \\ \hline
	\(x\;\mathrm{mod}\;y\)                               & Remainder, i.e., \(x-y\floor{x/y}\), for \(y \neq 0\)                                                                                                                                       \\ \hline
	\(x\;\mathrm{div}\;y\)                               & Quotient \cite{rosenDiscreteMathematicsIts2011}                                                                                                                                             \\ \hline
	\(x \equiv y\;(\mathrm{mod}\;m)\)                    & Congruent, i.e.,  \(m \backslash (x-y)\) \cite{rosenDiscreteMathematicsIts2011}                                                                                                             \\ \hline
	\(\mathrm{frac}\left(x\right)\)                      & Fractional part, i.e., \(x\;\mathrm{mod}\;1\) \cite{grahamConcreteMathematicsFoundation1989}                                                                                                \\ \hline
	\(a \backslash b\), \(a \mid b\)                     & \(b\) is a positive integer multiple of \(a\), i.e., \( \exists\; n \in \mathbb{Z}_{++} \mid b = n a \) \cite{grahamConcreteMathematicsFoundation1989,rosenDiscreteMathematicsIts2011}      \\ \hline
	\(a \centernot\backslash b\), \(a \centernot\mid b\) & \(b\) is not a positive integer multiple of \(a\), i.e., \( \nexists\; n \in \mathbb{Z}_{++} \mid b = n a \) \cite{grahamConcreteMathematicsFoundation1989,rosenDiscreteMathematicsIts2011} \\ \hline
	\(\ceil{\cdot}\)                                     & Ceiling operation \cite{grahamConcreteMathematicsFoundation1989}                                                                                                                            \\ \hline
	\(\floor{\cdot}\)                                    & Floor operation \cite{grahamConcreteMathematicsFoundation1989}
\end{xltabular}

\section{Vector Calculus}
\begin{xltabular}{\textwidth}{XX}
	\(\nabla\)                                                                                                                                                                                                                                 & Vector differential operator (Nabla symbol), i.e., \(\nabla f\) is the gradient of the scalar-valued function \(f\), i.e., \(f: \mathbb{R}^n \rightarrow \mathbb{R}\)                                                                                   \\ \hline
	\(t, (u,v)\)                                                                                                                                                                                                                               & Parametric variables commonly used, \(t\) for one variable, \((u,v)\) for two variables\cite{stewartCalculus2011}                                                                                                                                       \\ \hline
	\(\dd{\mathbf{l}}, \dd{\mathbf{r}}\)                                                                                                                                                                                                       & Vector position, i.e., \((x, y, z)\). Stewart \cite{stewartCalculus2011} utilizes the letter \(\mathbf{r}\) to denote it, but it appears in many electromagnetics books as \(\dd{\mathbf{l}}\)                                                          \\ \hline
	\(\mathbf{l}(t)\)                                                                                                                                                                                                                          & Vector position parametrized by \(t\), i.e., \((x(t), y(t), z(t))\) \cite{stewartCalculus2011,ramoFieldsWavesCommunication1994}                                                                                                                         \\ \hline
	\(\mathbf{l}'(t), \dd{\mathbf{l}}/\dd{t}\)                                                                                                                                                                                                 & First derivative of \(\mathbf{l}(t)\), i.e., the tangent vector of the curve \((x(t), y(t), z(t))\) \cite{stewartCalculus2011}                                                                                                                          \\ \hline
	\(\mathbf{T}(t), \mathbf{u}(t)\)                                                                                                                                                                                                           & Tangent unit vector of \(\mathbf{l}(t)\), i.e., \newline  \(\mathbf{u}(t) = \mathbf{l}'(t)/\abs{\mathbf{l}'(t)}\)\cite{stewartCalculus2011,kreyszigAdvancedEngineeringMathematics2008}                                                                  \\ \hline
	\(\mathbf{n}(t), \left( \frac{y'(t)}{\abs{\mathbf{l}'(t)}}, -\frac{x'(t)}{\abs{\mathbf{l}'(t)}} \right)\)                                                                                                                                  & Normal vector of \(\mathbf{l}(t)\), i.e., \newline \(\mathbf{n}(t)\perp \mathbf{T}(t) \)\cite{stewartCalculus2011}                                                                                                                                      \\ \hline
	\(C\)                                                                                                                                                                                                                                      & Contour that traveled by \(\mathbf{l}(t)\), for \(a \leq t \leq b\) \cite{stewartCalculus2011}                                                                                                                                                          \\ \hline
	\(L, L(C)\)                                                                                                                                                                                                                                & Total length of the contour \(C\) (which can be defined the vector \(\mathbf{l}\), parametrized by \(t\)), i.e., \(L_C = \int_a^b \abs{\mathbf{l}'(t)} \dd{t}\)\cite{stewartCalculus2011}                                                               \\ \hline
	\(s(t)\)                                                                                                                                                                                                                                   & Length of the arc, which can be defined by the vector \(\mathbf{l}\) and \(t\), that is, \(s(t) = \int_a^t \abs{\mathbf{l}'(u)} \dd{u}\) (\(s(b) = L\))\cite{stewartCalculus2011}                                                                       \\ \hline
	\(\dd{s}\)                                                                                                                                                                                                                                 & Differential operator of the length of the contour \(C\), i.e., \(\dd{s} = \abs{\mathbf{l}'(t)} \dd{t}\) \cite{stewartCalculus2011}                                                                                                                     \\ \hline
	\(\int_C f(\mathbf{l}) \dd{s}, \int_a^b f(\mathbf{l}(t)) \abs{\mathbf{l}'(t)} \dd{t}\)                                                                                                                                                     & Line integral of the function \(f: \mathbb{R}^{n} \rightarrow \mathbb{R}\) along the contour \(C\). In the context of integrals in the complex plane, it is also called ``contour integral'' \cite{apostolCalculus2ndEdn1967,stewartCalculus2011}                                                                                                 \\ \hline
	\(\int_C \mathbf{F}\cdot\dd{\mathbf{l}}, \int_a^b \mathbf{F}(\mathbf{l}(t)) \cdot \mathbf{l}'(t) \dd{t}, \int_C \mathbf{F}\cdot\mathbf{T} \dd{s}\)                                                                                         & Line integral of vector field \(\mathbf{F}\) along the contour \(C\)  \cite{apostolCalculus2ndEdn1967,stewartCalculus2011}                                                                                                                              \\ \hline
	\(\int_\mathbf{a}^\mathbf{b} \mathbf{F}, \int_\mathbf{a}^\mathbf{b} \mathbf{F}\cdot\dd{\mathbf{l}}\)                                                                                                                                       & Alternative notation to the line integral, where the parametric variable \(t\) goes from \(a\) to \(b\), making \(r\) goes from \(\mathbf{l}(a) = \mathbf{a}\) to \(\mathbf{l}(b) = \mathbf{b}\) \cite{apostolCalculus2ndEdn1967}                       \\ \hline
	\(\oint_C, \varointctrclockwise_C\)                                                                                                                                                                                                        & Line integral along the closed contour \(C\). The arrow indicates the contour integral orientation, which is counterclockwise, by default. In the context of integrals in the complex plane, it is also called ``closed contour integral''.                                                                                                              \\ \hline
	\(\oiint_S\)                                                                                                                                                                                                                               & Surface integral over the closed surface \(S\)                                                                                                                                                                                                          \\ \hline
	\(\mathbf{l}(u,v)\)                                                                                                                                                                                                                        & Vector position \((x(u,v), y(u,v), z(u,v))\) parametrized by \((u,v)\)                                                                                                                                                                                  \\ \hline
	\(\mathbf{l}_u\)                                                                                                                                                                                                                           & \((\partial x/ \partial u, \partial y/ \partial u, \partial z/ \partial u)\)                                                                                                                                                                            \\ \hline
	\(\mathbf{l}_v\)                                                                                                                                                                                                                           & \((\partial x/ \partial v, \partial y/ \partial v, \partial z/ \partial v)\)                                                                                                                                                                            \\ \hline
	\(\dd{A}\)                                                                                                                                                                                                                                 & Differential operator of a 2D area (denoted by \(D\) or \(R\)) in the \(\mathbb{R}^2\) domain. This differential operator can be solved in different ways (rectangular, polar, cylindric, etc) \cite{stewartCalculus2011}                               \\ \hline
	\(D, R\)                                                                                                                                                                                                                                   & Integration domain in which \(\dd{A}\) is integrated, i.e., \(\iint_D f \dd{A}\) \cite{stewartCalculus2011}                                                                                                                                             \\ \hline
	\(S\)                                                                                                                                                                                                                                      & Smooth surface \(S\), i.e., a 2D area in a 3D space (\(\mathbb{R}^3\) domain)                                                                                                                                                                           \\ \hline
	\(\dd{S}, \abs{\mathbf{l}_u\times\mathbf{l}_v} \dd{A} \)                                                                                                                                                                                   & Differential operator of a 2D area in a 3D domain (an surface). Note that \(\dd{S} = \abs{\mathbf{l}_u\times\mathbf{l}_v} \dd{A}\) should be accompanied with the change of the integration interval(from \(S\) to \(D\))                               \\ \hline
	\(A(S), \iint_S \dd{S}, \iint_D \abs{\mathbf{l}_u\times\mathbf{l}_v} \dd{A}\)                                                                                                                                                              & Area of the surface \(S\) parametrized by \((u,v)\), in which \(\dd{A}\) is the area defined in the \(D\) domain (which is form by the \(u\)-by-\(v\) graph)                                                                                            \\ \hline
	\(\dd{V}\)                                                                                                                                                                                                                                 & Differential operator of a shape volume (denoted by \(E\)) in \(\mathbb{R}^3\) domain, i.e., \(\iiint_E \dd{V} = V\)                                                                                                                                    \\ \hline
	\(E\)                                                                                                                                                                                                                                      & Integration domain in which \(\dd{V}\) is integrated, i.e., \(\iiint_E f \dd{V}\) \cite{stewartCalculus2011}                                                                                                                                            \\ \hline
	\(V, \iint_D f \dd{A}, \iiint_E f \dd{V}\)                                                                                                                                                                                                 & Volume of the function \(f\) over the regions \(D\) (in the case of double integrals) or \(E\) (in the case of triple integrals)                                                                                                                        \\ \hline
	\(\iint_S f \dd{S}, \iint_D f \abs{\mathbf{l}_u\times\mathbf{l}_v} \dd{A}\)                                                                                                                                                                & Surface integral over \(S\)                                                                                                                                                                                                                             \\ \hline
	\(\mathbf{n}(u,v), \frac{\mathbf{l}_u(u,v)\times\mathbf{l}_v(u,v)}{\abs{\mathbf{l}_u(u,v)\times\mathbf{l}_v(u,v)}}\)                                                                                                                       & Normal vector of of the smooth surface \(S\)                                                                                                                                                                                                            \\ \hline
	\(\iint_S \mathbf{F}\cdot \mathbf{n} \dd{S}, \iint_S \mathbf{F} \cdot \dd{\mathbf{S}}, \newline \iint_D \mathbf{F} \cdot (\mathbf{l}_u\times\mathbf{l}_v) \dd{A}\)                                                                         & Flux integral of vector field \(\mathbf{F}\) through the smooth surface \(S\) (\(\mathbf{n} \dd{S} \triangleq \dd{\mathbf{S}}\))                                                                                                                        \\ \hline
	\(\oiint_S \mathbf{F}\cdot \mathbf{n} \dd{S}, \oiint_S \mathbf{F} \cdot \dd{\mathbf{S}}, \newline \iint_D \mathbf{F} \cdot (\mathbf{l}_u\times\mathbf{l}_v) \dd{A}\)                                                                       & Flux integral of vector field \(\mathbf{F}\) through the smooth and closed surface \(S\) (\(\mathbf{n} \dd{S} \triangleq \dd{\mathbf{S}}\))                                                                                                             \\ \hline
	\(\nabla \times \mathbf{F} , \textnormal{curl } \mathbf{F}\)                                                                                                                                                                               & Curl (rotacional) of the vector field \(\mathbf{F}\)                                                                                                                                                                                                    \\ \hline
	\(\nabla \cdot \mathbf{F} , \textnormal{div } \mathbf{F}\)                                                                                                                                                                                 & Divercence of the vector field \(\mathbf{F}\)                                                                                                                                                                                                           \\ \hline
	\(\nabla^2 f, \nabla \cdot (\nabla f), \Delta f, \newline \partial^2f/\partial x^2 + \partial^2f/\partial y^2 + \partial^2f/\partial z^2\)                                                                                                 & Scalar Laplacian operator (performed on a scalar-valued function \(f: \mathbb{R}^{n} \rightarrow \mathbb{R}\))                                                                                                                                          \\ \hline
	\(\nabla^2 \mathbf{F}, \nabla \times \nabla \times \mathbf{F} - \nabla(\nabla \cdot \mathbf{F}), \Delta \mathbf{F}, \newline (\partial^2\mathbf{F}/\partial x^2 , \partial^2\mathbf{F}/\partial y^2 , \partial^2\mathbf{F}/\partial z^2)\) & Vector Laplacian operator (performed on a vector field, i.e., a vector-valued function, \(\mathbf{F}: \mathbb{R}^{n} \rightarrow \mathbb{R}^{n}\)). \(\nabla^2\) denotes the scalar (vector) Laplacian if the function is scalar-valued (vector-valued). The notation \(\Delta\) must be avoided as it is overused in many contexts \\
\end{xltabular}

\section{Electromagnetic waves}
\begin{xltabular}{\textwidth}{XX}
	\(\mathbf{\Phi}\)                                            & Electric flux (scalar) (in \(\si{\volt\meter}\))                                                                                                                                         \\ \hline
	\(\mathbf{J}\)                                               & Electric current density vector (in \(\si{\ampere\per\square\meter}\))                                                                                                                   \\ \hline
	\(\mathbf{H}\)                                               & Magnetic field vector (in \(\si{\ampere\per\meter}\))                                                                                                                                    \\ \hline
	\(\mathbf{B}\)                                               & Magnetic flux density vector (in \(\si{\weber\per\meter\squared} = \si{\tesla}\))                                                                                                        \\ \hline
	\(q_{\textnormal{free}}\)                                    & Free electric charge (in \(\si{\coulomb}\))                                                                                                                                              \\ \hline
	\(q_{\textnormal{bound}}\)                                   & Bound electric charge (in \(\si{\coulomb}\))                                                                                                                                             \\ \hline
	\(q, q_{\textnormal{free}}+q_{\textnormal{bound}}\)          & Electric charge (in \(\si{\coulomb}\))                                                                                                                                                   \\ \hline
	\(\rho_{\textnormal{free}}\)                                 & Free electric charge density                                                                                                                                                             \\ \hline
	\(\rho_{\textnormal{bound}}\)                                & Electric charge density                                                                                                                                                                  \\ \hline
	\(\rho, \rho_{\textnormal{free}}+\rho_{\textnormal{bound}}\) & Electric charge density (it can be in \(\si{\coulomb\per\meter^3}, \si{\coulomb\per\meter^2}\) or \(\si{\coulomb\per\meter}\) depending whether it is a volume, surface, or line shapes) \\ \hline
	\(\mathbf{f}\)                                               & Electrostatic force (Coulomb force), (in \(\si{\kilo\gram\meter\per\second\squared}\))                                                                                                   \\ \hline
	\(\varepsilon\)                                              & Electric permittivity(in \(\si{\farad\per\meter}\)) \cite{ramoFieldsWavesCommunication1994}                                                                                              \\ \hline
	\(\varepsilon_r\)                                            & Relative electric permittivity or dielectric constant (in \(\si{\farad\per\meter}\)) \cite{ramoFieldsWavesCommunication1994}                                                             \\ \hline
	\(\varepsilon_0\)                                            & Electric permittivity in vacuum, \(\SI{8.854e-12}{\farad\per\meter}\) \cite{ramoFieldsWavesCommunication1994}                                                                            \\ \hline
	\(\mathbf{E}\)                                               & Electric field vector (in \(\si{\volt\per\meter}\))                                                                                                                                      \\ \hline
	\(\mathbf{D}\)                                               & Electric flux density, electric displacement, or electric induction vector (in \(\si{\coulomb\per\meter\squared}\))                                                                      \\ \hline
	\(\Phi_D, \varPsi, \oiint_S \mathbf{D} \dd{\mathbf{S}}\)     & Electric flux (\(\mathbf{D}\)-filed flux) \cite{wiki:D-field-flux}                                                                                                                       \\ \hline
	\(\Phi_E, \oiint_S \mathbf{E} \dd{\mathbf{S}}\)              & Electric flux (\(\mathbf{E}\)-filed flux) \cite{wiki:electric-flux}                                                                                                                      \\ \hline
	\(\mathbf{P}\)                                               & Electric polarization of the material (in \(\si{\coulomb\per\meter\squared}\))                                                                                                           \\ \hline
	\(\chi_e\)                                                   & Electric susceptibility (for linear and isotropic materials)                                                                                                                             \\ \hline
	\(\mu\)                                                      & Magnetic permeability                                                                                                                                                                    \\ \hline
	\(\mu_0\)                                                    & Magnetic permeability in vacuum                                                                                                                                                          \\
\end{xltabular}

\section{Generic mathematical symbols}
\begin{xltabular}{\textwidth}{XX}
	\(\blacksquare\)   & Q.E.D.                                            \\ \hline
	\(\triangleq\)     & Equal by definition                               \\ \hline
	\(:=, \leftarrow\) & Assignment \cite{rosenDiscreteMathematicsIts2011} \\ \hline
	\(\neq\)           & Not equal                                         \\ \hline
	\(\infty\)         & Infinity                                          \\ \hline
	\(j\)              & \(\sqrt{-1}\)                                     \\
\end{xltabular}

\section{Abbreviations}
PS: Only names of techniques and algorithms or usual abbreviations are considered.
\begin{xltabular}{\textwidth}{XX}
	wrt. & With respect to                                                                      \\ \hline
	st.  & Subject to                                                                           \\ \hline
	iff. & If and only if                                                                       \\ \hline
	EVD  & Eigenvalue decomposition, or eigendecomposition \cite{nossekAdaptiveArraySignal2015} \\ \hline
	SVD  & Singular value decomposition                                                         \\ \hline
	CP   & CANDECOMP/PARAFAC                                                                    \\ \hline
	SGD  & Stochastic gradient descent                                                          \\ \hline
	SVM  & Support vector machine                                                               \\ \hline
	BPNN & Backpropagation neural network \cite{jiaoAutomaticEquatorialGPS2017}                 \\ \hline
	RBF  & Radial basis function                                                                \\ \hline
\end{xltabular}

\printbibliography

\end{document}
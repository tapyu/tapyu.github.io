% put %%%%%%%%% new commands %%%%%%%%%
\usepackage{xparse} % make command behave differently depending on the number of arguments
\usepackage{xstring} % testing a string's contents, extracting substrings, substitution of substrings

%%%%%%%%% items, tables and figs %%%%%%%%%
\makeatletter
\@ifclassloaded{beamer}{}{\usepackage{enumitem}}
\makeatother
\usepackage{xltabular}
\usepackage{float} % use the [H] option
\usepackage{tikz}
\usepackage{graphicx} % extends the basic LaTeX graphic capabilities: placement, scaling, rotation, file formats
\usepackage{standalone} % place tikz environments or other material in own source files

%%%%%%%%% references %%%%%%%%%
\makeatletter
\@ifclassloaded{beamer}{}{
	\usepackage[colorlinks=true]{hyperref} % make references blue and clickable
	\hypersetup{ % you can use allcolors=blue in hyperref options instead this setup. Color it as black to disable it
		linkcolor=blue,          % For \gls{} entries
		citecolor=blue,          % For \cite{} entries
	}
}
\makeatother
\usepackage{glossaries} % \newglossaryentry{} entries. If you want to make the entries nonclickable, load this package before hyperref
\usepackage[nonumberlist]{glossaries-extra}
%%%%%%%%% bibliography %%%%%%%%%
\usepackage[style=numeric-comp,backend=biber]{biblatex} % tells LaTeX to use the biber tool instead of the traditional bibtex tool for sorting and formatting your bibliography. Biber is a more advanced and powerful tool than bibtex and it provides more features, such as advanced sorting, Unicode support and support for various data sources.
\addbibresource{/home/tapyu/.cache/zotero/refs.bib} % add reference file, in the document, add \printbibliography to print bibliography

%%%%%%%%% math %%%%%%%%%
\usepackage{amsmath}
\usepackage{mathtools} % is an extension package to amsmath to use \DeclarePairedDelimiter
\usepackage{amsfonts}
\usepackage{amssymb}
\usepackage{esint} % for \oiint
\usepackage{newtxmath} % for Greek variants (bold, nonitalic, etc...)
\usepackage{stmaryrd} % provides new symbols, such as \llbracket \rrbracket
\usepackage{bm} % for writing tensor, e.g., $\bm{\mathcal{A}}$
\usepackage{IEEEtrantools} % provides the IEEEeqnarray environment. WARNING: this package may crash your build if the \documentclass is IEEE-like, e.g., IEEEtran. In this case, delete this line!
\usepackage{siunitx} % typesetting units and unitless numbers SI (Système International d’Unités) units (PS: use \SI{} instead \qty{} to avoid conflicts with the same command from the package physics)
% use \SI for number and unit, e.g., \SI{10}{\meter\per\second}
% use \si for unit only, e.g., \si{\meter\per\second}
% use \SIrange[options]{value1}{value2}{unit commands} for a range of values, possible options:
% 1. "range-phrase=-" ->  change the word "to" to "--"
% 2. range-units=single -> chage "1m to 10m" to "1 to 10m"
% 3. mode=text % writes the unit in full
% use the [per-mode=symbol] option in the \si or \SI command to change the writing of \per, or change the line below to set it globally
\sisetup{per-mode=symbol}
% \squared -> ^2

\usepackage{physics} % several math macros, such as
% \tr -> trace
% \rank -> rank
% \expval -> angle brackets to inner product,〈a,b〉, or ensemble average〈s(t)〉
% \abs -> |x|
% \norm -> ||x||
% \eval -> evaluated bar x|_y=a
% \order -> Big-O notation
% \Re -> real
% \Im -> imaginary
% \dd[]{} -> differential
% \dv -> derivative
% \pdv -> partial derivative

%%%%%%%%% operators %%%%%%%%%
\let\oldemptyset\emptyset % change empty set
\let\emptyset\varnothing
% circular convolution a lá Oppenheim (if possible, prefer \circledast)
\newcommand*\circconv[1]{%
	\begin{tikzpicture}
		\node[draw,circle,inner sep=1pt] {#1};
	\end{tikzpicture}}
\newcommand{\intersection}{\bigcap\limits} % intersection operator

%%%%%%%%% delimiters %%%%%%%%%
\makeatletter % changes the catcode of @ to 11
\DeclarePairedDelimiter\ceil{\lceil}{\rceil} % ⌈x⌉
\let\oldceil\ceil
\def\ceil{\@ifstar{\oldceil}{\oldceil*}} % swap the asterisk and the nonasterisk behaviors

\DeclarePairedDelimiter\floor{\lfloor}{\rfloor} % ⌊x⌋
\let\oldfloor\floor
\def\floor{\@ifstar{\oldfloor}{\oldfloor*}}
\makeatother % changes the catcode of @ back to 12

%%%%%%%%% simple functions %%%%%%%%%
\newcommand{\adj}[1]{\ensuremath{\operatorname{adj}\left(#1\right)}} % adjugate matrix
\newcommand{\cof}[1]{\ensuremath{\operatorname{cof}\left(#1\right)}} % cofactor matrix
\newcommand{\eig}[1]{\ensuremath{\operatorname{eig}\left(#1\right)}} % eigenvalues
\newcommand{\nullspace}[1]{\ensuremath{\operatorname{N}\left(#1\right)}} % nullspace or kernel of the matrix
\newcommand{\nullity}[1]{\ensuremath{\operatorname{nullity}\left(#1\right)}} % nullity=dim(N(A))
\newcommand{\spn}[1]{\ensuremath{\operatorname{span}\left\{#1\right\}}} % span of a set of vectors, \span is a reserved word
\newcommand{\range}[1]{\ensuremath{\operatorname{C}\left(#1\right)}} % range or columnspace of a matrix=span(a1,a2, ..., an), where ai is the ith column vector of the matrix A
\newcommand{\unvec}[1]{\ensuremath{\operatorname{unvec}\left(#1\right)}} % unvectorize operator
\newcommand{\diag}[1]{\ensuremath{\operatorname{diag}\left(#1\right)}} % diagonal operator
\newcommand{\dom}[1]{\ensuremath{\operatorname{dom}\left(#1\right)}} % domain of the function
\newcommand{\frob}[1]{\ensuremath{\norm{#1}_\textrm{F}}} % Frobenius norm
\renewcommand{\dim}[1]{\ensuremath{\operatorname{dim}\left(#1\right)}} % dimension of a set

%%%%%%%%% complex functions %%%%%%%%%
\NewDocumentCommand{\argmin}{ e{_} m }{% \argmin_{x ∈ A}{f(x)} or \argmin{f(x)}
	\ensuremath{%
		\IfValueTF{#1}{%
			\underset{#1}{\operatorname{arg\,min}\;#2}
		}{% else
			\operatorname{arg\,min}\;#2
		}
	}
}
\NewDocumentCommand{\argmax}{ e{_} m }{% \argmax_{x ∈ A}{f(x)} or \argmax{f(x)}
	\ensuremath{%
		\IfValueTF{#1}{%
			\underset{#1}{\operatorname{arg\,max}\;#2}
		}{% else
			\operatorname{arg\,max}\;#2
		}
	}
}
% limit in the mean (see Van Tress, chapter 6)
\NewDocumentCommand{\limean}{ E{_}{{}} m }{ % \limean_{x ∈ A}{f(x)} or \limean{f(x)}
	\ensuremath{
		\IfValueTF{#1}{%
			\underset{#1}{\operatorname{l.i.m.}\;#2}
		}{% else
			\operatorname{l.i.m.}\;#2
		}
	}
}
% \E{A}, \E_{u}{A}
% if you want only the left or right part of \E (useful for breaking line) you can use \E_{u}[left-only]{A} or \E_{u}[right-only]{A}, respectively
\NewDocumentCommand{\E}{ E{_}{{}} o m }{%
\ensuremath{%
\IfValueTF{#2}{%
\IfStrEq{#2}{left-only}{%
\operatorname{E}_{#1}\left[#3\right.%
}{% right-only
\left.#3\right]%
}%
}{%
\operatorname{E}_{#1}\left[#3\right]%
}%
}
}
\NewDocumentCommand{\cov}{O{} m}{\ensuremath{\operatorname{cov}_{#1}\left[#2\right]}} % covariance, e.g., \cov[u]{x} or \cov{x}
\let\var\undefined\NewDocumentCommand{\var}{O{} m}{\ensuremath{\operatorname{var}_{#1}\left[#2\right]}} % variance, e.g., \var[u]{x} or \var{x}
\let\vec\undefined\NewDocumentCommand{\vec}{O{} m}{\ensuremath{\operatorname{vec}_{#1}\left[#2\right]}} % vectorize operator, e.g., \vec[u]{\mathbf{A}}, [u] is optional
% normalize all trigonometry functions (new and predefined) to put a parenthesis around the input argument
\RenewDocumentCommand{\sin}{s m}{% \sin{x} -> sin(x) \sin*{x} -> sin x
	\ensuremath{
		\IfBooleanTF{#1}{\operatorname{sin} #2}{\operatorname{sin}\left(#2\right)}%
	}
}
\RenewDocumentCommand{\cos}{s m}{% \cos{x} -> cos(x) \cos*{x} -> cos x
	\ensuremath{
		\IfBooleanTF{#1}{\operatorname{cos} #2}{\operatorname{cos}\left(#2\right)}%
	}
}
\RenewDocumentCommand{\tan}{s m}{% \tan{x} -> tan(x) \tan*{x} -> tan x
	\ensuremath{
		\IfBooleanTF{#1}{\operatorname{tan} #2}{\operatorname{tan}\left(#2\right)}%
	}
}
\NewDocumentCommand{\sinc}{s m}{% \sin{x} -> sinc(x) \sinc*{x} -> sinc x
	\ensuremath{
		\IfBooleanTF{#1}{\operatorname{sinc} #2}{\operatorname{sinc}\left(#2\right)}%
	}
}
\NewDocumentCommand{\sgn}{s m}{% \sgn{x} -> sgn(x) \sng*{x} -> sgn x
	\ensuremath{
		\IfBooleanTF{#1}{\operatorname{sng} #2}{\operatorname{sgn}\left(#2\right)}%
	}
}

%%%%%%%%% minor adjusts in some mathematical symbols %%%%%%%%%
\DeclareMathAlphabet{\mathcal}{OMS}{cmsy}{m}{n} % newtxmath changes the font style of \mathcal. It prevents \mathcal from being changed
\let\oldforall\forall % put some spaces between the \forall command
\renewcommand{\forall}{\;\oldforall\;}

%%%%%%%%% comments, corrections, or revisions %%%%%%%%%
\usepackage{luacolor} % color support based on LuaTEX’s
\usepackage[soul]{lua-ul} % provide underlining, strikethrough, and highlighting using features in LuaLATEX which avoid the restrictions imposed by other methods
\newcommand{\obs}[1]{\textcolor{red}{(#1)}} % comment
\newcommand{\sizecorr}[1]{\makebox[0cm]{\phantom{$\displaystyle #1$}}} % Used to seize the height of equation
\newcommand{\ensureoperation}{\negmedspace {}} % To ensure that a new line symbol is an operation instead of a sign on your main command
%%%%%%%%% new commands %%%%%%%%%
\usepackage{xparse} % make command behave differently depending on the number of arguments

%%%%%%%%% items, tables and figs %%%%%%%%%
\usepackage{enumitem}
\usepackage{xltabular}
\usepackage{float} % use the [H] option
\usepackage{tikz}
\usepackage{graphicx} % extends the basic LaTeX graphic capabilities: placement, scaling, and rotation, file formats
\usepackage{standalone} % place tikz environments or other material in own source files

%%%%%%%%% refs %%%%%%%%%
\usepackage{hyperref} % make references clickable
\usepackage[style=numeric-comp,backend=biber]{biblatex} % tells LaTeX to use the biber tool instead of the traditional bibtex tool for sorting and formatting your bibliography. Biber is a more advanced and powerful tool than bibtex and it provides more features, such as advanced sorting, Unicode support and support for various data sources.
\addbibresource{/home/tapyu/.cache/zotero/refs.bib} % add reference file, in the document, add \printbibliography to print bibliography

%%%%%%%%% math %%%%%%%%%
\usepackage{amsmath}
\usepackage{mathtools} % is an extension package to amsmath to use \DeclarePairedDelimiter
\usepackage{amsfonts}
\usepackage{amssymb}
\usepackage{newtxmath} % for Greek variants (bold, nonitalic, etc...)
\usepackage{stmaryrd} % provides new symbols, such as \llbracket \rrbracket
\usepackage{bm} % for writing tensor, e.g., $\bm{\mathcal{A}}$
\usepackage{IEEEtrantools}
\usepackage{siunitx} % typesetting units and unitless numbers SI (Système International d’Unités) units (PS: use \SI{} instead \qty{} to avoid conflicts with the same command from the package physics)
% use \SI for number and unit, e.g., \SI{10}{\meter\per\second}
% use \si for unit only, e.g., \si{\meter\per\second}
% use \SIrange[options]{value1}{value2}{unit commands} for a range ot values, possible options:
% 1. "range-phrase=-" ->  change the word "to" to "--"
% 2. range-units=single -> chage "1m to 10m" to "1 to 10m"
% use the [per-mode=symbol] option in the \si or \SI command to change the writing of \per, or change the line below to set it globally
\sisetup{per-mode=symbol}
% \squared -> ^2

\usepackage{physics} % several math macros, such as
% \tr -> trace
% \rank -> rank
% \expval -> angle brackets to inner product,〈a|b〉, or ensamble average〈s (t)〉
% \abs -> |x|
% \norm -> ||x||
% \eval -> evaluated bar x|_y=a
% \order -> Big-O notation
% \Re -> real
% \Im -> imaginary
% \dd[]{} -> differential
% \dv -> derivative
% \pdv -> partial derivative

%%%%%%%%% creating new operators %%%%%%%%%
\let\oldemptyset\emptyset % change empty set
\let\emptyset\varnothing

% for simple operator definitions, use \newcommand{}[]{\operatorname{}}
\newcommand{\adj}[1]{\ensuremath{\operatorname{adj}\left(#1\right)}} % adjugate matrix
\newcommand{\cof}[1]{\ensuremath{\operatorname{cof}\left(#1\right)}} % cofactor matrix
\newcommand{\eig}[1]{\ensuremath{\operatorname{eig}\left(#1\right)}} % eigenvalues
\newcommand{\nullspace}[1]{\ensuremath{\operatorname{N}\left(#1\right)}} % nullspace or kernel of the matrix
\newcommand{\nullity}[1]{\ensuremath{\operatorname{nullity}\left(#1\right)}} % nullity=dim(N(A))
\newcommand{\spn}[1]{\ensuremath{\operatorname{span}\left\{#1\right\}}} % span of a set of vectors, \span is a reserved word
\newcommand{\range}[1]{\ensuremath{\operatorname{C}\left(#1\right)}} % range or columnspace of a matrix=span(a1,a2, ..., an), where ai is the ith column vector of the matrix A
\newcommand{\unvec}[1]{\ensuremath{\operatorname{unvec}\left(#1\right)}} % unvectorize operator
\newcommand{\diag}[1]{\ensuremath{\operatorname{diag}\left(#1\right)}} % diagonal operator
\newcommand{\dom}[1]{\ensuremath{\operatorname{dom}\left(#1\right)}} % domain of the function
\newcommand{\frob}[1]{\ensuremath{\norm{#1}_\textrm{F}}} % Frobenius norm
\newcommand{\intersection}{\bigcap\limits} % intersection operator
\renewcommand{\dim}[1]{\ensuremath{\operatorname{dim}\left(#1\right)}} % dimension of a set

% for more complex operator difinition with optional arguments, use \NewDocumentCommand{}{}{\operatorname}
\NewDocumentCommand{\argmin}{ O{} m }{\underset{#1}{\ensuremath{\operatorname{arg\,min}}\;#2}} % arg min, e.g., \argmin[x \in \mathcal{A}]{f(x)}
\NewDocumentCommand{\argmax}{ O{} m }{\underset{#1}{\ensuremath{\operatorname{arg\,max}}\;#2}}
\NewDocumentCommand{\E}{O{} m}{\ensuremath{\operatorname{E}_{#1}\left[#2\right]}} % statistical expectation operator, e.g., \E[u]{x} or \E{x}
\NewDocumentCommand{\cov}{O{} m}{\ensuremath{\operatorname{cov}_{#1}\left[#2\right]}} % covariance, e.g., \cov[u]{x} or \cov{x}
\let\var\undefined\NewDocumentCommand{\var}{O{} m}{\ensuremath{\operatorname{var}_{#1}\left[#2\right]}} % variance, e.g., \var[u]{x} or \var{x}
\let\vec\undefined\NewDocumentCommand{\vec}{O{} m}{\ensuremath{\operatorname{\bf E}_{#1}\left[#2\right]}} % vectorize operator, e.g., \vec[u]{\mathbf{A}}, [u] is optional

% circular convolution a lá Oppenheim (if possible, prefer \circledast)
\newcommand*\circconv[1]{%
  \begin{tikzpicture}
    \node[draw,circle,inner sep=1pt] {#1};
  \end{tikzpicture}}

%%%%%%%%% creating delimited operators %%%%%%%%%
\makeatletter % changes the catcode of @ to 11
\DeclarePairedDelimiter\ceil{\lceil}{\rceil} % ⌈x⌉
\let\oldceil\ceil
\def\ceil{\@ifstar{\oldceil}{\oldceil*}} % swap the asterist and the nonasterisk behaviors

\DeclarePairedDelimiter\floor{\lfloor}{\rfloor} % ⌊x⌋
\let\oldfloor\floor
\def\floor{\@ifstar{\oldfloor}{\oldfloor*}}

\makeatother % changes the catcode of @ back to 12

%%%%%%%%% minor adjusts in some mathematical symbols %%%%%%%%%
\let\oldforall\forall % put some spaces between the \forall command
\renewcommand{\forall}{\;\oldforall\;}

%%%%%%%%% comments, corrections, or revisions %%%%%%%%%
\usepackage{luacolor} % color support based on LuaTEX’s
\usepackage[soul]{lua-ul} % provide underlining, strikethough, and highlighting using features in LuaLATEX which avoid the restrictions imposed by other methods
\newcommand{\obs}[1]{\textcolor{red}{(#1)}} % comment
\newcommand{\sizecorr}[1]{\makebox[0cm]{\phantom{$\displaystyle #1$}}} % Used to seize the height of equation
\newcommand{\ensureoperation}{\negmedspace {}} % To ensure that a new line symbol is an operation instead of a sign
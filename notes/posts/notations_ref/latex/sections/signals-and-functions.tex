\section{Signals and functions}
\subsection{Time indexing}
\begin{xltabular}{\textwidth}{p{5.7cm}X}
	\(x(t)\)                                                                                                                                                                                      & Continuous-time \(t\)                                                                                                                              \\ \hline
	\(x[n], x[k], x[m], x[i], \dots\) \(x_n, x_k, x_m, x_i, \dots\) \(x(n), x(k), x(m), x(i), \dots\)                                                                                             & Discrete-time \(n, k, m, i, \dots\) (parenthesis should be adopted only if there are no continuous-time signals in the context to avoid ambiguity) \\ \hline
	\(x\left[ \left( \left( n - m \right) \right)_N \right]\)\cite{oppenheimDiscreteTimeSignalProcessing2009}, \(x \left( \left( n - m \right) \right)_N\)\cite{ingleDigitalSignalProcessing2000} & Circular shift in \(m\) samples within a \(N\)-samples window
\end{xltabular}
\subsection{Common signals}
\begin{xltabular}{\textwidth}{XX}
	\(\delta(t)\)                  & Delta function                                  \\ \hline
	\(\delta[n], \delta_{i,j}\)    & Kronecker function (\(n = i-j\))                \\ \hline
	\(h(t), h[n]\)                 & Impulse response (continuous and discrete time) \\ \hline
	\(\tilde{x}[n], \tilde{x}(t)\) & Periodic discrete- or continuous-time signal    \\ \hline
	\(\hat{x}[n], \hat{x}(t)\)     & Estimate of \(x[n]\) or \(x(t)\)                \\ \hline
	\(\dot{x}[m]\)                 & Interpolation of \(x[n]\)                       \\
\end{xltabular}
\subsection{Common functions}
\begin{xltabular}{\textwidth}{XX}
	\(\mathcal{O}(\cdot), O(\cdot)\) & Big-O notation                                                  \\ \hline
	\(\Gamma(\cdot)\)                & Gamma function                                                  \\ \hline
	\(Q(\cdot)\)                     & Quantization function                                           \\ \hline
	\(\textnormal{sgn}(\cdot)\)      & Signum function                                                 \\ \hline
	\(\textnormal{tanh}(\cdot)\)     & Hyperbolic tangent function                                     \\ \hline
	\(I_\alpha(\cdot)\)              & Modified Bessel function of the first kind and order \(\alpha\) \\ \hline
	\(\left( \begin{array}{cc}
			n \\
			k
		\end{array} \right)\)       & Binomial coefficient
\end{xltabular}
\subsection{Operations and symbols}
\begin{xltabular}{\textwidth}{XX}
	\(f: A \rightarrow B\)                                                                                                     & A function \(f\) whose domain is \(A\) and codomain is \(B\)                                                                                                                                                                                                  \\ \hline
	\(\mathbf{f}: A \rightarrow \mathbb{R}^n\)                                                                                 & A vector-valued function \(\mathbf{f}\), i.e., \(n \geq 2\)                                                                                                                                                                                                   \\ \hline
	\(f^{n}, x^{n}(t), x^{n}[k]\)                                                                                              & \(n\)th power of the function \(f\), \(x[n]\) or \(x(t)\)                                                                                                                                                                                                     \\ \hline
	\(f^{\left( n \right)},  x^{(n)}(t)\)                                                                                      & \(n\)th derivative of the function \(f\) or \(x(t)\)                                                                                                                                                                                                          \\ \hline
	\(f', f^{\left( 1 \right)}, x'(t)\)                                                                                        & \(1\)th derivative of the function \(f\) or \(x(t)\)                                                                                                                                                                                                          \\ \hline
	\(f'', f^{\left( 2 \right)}, x''(t)\)                                                                                      & \(2\)th derivative of the function \(f\) or \(x(t)\)                                                                                                                                                                                                          \\ \hline
	\(\argmax[x \in \mathcal{A}]{f(x)} \)                                                                                      & Value of \(x\) that minimizes \(x\)                                                                                                                                                                                                                           \\ \hline
	\( \argmin[x \in \mathcal{A}]{f(x)} \)                                                                                     & Value of \(x\) that minimizes \(x\)                                                                                                                                                                                                                           \\ \hline
	\(f(\mathbf{x}) = \underset{\mathbf{y} \in \mathcal{A}}{\textnormal{inf }} g(\mathbf{x},\mathbf{y})\)                      & Infimum, i.e., \(f(\mathbf{x}) = \min{\left\{ g(\mathbf{x}, \mathbf{y}) \mid \mathbf{y} \in \mathcal{A} \wedge \left( \mathbf{x}, \mathbf{y} \right) \in \dom{g} \right\}}\), which is the greatest lower bound of this set \cite{boydConvexOptimization2004} \\ \hline
	\(f(\mathbf{x}) = \underset{\mathbf{y} \in \mathcal{A}}{\textnormal{sup }} g(\mathbf{x},\mathbf{y})\)                      & Supremum, i.e., \(f(\mathbf{x}) = \max{\left\{ g(\mathbf{x}, \mathbf{y}) \mid \mathbf{y} \in \mathcal{A} \wedge \left( \mathbf{x}, \mathbf{y} \right) \in \dom{g} \right\}}\), which is the least upper bound of this set \cite{boydConvexOptimization2004}   \\ \hline
	\(f \circ g\)                                                                                                              & Composition of the functions \(f\) and \(g\)                                                                                                                                                                                                                  \\ \hline
	\(*\)                                                                                                                      & Convolution (discrete or continuous)                                                                                                                                                                                                                          \\ \hline
	\(\circledast\) \cite{dinizDigitalSignalProcessing2010}, \(\circconv{N}\) \cite{oppenheimDiscreteTimeSignalProcessing2009} & Circular convolution                                                                                                                                                                                                                                          \\
\end{xltabular}
\subsection{Digital signal processing}
\begin{xltabular}{\textwidth}{XX}
	\(W_N\)                                                                                         & Twiddle factor, \(e^{-j\frac{2\pi}{N}}\) \cite{ingleDigitalSignalProcessing2000}                                                                                                                            \\ \hline
	\(N\)                                                                                           & Number of samples in the DFT/FFT                                                                                                                                                                            \\ \hline
	\(\Omega\) \cite{ingleDigitalSignalProcessing2000}                                              & Continuous angular frequency (in \(\si{\radian\per\second}\))                                                                                                                                               \\ \hline
	\(\omega\)                                                                                      & Discrete angular frequency. As \(\omega\) is also used to denote continuous angular frequency outside the DSP context, it is always convenient to state that it denotes the discrete frequency when it does \\ \hline
	\(f_c\)                                                                                         & Continuous linear frequency (in \(\si{\hertz}\))                                                                                                                                                            \\ \hline
	\(f\)                                                                                           & Discrete linear frequency. As \(f\) is also used to denote continuous linear frequency outside the DSP context, it is always convenient to state that it denotes the discrete frequency when it does        \\ \hline
	\(\mathcal{R}_N [n]\)                                                                           & Rectangular window used to cut off the discrete sequences \cite{ingleDigitalSignalProcessing2000}                                                                                                           \\ \hline
	\(T\)\cite{oppenheimDiscreteTimeSignalProcessing2009}, \(T_s\)                                  & Sampling period                                                                                                                                                                                             \\ \hline
	\(f_s\)                                                                                         & Sampling frequency (in \(\si{\hertz}\)), i.e., \(1/T\)                                                                                                                                                      \\ \hline
	\(\Omega_s\)                                                                                    & Sampling frequency (in \(\si{\radian\per\second}\)), i.e., \(2\pi f_s\)                                                                                                                                     \\ \hline
	\(\Omega_N\) \cite{oppenheimDiscreteTimeSignalProcessing2009}, \(B\)                            & One-sided effective bandwidth of the continuous-time signal spectrum                                                                                                                                        \\ \hline
	\(\omega_s\)                                                                                    & Stop frequency \cite{ingleDigitalSignalProcessing2000}                                                                                                                                                      \\ \hline
	\(\omega_p\)                                                                                    & Pass frequency \cite{ingleDigitalSignalProcessing2000}                                                                                                                                                      \\ \hline
	\(\Delta \omega\)                                                                               & \(\omega_s - \omega_p\) \cite{ingleDigitalSignalProcessing2000}                                                                                                                                             \\ \hline
	\(\omega_c\)                                                                                    & Cutoff frequency \cite{ingleDigitalSignalProcessing2000}                                                                                                                                                    \\ \hline
	\(s(t)\)                                                                                        & Impulse train                                                                                                                                                                                               \\ \hline
	\(\textrm{gdr}\left[ H (e^{j\omega}) \right]\) \cite{oppenheimDiscreteTimeSignalProcessing2009} & Group delay of \(H (e^{j\omega})\)                                                                                                                                                                          \\ \hline
	\(\angle H (e^{j\omega})\) \cite{oppenheimDiscreteTimeSignalProcessing2009}                     & Phase response of \(H (e^{j\omega})\)                                                                                                                                                                       \\ \hline
	\(\abs{H (e^{j\omega})}\) \cite{oppenheimDiscreteTimeSignalProcessing2009}                      & Magnitude (or gain) of \(H (e^{j\omega})\)                                                                                                                                                                  \\ \hline
	\(x_c(t)\) \cite{oppenheimDiscreteTimeSignalProcessing2009}, \(x(t)\)                           & Continuous-time signal                                                                                                                                                                                      \\ \hline
	\(x_s(t)\)                                                                                      & Sampled version of \(x(t)\), i.e., \(x(t)s(t)\)                                                                                                                                                             \\ \hline
	\(x_r(t)\)                                                                                      & Reconstruction of \(x(t)\) from interpolation                                                                                                                                                               \\ \hline
	\(\tilde{x}[n]\)                                                                                & Periodic extension of the the aperiodic signal \(x[n]\)                                                                                                                                                     \\ \hline
\end{xltabular}
\subsection{Transformations}
\begin{xltabular}{\textwidth}{XX}
	\(\mathcal{F}\left\{ \cdot \right\}\)                                                                                   & Fourier transform (FT)                                                                                                                        \\ \hline
	\(\mathrm{DTFT}\left\{ \cdot \right\}\), \(\mathrm{DFS}\left\{ \cdot \right\}\), \(\mathrm{FFT}\left\{ \cdot \right\}\) & Discrete-time Fourier Transform (DTFT), Discrete Fourier Transform (DFT), Discrete Fourier Series (DFS), respectively                         \\ \hline
	\(\mathcal{L}\left\{ \cdot \right\}\)                                                                                   & Laplace transform                                                                                                                             \\ \hline
	\(\mathcal{Z}\left\{ \cdot \right\}\)                                                                                   & \(z\)-transform                                                                                                                               \\ \hline
	\(\hat{x}(t), \hat{x}[n]\)                                                                                              & Hilbert transform of \(x(t)\) or \(x[n]\)                                                                                                     \\ \hline
	\(X(s)\)                                                                                                                & Laplace transform of \(x(t)\)                                                                                                                 \\ \hline
	\(X(f)\)                                                                                                                & Fourier transform (FT) (in linear frequency, \(\unit{\Hz}\)) of \(x(t)\)                                                                      \\ \hline
	\(X(j\omega)\)                                                                                                          & Fourier transform (FT) (in angular frequency, \(\unit{\radian\per\sec}\)) of \(x(t)\)                                                         \\ \hline
	\(X(e^{j\omega})\)                                                                                                      & Discrete-time Fourier transform (DTFT) of \(x[n]\)                                                                                            \\ \hline
	\(X[k], X(k), X_k\)                                                                                                     & Discrete Fourier transform (DFT) or fast Fourier transform (FFT) of \(x[n]\), or even the Fourier series (FS) of the periodic signal \(x(t)\) \\ \hline
	\(\tilde{X}[k], \tilde{X}(k), \tilde{X}_k\)                                                                             & Discrete Fourier series (DFS) of \(\tilde{x}[n]\)                                                                                             \\ \hline
	\(X(z)\)                                                                                                                & \(z\)-transform of \(x[n]\)                                                                                                                   \\
\end{xltabular}
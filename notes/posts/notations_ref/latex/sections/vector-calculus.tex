\section{Vector Calculus}
\begin{xltabular}{\textwidth}{XX}
	\(\nabla f\)\cite{stewartCalculus2011}, \(\textrm{grad} f\)\cite{ramoFieldsWavesCommunication1994}                                                                                                                                                                                                                                 & Vector differential operator (Nabla symbol), i.e., \(\nabla f\) is the gradient of the scalar-valued function \(f\), i.e., \(f: \mathbb{R}^n \rightarrow \mathbb{R}\)                                                                                                                                                               \\ \hline
	\(t, (u,v)\)                                                                                                                                                                                                                               & Parametric variables commonly used, \(t\) for one variable, \((u,v)\) for two variables\cite{stewartCalculus2011}                                                                                                                                                                                                                   \\ \hline
	\(\mathbf{l}(x,y,z)\) \cite{ramoFieldsWavesCommunication1994}, \(\mathbf{r}(x,y,z)\) \cite{stewartCalculus2011}, \(x \hat{\mathbf{i}} + y \hat{\mathbf{j}} + z \hat{\mathbf{k}}\)                                                                                                                                                                                           & Vector position, i.e., \((x, y, z)\). \\ \hline
	\(\mathbf{l}(t)\)                                                                                                                                                                                                                          & Vector position parametrized by \(t\), i.e., \((x(t), y(t), z(t))\) \cite{stewartCalculus2011,ramoFieldsWavesCommunication1994}                                                                                                                                                                                                     \\ \hline
	\(\mathbf{l}'(t), \dd{\mathbf{l}}/\dd{t}\)                                                                                                                                                                                                 & First derivative of \(\mathbf{l}(t)\), i.e., the tangent vector of the curve \((x(t), y(t), z(t))\) \cite{stewartCalculus2011}                                                                                                                                                                                                      \\ \hline
	\(\mathbf{u}(t)\)\cite{kreyszigAdvancedEngineeringMathematics2008} \(\mathbf{T}(t)\)\cite{stewartCalculus2011}, \(\dd{\mathbf{l}}(t)\)\cite{ramoFieldsWavesCommunication1994}                                                                   & Tangent unit vector of \(\mathbf{l}(t)\), i.e., \newline  \(\mathbf{u}(t) = \mathbf{l}'(t)/\abs{\mathbf{l}'(t)}\) \\ \hline
	\(\mathbf{n}(t), \left( \frac{y'(t)}{\abs{\mathbf{l}'(t)}}, -\frac{x'(t)}{\abs{\mathbf{l}'(t)}} \right)\)                                                                                                                                  & Normal vector of \(\mathbf{l}(t)\), i.e., \newline \(\mathbf{n}(t)\perp \mathbf{T}(t) \)\cite{stewartCalculus2011}                                                                                                                                                                                                                  \\ \hline
	\(C\)                                                                                                                                                                                                                                      & Contour that traveled by \(\mathbf{l}(t)\), for \(a \leq t \leq b\) \cite{stewartCalculus2011}                                                                                                                                                                                                                                      \\ \hline
	\(L, L(C)\)                                                                                                                                                                                                                                & Total length of the contour \(C\) (which can be defined the vector \(\mathbf{l}\), parametrized by \(t\)), i.e., \(L_C = \int_a^b \abs{\mathbf{l}'(t)} \dd{t}\)\cite{stewartCalculus2011}                                                                                                                                           \\ \hline
	\(s(t)\)                                                                                                                                                                                                                                   & Length of the arc, which can be defined by the vector \(\mathbf{l}\) and \(t\), that is, \(s(t) = \int_a^t \abs{\mathbf{l}'(u)} \dd{u}\) (\(s(b) = L\))\cite{stewartCalculus2011}                                                                                                                                                   \\ \hline
	\(\dd{s}\)                                                                                                                                                                                                                                 & Differential operator of the length of the contour \(C\), i.e., \(\dd{s} = \abs{\mathbf{l}'(t)} \dd{t}\) \cite{stewartCalculus2011}                                                                                                                                                                                                 \\ \hline
	\(\int_C f(\mathbf{l}) \dd{s}\), \(\int_a^b f(\mathbf{l}(t)) \abs{\mathbf{l}'(t)} \dd{t}\)                                                                                                                                                     & Line integral of the function \(f: \mathbb{R}^{n} \rightarrow \mathbb{R}\) along the contour \(C\). In the context of integrals in the complex plane, it is also called ``contour integral'' \\ \hline
    \(\theta\) \cite{ramoFieldsWavesCommunication1994} & Angle between the contour \(C\) and the vector field \(\mathbf{F}\) \\ \hline
	\(\int_C \mathbf{F}\cdot\dd{\mathbf{l}}\), \(\int_a^b \mathbf{F}(\mathbf{l}(t)) \cdot \mathbf{l}'(t) \dd{t}\) \cite{apostolCalculus2ndEdn1967,stewartCalculus2011}, \(\int_C \mathbf{F}\cdot\mathbf{u} \dd{s}\), \(\int_C \mathbf{F} \cos{\theta} \dd{s}\) \cite{ramoFieldsWavesCommunication1994}                                                                                         & Line integral of vector field \(\mathbf{F}\) along the contour \(C\)                                                                                                                                                                                                          \\ \hline
	\(\int_C \mathbf{F}\cdot\dd{\mathbf{u}}\) \cite{ramoFieldsWavesCommunication1994}                                                                                         & In the field of electromagnetics, it is common to apply the line integral between the vector field \(\mathbf{F}\) and the unit vector \(\mathbf{u}(t)\). Therefore, this line integral may appear as well                                                                                                                                                                                                            \\ \hline
	\(\int_\mathbf{a}^\mathbf{b} \mathbf{F}, \int_\mathbf{a}^\mathbf{b} \mathbf{F}\cdot\dd{\mathbf{l}}\)                                                                                                                                       & Alternative notation to the line integral, where the parametric variable \(t\) goes from \(a\) to \(b\), making \(r\) goes from \(\mathbf{l}(a) = \mathbf{a}\) to \(\mathbf{l}(b) = \mathbf{b}\) \cite{apostolCalculus2ndEdn1967}                                                                                                   \\ \hline
	\(\oint_C, \varointctrclockwise_C\)                                                                                                                                                                                                        & Line integral along the closed contour \(C\). The arrow indicates the contour integral orientation, which is counterclockwise, by default. In the context of integrals in the complex plane, it is also called ``closed contour integral''.                                                                                         \\ \hline
	\(\oiint_S\)                                                                                                                                                                                                                               & Surface integral over the closed surface \(S\)                                                                                                                                                                                                                                                                                      \\ \hline
	\(\mathbf{l}(u,v)\)                                                                                                                                                                                                                        & Vector position \((x(u,v), y(u,v), z(u,v))\) parametrized by \((u,v)\)                                                                                                                                                                                                                                                              \\ \hline
	\(\mathbf{l}_u\)                                                                                                                                                                                                                           & \((\partial x/ \partial u, \partial y/ \partial u, \partial z/ \partial u)\)                                                                                                                                                                                                                                                        \\ \hline
	\(\mathbf{l}_v\)                                                                                                                                                                                                                           & \((\partial x/ \partial v, \partial y/ \partial v, \partial z/ \partial v)\)                                                                                                                                                                                                                                                        \\ \hline
	\(\dd{A}\)                                                                                                                                                                                                                                 & Differential operator of a 2D area (denoted by \(D\) or \(R\)) in the \(\mathbb{R}^2\) domain. This differential operator can be solved in different ways (rectangular, polar, cylindric, etc) \cite{stewartCalculus2011}                                                                                                           \\ \hline
	\(D, R\)                                                                                                                                                                                                                                   & Integration domain in which \(\dd{A}\) is integrated, i.e., \(\iint_D f \dd{A}\) \cite{stewartCalculus2011}                                                                                                                                                                                                                         \\ \hline
	\(S\)                                                                                                                                                                                                                                      & Smooth surface \(S\), i.e., a 2D area in a 3D space (\(\mathbb{R}^3\) domain)                                                                                                                                                                                                                                                       \\ \hline
	\(\dd{S}, \abs{\mathbf{l}_u\times\mathbf{l}_v} \dd{A} \)                                                                                                                                                                                   & Differential operator of a 2D area in a 3D domain (an surface). Note that \(\dd{S} = \abs{\mathbf{l}_u\times\mathbf{l}_v} \dd{A}\) should be accompanied with the change of the integration interval(from \(S\) to \(D\))                                                                                                           \\ \hline
	\(A(S), \iint_S \dd{S}, \iint_D \abs{\mathbf{l}_u\times\mathbf{l}_v} \dd{A}\)                                                                                                                                                              & Area of the surface \(S\) parametrized by \((u,v)\), in which \(\dd{A}\) is the area defined in the \(D\) domain (which is form by the \(u\)-by-\(v\) graph)                                                                                                                                                                        \\ \hline
	\(\dd{V}\)                                                                                                                                                                                                                                 & Differential operator of a shape volume (denoted by \(E\)) in \(\mathbb{R}^3\) domain, i.e., \(\iiint_E \dd{V} = V\)                                                                                                                                                                                                                \\ \hline
	\(E\)                                                                                                                                                                                                                                      & Integration domain in which \(\dd{V}\) is integrated, i.e., \(\iiint_E f \dd{V}\) \cite{stewartCalculus2011}                                                                                                                                                                                                                        \\ \hline
	\(V, \iint_D f \dd{A}, \iiint_E f \dd{V}\)                                                                                                                                                                                                 & Volume of the function \(f\) over the regions \(D\) (in the case of double integrals) or \(E\) (in the case of triple integrals)                                                                                                                                                                                                    \\ \hline
	\(\iint_S f \dd{S}, \iint_D f \abs{\mathbf{l}_u\times\mathbf{l}_v} \dd{A}\)                                                                                                                                                                & Surface integral over \(S\)                                                                                                                                                                                                                                                                                                         \\ \hline
	\(\mathbf{n}(u,v), \frac{\mathbf{l}_u(u,v)\times\mathbf{l}_v(u,v)}{\abs{\mathbf{l}_u(u,v)\times\mathbf{l}_v(u,v)}}\)                                                                                                                       & Normal vector of of the smooth surface \(S\)                                                                                                                                                                                                                                                                                        \\ \hline
	\(\iint_S \mathbf{F}\cdot \mathbf{n} \dd{S}, \iint_S \mathbf{F} \cdot \dd{\mathbf{S}}, \newline \iint_D \mathbf{F} \cdot (\mathbf{l}_u\times\mathbf{l}_v) \dd{A}\)                                                                         & Flux integral of vector field \(\mathbf{F}\) through the smooth surface \(S\) (\(\mathbf{n} \dd{S} \triangleq \dd{\mathbf{S}}\))                                                                                                                                                                                                    \\ \hline
	\(\oiint_S \mathbf{F}\cdot \mathbf{n} \dd{S}, \oiint_S \mathbf{F} \cdot \dd{\mathbf{S}}, \newline \iint_D \mathbf{F} \cdot (\mathbf{l}_u\times\mathbf{l}_v) \dd{A}\)                                                                       & Flux integral of vector field \(\mathbf{F}\) through the smooth and closed surface \(S\) (\(\mathbf{n} \dd{S} \triangleq \dd{\mathbf{S}}\))                                                                                                                                                                                         \\ \hline
	\(\nabla \times \mathbf{F} , \textnormal{curl } \mathbf{F}\)                                                                                                                                                                               & Curl (rotacional) of the vector field \(\mathbf{F}\)                                                                                                                                                                                                                                                                                \\ \hline
	\(\nabla \cdot \mathbf{F} , \textnormal{div } \mathbf{F}\)                                                                                                                                                                                 & Divercence of the vector field \(\mathbf{F}\)                                                                                                                                                                                                                                                                                       \\ \hline
	\(\nabla^2 f, \nabla \cdot (\nabla f), \Delta f, \newline \partial^2f/\partial x^2 + \partial^2f/\partial y^2 + \partial^2f/\partial z^2\)                                                                                                 & Scalar Laplacian operator (performed on a scalar-valued function \(f: \mathbb{R}^{n} \rightarrow \mathbb{R}\))                                                                                                                                                                                                                      \\ \hline
	\(\nabla^2 \mathbf{F}, \nabla \times \nabla \times \mathbf{F} - \nabla(\nabla \cdot \mathbf{F}), \Delta \mathbf{F}, \newline (\partial^2\mathbf{F}/\partial x^2 , \partial^2\mathbf{F}/\partial y^2 , \partial^2\mathbf{F}/\partial z^2)\) & Vector Laplacian operator (performed on a vector field, i.e., a vector-valued function, \(\mathbf{F}: \mathbb{R}^{n} \rightarrow \mathbb{R}^{n}\)). \(\nabla^2\) denotes the scalar (vector) Laplacian if the function is scalar-valued (vector-valued). The notation \(\Delta\) must be avoided as it is overused in many contexts \\
\end{xltabular}

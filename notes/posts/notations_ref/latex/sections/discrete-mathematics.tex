\section{Discrete mathematics}
\subsection{Quantifiers, inferences}
\begin{xltabular}{\textwidth}{XX}
	\(\forall\)                  & For all (universal quantifier) \cite{grahamConcreteMathematicsFoundation1989}                                                                                                                                \\ \hline
	\(\exists\)                  & There exists (existential quantifier) \cite{grahamConcreteMathematicsFoundation1989}                                                                                                                         \\ \hline
	\(\nexists\)                 & There does not exist \cite{grahamConcreteMathematicsFoundation1989}                                                                                                                                          \\ \hline
	\(\exists!\)                 & There exists an unique \cite{grahamConcreteMathematicsFoundation1989}                                                                                                                                        \\ \hline
	\(\exists_{n}\)              & There exists exactly \(n\) \cite{rosenDiscreteMathematicsIts2011}                                                                                                                                            \\ \hline
	\(\in\)                      & Belongs to \cite{grahamConcreteMathematicsFoundation1989}                                                                                                                                                    \\ \hline
	\(\not\in\)                  & Does not belong to \cite{grahamConcreteMathematicsFoundation1989}                                                                                                                                            \\ \hline
	\(\because\)                 & Because \cite{grahamConcreteMathematicsFoundation1989}                                                                                                                                                       \\ \hline
	\(\mid, :\)                  & Such that, sometimes that parentheses is used \cite{grahamConcreteMathematicsFoundation1989}                                                                                                                 \\ \hline
	\(, , \left( \cdot \right)\) & Used to separate the quantifier with restricted domain from its scope, e.g., \(\forall \; x < 0 \left( x^{2} > 0 \right)\) or \(\forall \; x < 0, x^{2} > 0\) \cite{grahamConcreteMathematicsFoundation1989} \\ \hline
	\(\therefore\)               & Therefore \cite{grahamConcreteMathematicsFoundation1989}                                                                                                                                                     \\
\end{xltabular}

\subsection{Propositional Logic}
\begin{xltabular}{\textwidth}{XX}
	\(\lnot a\)                                    & Logical negation of \(a\) \cite{rosenDiscreteMathematicsIts2011}                                                                                       \\ \hline
	\(a \wedge b\)                                 & Conjunction (logical AND) operator between \(a\) and \(b\)\cite{rosenDiscreteMathematicsIts2011}                                                       \\ \hline
	\(a \vee b\)                                   & Disjunction (logical OR) operator between \(a\) and \(b\)\cite{rosenDiscreteMathematicsIts2011}                                                        \\ \hline
	\(a \oplus b\)                                 & Exclusive OR (logical XOR) operator between \(a\) and \(b\)\cite{rosenDiscreteMathematicsIts2011}                                                      \\ \hline
	\(a \rightarrow b\)                            & Implication (or conditional) statement\cite{rosenDiscreteMathematicsIts2011}                                                                           \\ \hline
	\(a \leftrightarrow b\)                        & Bi-implication (or biconditional) statement, i.e., \(\left( a \rightarrow b \right) \wedge (b \rightarrow a )\) \cite{rosenDiscreteMathematicsIts2011} \\ \hline
	\(a \equiv b, a \iff b, a \Leftrightarrow b \) & Logical equivalence, i.e., \(a \leftrightarrow b\) is a tautology\cite{rosenDiscreteMathematicsIts2011}                                                \\
\end{xltabular}

\subsection{Operations}
\begin{xltabular}{\textwidth}{XX}
	\(\abs{a}\)                                                                                                                                             & Absolute value of \(a\)                                                                                                     \\ \hline
	\(\log\)                                                                                                                                                & Base-10 logarithm or decimal logarithm                                                                                      \\ \hline
	\(\ln\)                                                                                                                                                 & Natual logarithm                                                                                                            \\ \hline
	\(\textnormal{Re}\left\{ x \right\}\)                                                                                                                   & Real part of \(x\)                                                                                                          \\ \hline
	\(\textnormal{Im}\left\{ x \right\}\)                                                                                                                   & Imaginary part of \(x\)                                                                                                     \\ \hline
	\(\angle\cdot\)                                                                                                                                         & Phase (complex argument)                                                                                                    \\ \hline
	\(x\;\mathrm{mod}\;y\)                                                                                                                                  & Remainder, i.e., \(x-y\floor{x/y}\), for \(y \neq 0\)                                                                       \\ \hline
	\(x\;\mathrm{div}\;y\)                                                                                                                                  & Quotient \cite{rosenDiscreteMathematicsIts2011}                                                                             \\ \hline
	\(x \equiv y\;(\mathrm{mod}\;m)\)                                                                                                                       & Congruent, i.e.,  \(m \backslash (x-y)\) \cite{rosenDiscreteMathematicsIts2011}                                             \\ \hline
	\(\mathrm{frac}\left(x\right)\)                                                                                                                         & Fractional part, i.e., \(x\;\mathrm{mod}\;1\) \cite{grahamConcreteMathematicsFoundation1989}                                \\ \hline
	\(a \backslash b\) \cite[Section 4.1]{grahamConcreteMathematicsFoundation1989}, \(a \mid b\) \cite{rosenDiscreteMathematicsIts2011}                     & \(b\) is a positive integer multiple of \(a \in \mathbb{Z}\), i.e., \( \exists!\; n \in \mathbb{Z}_{++} \mid b = n a \)     \\ \hline
	\(a \centernot\backslash b\) \cite[Section 4.1]{grahamConcreteMathematicsFoundation1989}, \(a \centernot\mid b\) \cite{rosenDiscreteMathematicsIts2011} & \(b\) is not a positive integer multiple of \(a \in \mathbb{Z}\), i.e., \( \nexists\; n \in \mathbb{Z}_{++} \mid b = n a \) \\ \hline
	\(\ceil{\cdot}\)                                                                                                                                        & Ceiling operation \cite{grahamConcreteMathematicsFoundation1989}                                                            \\ \hline
	\(\floor{\cdot}\)                                                                                                                                       & Floor operation \cite{grahamConcreteMathematicsFoundation1989}
\end{xltabular}

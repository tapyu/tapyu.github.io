\section{Linear Algebra}
\subsection{Common matrices and vectors}
\begin{xltabular}{\textwidth}{XX}
	\(\mathbf{W}, \mathbf{D}\)                  & Diagonal matrix                                                       \\ \hline
	\(\mathbf{P}\)                              & Projection matrix; Permutation matrix                                 \\ \hline
	\(\mathbf{S}\)                              & Symmetric matrix                                                      \\ \hline
	\(\mathbf{J}\)                              & Jordan matrix                                                         \\ \hline
	\(\mathbf{L}\)                              & Lower matrix                                                          \\ \hline
	\(\mathbf{U}\)                              & Upper matrix; Unitary matrix                                          \\ \hline
	\(\mathbf{C}\)                              & Cofactor matrix                                                       \\ \hline
	\(\mathbf{C}_\mathbf{A}, \cof{\mathbf{A}}\) & Cofactor matrix of \(\mathbf{A}\)                                     \\ \hline
	\(\mathbf{S}\)                              & Symmetric matrix                                                      \\ \hline
	\(\mathbf{Q}\)                              & Orthogonal matrix                                                     \\ \hline
	\(\mathbf{I}_N\)                            & \(N\times N\)-dimensional identity matrix                             \\ \hline
	\(\mathbf{0}_{M\times N}\)                  & \(M\times N\)-dimensional null matrix                                 \\ \hline
	\(\mathbf{0}_{N}\)                          & \(N\)-dimensional null vector                                         \\ \hline
	\(\mathbf{1}_{M\times N}\)                  & \(M\times N\)-dimensional ones matrix                                 \\ \hline
	\(\mathbf{1}_{N}\)                          & \(N\)-dimensional ones vector                                         \\ \hline
	\(\mathbf{0}\)                              & Null matrix, vector, or tensor (dimensionality understood by context) \\ \hline
	\(\mathbf{1}\)                              & Ones matrix, vector, or tensor (dimensionality understood by context) \\
\end{xltabular}

\subsection{Indexing}
\begin{xltabular}[l]{\linewidth}{XX}
	\(x_{i_1,i_2, \dots, i_N}, \left[ \bm{\mathcal{X}} \right]_{i_1,i_2, \dots, i_N}\) & Element in the position \((i_1,i_2, \dots, i_N)\) of the tensor \(\bm{\mathcal{X}}\) \\ \hline
	\(\bm{\mathcal{X}}^{(n)}\)                                                         & \(n\)th tensor of a nontemporal sequence                                             \\ \hline
	\(\mathbf{x}_{n}, \mathbf{x}_{:n}\)                                                & \(n\)th column of the matrix \(X\)                                                   \\ \hline
	\(\mathbf{x}_{n:}\)                                                                & \(n\)th row of the matrix \(X\)                                                      \\ \hline
	\(\mathbf{x}_{i_1,\dots,i_{n-1}, :, i_{n+1},\dots, i_N}\)                          & Mode-\(n\) fiber of the tensor \(\bm{\mathcal{X}}\)                                  \\ \hline
	\(\mathbf{x}_{:,i_2,i_3}\)                                                         & Column fiber (mode-\(1\) fiber) of the thrid-order tensor \(\bm{\mathcal{X}}\)       \\ \hline
	\(\mathbf{x}_{i_1,:,i_3}\)                                                         & Row fiber (mode-\(2\) fiber) of the thrid-order tensor \(\bm{\mathcal{X}}\)          \\ \hline
	\(\mathbf{x}_{i_1,i_2,:}\)                                                         & Tube fiber (mode-\(3\) fiber) of the thrid-order tensor \(\bm{\mathcal{X}}\)         \\ \hline
	\(\mathbf{X}_{i_1,:,:}\)                                                           & Horizontal slice of the thrid-order tensor \(\bm{\mathcal{X}}\)                      \\ \hline
	\(\mathbf{X}_{:,i_2,:}\)                                                           & Lateral slices slice of the thrid-order tensor \(\bm{\mathcal{X}}\)                  \\ \hline
	\(\mathbf{X}_{i_3}, \mathbf{X}_{:,:,i_3}\)                                         & Frontal slices slice of the thrid-order tensor \(\bm{\mathcal{X}}\)
\end{xltabular}

\subsection{General operations}
\begin{xltabular}{\textwidth}{XX}
	\(\expval{\mathbf{a}, \mathbf{b}}, \mathbf{a}^\top\mathbf{b}, \mathbf{a}\cdot\mathbf{b}\) & Inner or dot product                       \\ \hline
	\(\mathbf{a}\circ\mathbf{b}, \mathbf{a}\mathbf{b}^\top\)                                  & Outer product                              \\ \hline
	\(\otimes\)                                                                               & Kronecker product                          \\ \hline
	\(\odot\)                                                                                 & Hadamard (or Schur) (elementwise) product  \\ \hline
	\(\cdot^{\odot n}\)                                                                       & \(n\)th-order Hadamard power               \\ \hline
	\(\cdot^{\odot \frac{1}{n}}\)                                                             & \(n\)th-order Hadamard root                \\ \hline
	\(\oslash\)                                                                               & Hadamard (or Schur) (elementwise) division \\ \hline
	\(\diamond\)                                                                              & Khatri-Rao product                         \\ \hline
	\(\otimes\)                                                                               & Kronecker Product                          \\ \hline
	\(\times_n\)                                                                              & \(n\)-mode product                         \\
\end{xltabular}

\subsection{Operations with matrices and tensors}
\begin{xltabular}{\textwidth}{XX}
	\(\mathbf{A}^{-1}\)                                                                                  & Inverse matrix                                                                                                                                                                  \\ \hline
	\(\mathbf{A}^+, \mathbf{A}^{\dagger}\)                                                               & Moore-Penrose left pseudoinverse                                                                                                                                                \\ \hline
	\(\mathbf{A}^\top, \mathbf{A}^T, \mathbf{A}^t, \mathbf{A}^{'}\) \cite{searleMatrixAlgebraUseful2017} & Transpose                                                                                                                                                                       \\ \hline
	\(\mathbf{A}^{-\top}\)                                                                               & Transpose of the inverse, i.e., \(\left( \mathbf{A}^{-1} \right)^{\top} = \left( \mathbf{A}^{\top} \right)^{-1}\) \cite{petersenMatrixCookbook2008,golubMatrixComputations2013} \\ \hline
	\(\mathbf{A}^*\)                                                                                     & Complex conjugate                                                                                                                                                               \\ \hline
	\(\mathbf{A}^\mathsf{H}\)                                                                            & Hermitian                                                                                                                                                                       \\ \hline
	\(\frob{\mathbf{A}}\)                                                                                & Frobenius norm                                                                                                                                                                  \\ \hline
	\(\norm{\mathbf{A}}\)                                                                                & Matrix norm                                                                                                                                                                     \\ \hline
	\(\abs{\mathbf{A}}, \textnormal{det}\left( \mathbf{A} \right)\)                                      & Determinant                                                                                                                                                                     \\ \hline
	\(\diag{\mathbf{A}}\)                                                                                & The elements in the diagonal of \(\mathbf{A}\)                                                                                                                                  \\ \hline
	\(\vec[]{\mathbf{A}}\)                                                                               & Vectorization: stacks the columns of the matrix \(\mathbf{A}\) into a long column vector                                                                                        \\ \hline
	\(\vec[d]{\mathbf{A}}\)                                                                              & Extracts the diagonal elements of a square matrix and returns them
	in a column vector                                                                                                                                                                                                                                                                     \\ \hline
	\(\vec[l]{\mathbf{A}}\)                                                                              & Extracts the elements strictly below the main diagonal of a square matrix in a column-wise manner and returns them into a column vector                                         \\ \hline
	\(\vec[u]{\mathbf{A}}\)                                                                              & Extracts the elements strictly above the main diagonal of a square matrix in a column-wise manner and returns them into a column vector                                         \\ \hline
	\(\vec[b]{\mathbf{A}}\)                                                                              & Block vectorization operator: stacks square block matrices of the input into a long block column matrix                                                                         \\ \hline
	\(\unvec{\mathbf{A}}\)                                                                               & Reshapes a column vector into a matrix                                                                                                                                          \\ \hline
	\(\tr{\mathbf{A}}\)                                                                                  & trace                                                                                                                                                                           \\ \hline
	\(\mathbf{X}_{(n)}\)                                                                                 & \(n\)-mode matricization of the tensor \(\bm{\mathcal{X}}\)                                                                                                                     \\
\end{xltabular}
\subsection{Operations with vectors}
\begin{xltabular}{\textwidth}{XX}
	\(\norm{\mathbf{a}}\)                      & \(l_1\) norm, 1-norm, or Manhattan norm                                                    \\ \hline
	\(\norm{\mathbf{a}}, \norm{\mathbf{a}}_2\) & \(l_2\) norm, 2-norm, or Euclidean norm                                                    \\ \hline
	\(\norm{\mathbf{a}}_p\)                    & \(l_p\) norm, \(p\)-norm, or Minkowski norm                                                \\ \hline
	\(\norm{\mathbf{a}}_\infty\)               & \(l_\infty\) norm, \(\infty\)-norm, or Chebyshev norm                                      \\ \hline
	\(\diag{\mathbf{a}}\)                      & Diagonalization: a square, diagonal matrix with entries given by the vector \(\mathbf{a}\) \\
\end{xltabular}

\subsection{Decompositions}
\begin{xltabular}{\textwidth}{XX}
	\(\boldsymbol{\Lambdaup}\)                                                                  & Eigenvalue matrix \cite{strangIntroductionLinearAlgebra1993}                                                                                                                         \\ \hline
	\(\mathbf{Q}\)                                                                              & Eigenvectors matrix; Orthogonal matrix of the QR decomposition\cite{strangIntroductionLinearAlgebra1993}                                                                             \\ \hline
	\(\mathbf{R}\)                                                                              & Upper triangular matrix of the QR decomposition\cite{strangIntroductionLinearAlgebra1993}                                                                                            \\ \hline
	\(\mathbf{U}\)                                                                              & Left singular vectors\cite{strangIntroductionLinearAlgebra1993}                                                                                                                      \\ \hline
	\(\mathbf{U}_r\)                                                                            & Left singular nondegenerated vectors                                                                                                                                                 \\ \hline
	\(\boldsymbol{\Sigmaup}\)                                                                   & Singular value matrix                                                                                                                                                                \\ \hline
	\(\boldsymbol{\Sigmaup}_r\)                                                                 & Singular value matrix with nonzero singular values in the main diagonal                                                                                                              \\ \hline
	\(\boldsymbol{\Sigmaup}^{+}\)                                                               & Singular value matrix of the pseudoinverse \cite{strangIntroductionLinearAlgebra1993}                                                                                                \\ \hline
	\(\boldsymbol{\Sigmaup}^{+}_r\)                                                             & Singular value matrix of the pseudoinverse with nonzero singular values in the main diagonal                                                                                         \\ \hline
	\(\mathbf{V}\)                                                                              & Right singular vectors \cite{strangIntroductionLinearAlgebra1993}                                                                                                                    \\ \hline
	\(\mathbf{V}_r\)                                                                            & Right singular nondegenerated vectors                                                                                                                                                \\ \hline
	\(\eig{\mathbf{A}}\)                                                                        & Set of the eigenvalues of \(\mathbf{A}\) \cite{chellappaSignalProcessingTheory2014,leon-garciaProbabilityStatisticsRandom2007,petersenMatrixCookbook2008}                            \\ \hline
	\(\llbracket \mathbf{A}, \mathbf{B}, \mathbf{C}, \dots \rrbracket\)                         & CANDECOMP/PARAFAC (CP) decomposition of the tensor \(\bm{\mathcal{X}}\) from the outer product of column vectors of \(\mathbf{A}\), \(\mathbf{B}\), \(\mathbf{C}, \dots\)            \\ \hline
	\(\llbracket \boldsymbol{\lambdaup}; \mathbf{A}, \mathbf{B}, \mathbf{C}, \dots \rrbracket\) & Normalized CANDECOMP/PARAFAC (CP) decomposition of the tensor \(\bm{\mathcal{X}}\) from the outer product of column vectors of \(\mathbf{A}\), \(\mathbf{B}\), \(\mathbf{C}, \dots\) \\
\end{xltabular}
\subsection{Spaces and sets}
\subsubsection{Common spaces and sets}
\begin{xltabular}{\textwidth}{XX}
	\(\mathbb{R}\)                                                                                                                                          & Set of real numbers                                                                                                                                                                                                                                 \\ \hline
	\([a, b]\)                                                                                                                                              & Closed interval of a real set from \(a\) to \(b\)                                                                                                                                                                                                   \\ \hline
	\((a, b)\)                                                                                                                                              & Opened interval of a real set from \(a\) to \(b\)                                                                                                                                                                                                   \\ \hline
	\([a, b), (a, b]\)                                                                                                                                      & Half-opened intervals of a real set from \(a\) to \(b\)                                                                                                                                                                                             \\ \hline
	\(\mathbb{C}\)                                                                                                                                          & Set of complex numbers                                                                                                                                                                                                                              \\ \hline
	\(\mathbb{I}, j\mathbb{R}\)                                                                                                                             & Set of imaginary numbers                                                                                                                                                                                                                            \\ \hline
	\(\mathbb{Q}\)                                                                                                                                          & Set of rational number                                                                                                                                                                                                                              \\ \hline
	\(\mathbb{R} \setminus \mathbb{Q}\)                                                                                                                     & Set of irrational number                                                                                                                                                                                                                            \\ \hline
	\(\mathbb{Z}\)                                                                                                                                          & Set of integer number                                                                                                                                                                                                                               \\ \hline
	\(\mathbb{N}\)                                                                                                                                          & Set of natural numbers                                                                                                                                                                                                                              \\ \hline
	\(\left\{ 1,2, \dots, n \right\}\)                                                                                                                      & Discrete set containing the integer elements \(1,2, \dots, n\)                                                                                                                                                                                      \\ \hline
	\(\mathbb{B} = \left\{ 0, 1 \right\}\)                                                                                                                  & Boolean set                                                                                                                                                                                                                                         \\ \hline % Circuit Complexity and Neural Networks - Ian Parberry; Further Improvements in the Boolean Domain
	\(\emptyset\)                                                                                                                                           & Empty set                                                                                                                                                                                                                                           \\ \hline
	\(\mathbb{K} \in \left\{ \mathbb{R}, \mathbb{C} \right\}\)                                                                                              & Real or complex space (field)                                                                                                                                                                                                                       \\ \hline
	\(\mathbb{K}^{I_1\times I_2 \times \dots \times I_N}\)                                                                                                  & \(I_1\times I_2 \times \dots \times I_N\)-dimensional real (or complex) space                                                                                                                                                                       \\ \hline
	\(\mathbb{K}_{+}^{I_1\times I_2 \times \dots \times I_N}\) \cite{boydConvexOptimization2004} \cite[sec. 2.1.3]{dattorroConvexOptimizationEuclidean2010} & Nonnegative real (or complex) orthant. The name orthant is the higher-dimensional generalization of the term \emph{quadrant} from the classical Cartesian partition of \(\mathbb{R}^{2}\) \cite[sec 2.1.3]{dattorroConvexOptimizationEuclidean2010} \\ \hline
	\(\mathbb{K}_{-}^{I_1\times I_2 \times \dots \times I_N}\) \cite{boydConvexOptimization2004} \cite[sec. 2.1.3]{dattorroConvexOptimizationEuclidean2010} & Same, but for nonpositive real (or complex) orthant.                                                                                                                                                                                                \\ \hline
	\(\mathbb{K}_{++}^{I_1\times I_2 \times \dots \times I_N}\)                                                                                             & Positive real (or complex) orthant, i.e., \(\mathbb{K}_{++} = \mathbb{K}_{+}\setminus\left\{ \mathbf{0} \right\}\) \cite{boydConvexOptimization2004}                                                                                                \\ \hline
	\(\mathbb{K}_{--}^{I_1\times I_2 \times \dots \times I_N}\)                                                                                             & Negative real (or complex) orthant, i.e., \(\mathbb{K}_{++} = \mathbb{K}_{+}\setminus\left\{ \mathbf{0} \right\}\) \cite{boydConvexOptimization2004}                                                                                                \\ \hline
	\(U\)                                                                                                                                                   & Universe                                                                                                                                                                                                                                            \\ \hline
	\(2^A\)                                                                                                                                                 & Power set of \(A\)                                                                                                                                                                                                                                  \\ \hline
\end{xltabular}

\subsubsection{Convex sets (or spaces)}
\begin{xltabular}{\textwidth}{XX}
	\(\mathbb{S}^{n}\) \cite[sec. 2.2.2]{dattorroConvexOptimizationEuclidean2010}, \(\mathcal{S}^{n}\) \cite[sec. 1.6]{boydConvexOptimization2004}                        & Conic set (see \cite[p. 35]{boydConvexOptimization2004}) of the symmetric matrices in \(\mathbb{R}^{n\times n}\)                                                                                               \\ \hline
	\(\mathbb{S}^{n\perp}\) \cite[sec. 2.2.2]{dattorroConvexOptimizationEuclidean2010}                                                                                    & Conic set of the skew-symmetric (also called antisymmetric) matrices in \(\mathbb{R}^{n\times n}\)                                                                                                             \\ \hline
	\(\mathbb{S}_{+}^{n}\), \(\mathcal{S}_{+}^{n}\) \cite[sec. 2.2.5]{boydConvexOptimization2004}                                                                         & Conic set of the symmetric positive semidefinite matrices in \(\mathbb{R}^{n\times n}\) \cite{boydConvexOptimization2004}                                                                                      \\ \hline
	\(\mathbb{S}_{++}^{n}, \mathcal{S}_{++}^{n}\)   \cite[sec. 2.2.5]{boydConvexOptimization2004}                                                                         & Conic set of the symmetric positive definite matrices in \(\mathbb{R}^{n\times n}\), i.e., \(\mathbb{S}_{++}^{n} = \mathbb{S}_{+}^{n}\setminus \left\{ \mathbf{0} \right\}\) \cite{boydConvexOptimization2004} \\ \hline
	\(\mathbb{H}^{n}\) (?)                                                                                                                                                & Set of all hermitian matrices in \(\mathbb{C}^{n\times n}\)                                                                                                                                                    \\ \hline
	\(\textnormal{conv } A\) \cite[p. 34]{boydConvexOptimization2004}                                                                                                     & Convex hull of the set \(A\)                                                                                                                                                                                   \\ \hline
	\(\textnormal{aff } A\) \cite[p. 23]{boydConvexOptimization2004}                                                                                                      & Affine hull of the set \(A\)                                                                                                                                                                                   \\ \hline
	\(\partial A\) \cite[sec. 2.1.7]{dattorroConvexOptimizationEuclidean2010} \(\textnormal{bd } A\) \cite[appendix A.2]{boydConvexOptimization2004}                                                                                             & Boundary of the set \(A\)                                                                                                                                                                                      \\ \hline
	\(\textnormal{int } A\) \cite[sec. 2.1.6.1]{dattorroConvexOptimizationEuclidean2010} \cite[2.1.3]{boydConvexOptimization2004}                                         & Interior of the set \(A\)                                                                                                                                                                                      \\ \hline
	\(\textnormal{rel int } A\) \cite[sec. 2.1.6.1]{dattorroConvexOptimizationEuclidean2010} \newline \(\textnormal{relint } A\) \cite[2.1.3]{boydConvexOptimization2004} & Relative interior of the set \(A\)                                                                                                                                                                             \\ \hline
	\(\textnormal{cl } A\) \cite[appendix A.2]{boydConvexOptimization2004} \newline \(\bar{A}\) \cite[sec. 2.1.6.1]{dattorroConvexOptimizationEuclidean2010}              & Closure of \(A\)                                                                                                                                                                                               \\ \hline
	\(\mathcal{R}\) (?)                                                                                                                                                   & Ray                                                                                                                                                                                                            \\ \hline
	\(\mathcal{H}\) (?)                                                                                                                                                   & Hyperplane                                                                                                                                                                                                     \\ \hline
	\(\mathcal{H}_{+}, \mathcal{H}_{-}\) \cite[sec. 2.4]{dattorroConvexOptimizationEuclidean2010}                                                                         & Positive/negative halfspace                                                                                                                                                                                    \\ \hline
	\(B(\mathbf{x}_c, r)\) \cite[sec. 2.2.2]{boydAdditionalExercisesConvex}                                                                                               & Euclidean ball with radium \(r\) and centered at \(\mathbf{x}_c\)                                                                                                                                              \\ \hline
	\(\mathcal{E}\) \cite[sec. 2.2.2]{boydAdditionalExercisesConvex}                                                                                                      & Ellipsoid                                                                                                                                                                                                      \\ \hline
	\(C\) \cite[sec. 2.2.3]{boydConvexOptimization2004}                                                                                                                   & Norm cone                                                                                                                                                                                                      \\ \hline
	\(K\) \cite[sec. 2.4]{boydAdditionalExercisesConvex}                                                                                                                  & Proper cone                                                                                                                                                                                                    \\ \hline
	\(K^*\) \cite[sec. 2.6]{boydConvexOptimization2004}                                                                                                                   & Dual cone                                                                                                                                                                                                      \\ \hline
	\(\mathcal{P}\) \cite[sec. 2.2.4]{boydConvexOptimization2004}                                                                                                         & Polyhedra                                                                                                                                                                                                      \\ \hline
	\(S\) (?)                                                                                                                                                             & Simplex                                                                                                                                                                                                        \\ \hline
	\(C_\alpha\) \cite[sec. 3.1.6]{boydConvexOptimization2004}                                                                                                            & \(\alpha\)-sublevel set                                                                                                                                                                                        \\ \hline
	\(\textnormal{epi } f\)  \cite[sec. 3.1.7]{boydConvexOptimization2004}                                                                                                & Epigraph of the function \(f\)                                                                                                                                                                                 \\ \hline
	\(\textnormal{hypo } f\) \cite[sec 3.1.7]{boydConvexOptimization2004}                                                                                                 & Hypograph of the function \(f\)
\end{xltabular}

\subsubsection{Spaces from matrices or vectors}
\begin{xltabular}{\textwidth}{XX}
	\(\spn{\mathbf{a}_1, \mathbf{a}_2, \dots, \mathbf{a}_n}\)                                                                                          & Vector space spanned by the argument vectors \cite{golubMatrixComputations2013}                                                                                                                                                                               \\ \hline
	\(\range{\mathbf{A}}\), \(\mathrm{columnspace}(\mathbf{A})\), \(\mathrm{range}(\mathbf{A})\), \(\spn{\mathbf{A}}\), \(\mathrm{image}(\mathbf{A})\) & Columnspace, range or image, i.e., the space \(\spn{\mathbf{a}_1,\mathbf{a}_2, \dots, \mathbf{a}_n}\), where \(\mathbf{a}_i\) is the ith column vector of the matrix \(\mathbf{A}\) \cite{strangIntroductionLinearAlgebra1993, nossekAdaptiveArraySignal2015} \\ \hline
	\(\range{\mathbf{A}^\mathsf{H}}\)                                                                                                                  & Row space (also called left columnspace) \cite{strangIntroductionLinearAlgebra1993, nossekAdaptiveArraySignal2015}                                                                                                                                            \\ \hline
	\(\nullspace{\mathbf{A}}, \mathrm{nullspace}(\mathbf{A}), \mathrm{null}(\mathbf{A}), \mathrm{kernel}(\mathbf{A})\)                                 & Nullspace (or kernel space) \cite{strangIntroductionLinearAlgebra1993, nossekAdaptiveArraySignal2015,theodoridisMachineLearningBayesian2020}                                                                                                                  \\ \hline
	\(\nullspace{\mathbf{A^\mathsf{H}}}\)                                                                                                              & Left nullspace                                                                                                                                                                                                                                                \\ \hline
	\(\rank{\mathbf{A}}\)                                                                                                                              & Rank, that is, \(\dim{\spn{\mathbf{A}}} = \dim{\range{\mathbf{A}}}\) \cite{nossekAdaptiveArraySignal2015}                                                                                                                                                     \\ \hline
	\(\nullity{\mathbf{A}}\)                                                                                                                           & Nullity of \(\mathbf{A}\), i.e., \(\dim{\nullspace{\mathbf{A}}}\)                                                                                                                                                                                             \\ \hline
\end{xltabular}

\subsection{Set operations}
\begin{xltabular}{\textwidth}{XX}
	\(A + B\)                           & Set addition (Minkowski sum), i.e., \(\left\{ \mathbf{v} \in \mathbb{R}^{n} \mid \mathbf{v} = \mathbf{x}+\mathbf{y}, \forall \; \mathbf{x} \in \mathcal{X} \wedge \mathbf{y} \in \mathcal{Y} \right\}\) \cite{kouvaritakisModelPredictiveControl2016}                                                                                             \\ \hline
	\(A - B\)                           & Minkowski difference, i.e., \(\left\{ \mathbf{v} \in \mathbb{R}^{n} \mid \mathbf{v} = \mathbf{x}-\mathbf{y}, \forall \; \mathbf{x} \in \mathcal{X} \wedge \mathbf{y} \in \mathcal{Y} \right\}\)                                                                                                                                                   \\ \hline
	\(A \ominus B\)                     & Pontryagin difference, i.e., \(\left\{ \mathbf{v} \in \mathbb{R}^{n} \mid \mathbf{v} + \mathbf{y} \in \mathcal{X} , \forall \; \mathbf{y} \in \mathcal{Y} \right\}\) \cite{kouvaritakisModelPredictiveControl2016}                                                                                                                                \\ \hline
	\(A \setminus B, A-B\)              & Set difference or set subtraction, i.e., \(A \setminus B = \left\{ x \vert x \in A \wedge x \not\in B \right\}\) the set containing the elements of \(A\) that are not in \(B\) \cite{rosenDiscreteMathematicsIts2011}                                                                                                                            \\ \hline
	\(A \cup B\)                        & Set of union                                                                                                                                                                                                                                                                                                                                      \\ \hline
	\(A \cap B\)                        & Set of intersection                                                                                                                                                                                                                                                                                                                               \\ \hline
	\(A \times B\)                      & Cartesian product                                                                                                                                                                                                                                                                                                                                 \\ \hline
	\(A^n\)                             & \(\underbrace{A \times A \times \dots \times A}_{n \text{ times}}\)                                                                                                                                                                                                                                                                               \\ \hline
	\(A^{\perp}\)                       & Orthogonal complement of \(A\), e.g., \(\nullspace{\mathbf{A}} = \range{\mathbf{A}^{\top}}^{\perp}\) \cite{boydConvexOptimization2004}                                                                                                                                                                                                            \\ \hline
	\(\mathbf{a} \perp \mathbf{b}\)     & \(\mathbf{a}\) is orthogonal to \(\mathbf{b}\)                                                                                                                                                                                                                                                                                                    \\ \hline
	\(\mathbf{a} \not\perp \mathbf{b}\) & \(\mathbf{a}\) is not orthogonal to \(\mathbf{b}\)                                                                                                                                                                                                                                                                                                \\ \hline
	\(A \oplus B\)                      & Direct sum, i.e., each \(\mathbf{v} \in \left\{ \sum \mathbf{a}_i \mid \mathbf{a}_i \in S_i, i=1,\dots,k \right\}\) has a unique representation of \(\sum \mathbf{a}_i\) with \(\mathbf{a}_i \in S_i\). That is, they expand to a space. Note that \(\left\{ S_i \right\}\) might not be orthogonal each other \cite{golubMatrixComputations2013} \\ \hline
	\(A \overset{\perp}{\oplus} B\)     & Direct sum of two spaces that are orthogonal and span a \(n\)-dimensional space, e.g., \(\range{\mathbf{A}^{\top}} \overset{\perp}{\oplus} \range{\mathbf{A}^{\top}}^{\perp} = \mathbb{R}^{n}\) (this decomposition of \(\mathbb{R}^{n}\) is called the orthogonal decomposition induced by \(\mathbf{A}\)) \cite{boydConvexOptimization2004}     \\ \hline
	\(\bar{A}, A^{c}\)                  & Complement set (given $U$)                                                                                                                                                                                                                                                                                                                        \\ \hline
	\(\#A, \abs{A}\)                    & Cardinality of \(A\)                                                                                                                                                                                                                                                                                                                              \\ \hline
	\(a \in A\)                         & \(a\) is element of \(A\)                                                                                                                                                                                                                                                                                                                         \\ \hline
	\(a \notin A\)                      & \(a\) is not element of \(A\)                                                                                                                                                                                                                                                                                                                     \\ \hline
\end{xltabular}

\subsection{Inequalities}
\begin{xltabular}{\textwidth}{XX}
	\(\bm{\mathcal{X}} \leq 0\)         & Nonnegative tensor                                                                                                                                                                                           \\ \hline
	\(\mathbf{a} \preceq_K \mathbf{b}\) & Generalized inequality meaning that \(\mathbf{b}-\mathbf{a}\) belongs to the conic subset \(K\) in the space \(\mathbb{R}^{n}\)\cite{boydConvexOptimization2004}                                             \\ \hline
	\(\mathbf{a} \prec_K \mathbf{b}\)   & Strict generalized inequality meaning that \(\mathbf{b}-\mathbf{a}\) belongs to the interior of the conic subset \(K\) in the space \(\mathbb{R}^{n}\)\cite{boydConvexOptimization2004}                      \\ \hline
	\(\mathbf{a} \preceq \mathbf{b}\)   & Generalized inequality meaning that \(\mathbf{b}-\mathbf{a}\) belongs to the nonnegative orthant conic subset, \(\mathbb{R}_{+}^{n}\), in the space \(\mathbb{R}^{n}\).\cite{boydConvexOptimization2004}     \\ \hline
	\(\mathbf{a} \prec \mathbf{b}\)     & Strict generalized inequality meaning that \(\mathbf{b}-\mathbf{a}\) belongs to the positive orthant conic subset, \(\mathbb{R}_{++}^{n}\), in the space \(\mathbb{R}^{n}\)\cite{boydConvexOptimization2004} \\ \hline
	\(\mathbf{A} \preceq_K \mathbf{B}\) & Generalized inequality meaning that \(\mathbf{B}-\mathbf{A}\) belongs to the conic subset \(K\) in the space \(\mathbb{S}^{n}\)\cite{boydConvexOptimization2004}                                             \\ \hline
	\(\mathbf{A} \prec_K \mathbf{B}\)   & Strict generalized inequality meaning that \(\mathbf{B}-\mathbf{A}\) belongs to the interior of the conic subset \(K\) in the space \(\mathbb{S}^{n}\)\cite{boydConvexOptimization2004}                      \\ \hline
	\(\mathbf{A} \preceq \mathbf{B}\)   & Generalized inequality meaning that \(\mathbf{B}-\mathbf{A}\) belongs to the positive semidefinite conic subset, \(\mathbb{S}_{+}^{n}\), in the space \(\mathbb{S}^{n}\)\cite{boydConvexOptimization2004}    \\ \hline
	\(\mathbf{A} \prec \mathbf{B}\)     & Strict generalized inequality meaning that \(\mathbf{B}-\mathbf{A}\) belongs to the positive orthant conic subset, \(\mathbb{S}_{++}^{n}\), in the space \(\mathbb{S}^{n}\)\cite{boydConvexOptimization2004}
\end{xltabular}